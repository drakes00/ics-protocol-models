\documentclass[11pt]{article}
%\documentclass{IEEEtran}

%\usepackage{fullpage}
%\usepackage[left=3.8cm,right=3.8cm]{geometry}

\usepackage[T1]{fontenc}
\usepackage[utf8]{inputenc}

\usepackage[french]{babel}

\usepackage{authblk}

\usepackage{listings}
\usepackage{caption}
\usepackage{subcaption}

\usepackage{graphicx}
\graphicspath{{assets/}}
\makeatletter
    \def\input@path{{assets/}}
\makeatother

\usepackage{xspace}
\newcommand{\aramis}{ARAMIS\xspace}

\title{TODO titre: modèles d'attaques}

\author{Maxime Puys}
\author{Marie-Laure Potet}

\affil{
    Univ. Grenoble Alpes, VERIMAG, F-38000 Grenoble, France\\
    CNRS, VERIMAG, F-38000 Grenoble, France
    \thanks{Ce travail a été partiellement financé par le LabEx PERSYVAL-Lab (ANR–11-LABX-0025) et le projet Programme Investissement d’Avenir FSN AAP Sécurité Numérique \no 3 ARAMIS (P3342-146798).}
}

\date{}

\begin{document}

\maketitle

\begin{abstract}
\end{abstract}

\section{Introduction (1p)}

\subsection{Context}

Contexte industriel, s'aider du discours des slides:
\begin{itemize}
    \item Attaques récentes (blackout Ukraine, smart-grid américain, ...)
    \item Publications lois, référentiels, normes, ...
    \item Safety vs. Security
    \item Differences avec informatique de gestion
    \item Citer ARAMIS du coup ?
\end{itemize}

\subsection{State-of-the-art}

Pas de parallèle avec les méthodes type EBIOS (ajoue de la confusion ?) ?

Revoir S-CUBE.
\begin{itemize}
    \item Revoir leur biblio ?
\end{itemize}

Article Caire+Cochon à C\&ESAR.
\begin{itemize}
    \item Revoir leur biblio ?
\end{itemize}

\subsection{Objectives}

Model a SCADA system and its threats.

Model the security properties of protocols.

Deduce attack scenarios.

Generate corresponding traffic.

Si ARAMIS cité dans le contexte:
\begin{itemize}
    \item Expliquer que utile pour le test du module.
\end{itemize}

\section{Approach (formal) (2p)}

Reprendre les slides modèles d'attaques avec formalisation (git aramis-doc)

Renommer les éléments pour coller avec le livrable (modèles d'attaques en capacités, ...)

Ajouter la formalisation de la réalisation des attaques.

\section{Example (2p)}

Exemple ARAMIS (si éviter ARAMIS, on peut retirer le boitier au milieu,
il n'apporte rien en tant que tel).

Protocoles MODBUS, FTP, OPC-UA (autres ?)

On cite le nombre de scénarios pouvant être générés par cet exemple.

On donne une liste non-exhaustive (cf. livrable).

\section{Conclusion (1p)}

Reprendre objectifs

\subsection{Perspectives}

Génération de trafic.

\bibliographystyle{plain}
\bibliography{phdBiblio}

\end{document}

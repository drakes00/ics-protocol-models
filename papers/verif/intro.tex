%Contexte industriel, s'aider du discours des slides:
%\begin{itemize}
%    \item Attaques récentes (blackout Ukraine, smart-grid américain, ...)
%    \item Publications lois, référentiels, normes, ...
%    \item Safety vs. Security
%    \item Differences avec informatique de gestion
%\end{itemize}

Industrial systems also called SCADA (Supervisory Control And Data
Acquisition) have been known to be targeted by cyberattacks since the
famous Stuxnet case~\cite{Lan11} in 2010.  Due to the criticity of
their interaction with the physical world, these systems can
potentially be really harmful for humans and environment.  The
frequence of such attacks is increasing.  The security of industrial
systems is now one of the priorities for governmental agencies
(\eg~\cite{SFS11} from the US National Institute of Standards and
Technology (NIST) or
\cite{ANSSI12_guide_securite_industrielle_en} from the French {\em Agence
nationale de la sécurité des systèmes d'information} (ANSSI)).  As industrial
systems historically have been physically isolated from the rest of
the world, there were less exposed to cyber attacks and their security
was less considered. Recently, such attacks become feasible because
these systems are recently spreading geographically and communicating
more and more through unsafe mediums like Internet.


Industrial systems have specificities. First the isolation of such
systems requires the attacker to be physically present in the
system. Then compagnies focused more on the protection against natural
deceases and human mistakes than cybersecurity (often call security).
In the context of security there is an adversary willing to perform malicious
actions.  Industrial systems also differ from other systems because of
the really long lifetime of its devices.  This makes them hard to
patch in case of vulnerabilities. It also explain the number of legacy
hosts. Moreover, most of industrial protocols are proprietary and
provide a very low level of security if
any, \eg \modbus~\cite{MODBUS}, \profinet~\cite{PROFINET}, \etherip~\cite{Bro01}
or
\dnp~\cite{CR04}.
However, in 2006, the {\em OPC Foundation} (an industry consortium)
released the first version of \opcua~\cite{MLD09}, which is presented
as the new standard for industrial communications. %% This standard  and whose security
%% is quite closer to business IT's protocols such as TLS~\cite{DR08}.

As it already appeared for business IT's protocols for twenty years, automated
verification is crucial in order to discover flaws in the protocols'
specifications before assessing implementations.  This was emphasized
in 2006 by Igure et al.~\cite{ILW06} and in 2009 by Patel \emph{et
al.}~\cite{PBG09}, saying that automated protocol verification allows
to understand most of the vulnerabilities of a protocol before
changing its standards which helps at minimizing the number of
revisions which costs time and money.  Moreover, due to the
combinational explosion of the number of possible execution traces,
only automated verification can ensure the security of a protocol in a
given model.


%\subsection{Contributions}
\paragraph{Contributions}

We propose a formal study of the security of the standard \modbus protocol and
the correction proposed in~\cite{FCMT09}.%,HE13}.
We also assess the sub-protocols involved in the \opcua handshake, namely \opcua
OpenSecureChannel and \opcua CreateSession to either formaly prove their
security or exhibit attacks against them.
This work makes use of one of the best tools in the domain of cryptographic
protocol verification according to \cite{LP15}: \proverif developed by Blanchet
et al.~\cite{Bla01}.
It analyzes a protocol written either in Horn clauses format or using a subset
of the Pi-calculus for an unbounded number of sessions considering the
classical Dolev-Yao intruder model~\cite{DY81}.
This powerful intruder controls the network, listens, stops, forges, replays or
modifies some messages according to its capabilities and knowledge.
The perfect encryption hypothesis is often assumed, meaning that it is not
possible to decrypt a ciphertext without its encryption key or to flaw
the signature of a message.
In this context, verification tools are able to verify security properties
of a protocol such as secrecy and authentication.
The first property ensures that a secret message cannot be discovered by an
unauthorized agent (including the intruder).
The authentication property means that one participant of the protocol is
guaranteed to communicate with another one.


%\subsection{State-of-the-art}\label{sec:intro_sota}
\paragraph{State-of-the-art:}\label{sec:intro_sota}

The security of industrial protocols becomes a hot topic and some works are
present in the literature.
On one hand, most of them analyze the security of existing industrial protocols.
In 2004, Dzung et al. propose a detailed survey on the security in SCADA systems
including analysis on the security properties offered by \opc (ancestor of
\opcua), \mms~\cite{ISO_MMS}, \iec{61850}~\cite{IEC_61850}, \iccp
\cite{IEC_60870} and \etherip).
Still in 2004, Clarke et al.~\cite{CR04} discuss the security of \dnp and
\iccp.
In 2006, in the technical documentation of \opcua (specialy part 2
\cite{opcua_part2}, 4~\cite{opcua_part4} and 6~\cite{opcua_part6}) the authors
detail the security measures of the protocol and the contents of the messages.
In 2015, Wanying et al. summarize the security offered by \modbus, \dnp and \opc
and introduce industrial firewall from Tofino~\cite{TOFINO}

On the other hand, some propose new versions of existing protocols to make them
secure against malicious adversaries.
In 2007, Patel et al.~\cite{PY07} study the security of \dnp and propose two
ways of enhancing it through digital signatures and challenge-response models.
In 2009, Fovino et al.~\cite{FCMT09} proposed a secure version of \modbus
relying on well-known cryptographic primitives such as RSA~\cite{RSA78} or SHA2.
This version of the protocol also need to introduce new components in the system
to allow existing devices to use these cryptographic primitives.
In 2013, Hayes et al.~\cite{HE13} designed another secure \modbus protocol using
hash-based message authentication codes (HMACs) and built on the Stream
Transmission Control Protocol~\cite{Ste07} (STCP).

To the best of our knowledge, none of these works directly make use of formal
methods to prove the security of industrial protocols or find attack against
them.
However in 2005, Graham et al.~\cite{GP05} proposed a formal verification of
\dnp using \ofmc~\cite{BMV03} and \spear~\cite{SH01}.
\TODO Sadly we were not able to access their modelings to discuss them with
ours.
In 2008, Dutertre~\cite{Dut08} detailed formal specifications of \modbus
developed using PVS, a generic theorem prover in order to help proving the
consistancy of an implementation with the standards.

\paragraph{Outline:} In the next section we study the security of  MODBUS. In Section~\ref{sec:opcua}, we analyse the security of OPC-UA, before concluding in last section.

In conclusion, we provide a formal verification of the new industry standard
communication protocol: \opcua, relying its official specifications
\cite{MLD09,opcua_part2,opcua_part4,opcua_part6}.
It is a tidious task since specifications are willingly elusive to allow
interoperability.
This work makes use of a famous tool in the domain of protocol verification:
\proverif.
We are able to find attacks on both \securechan and \session sub-protocols and
provide counter-measures.
In the future, we plan on testing the attacks we found on real implementations.
This next work would analyze if those implementations are strictly following
the specifications or if they adopt the prudent engineering practices advised in
\cite{AN96} that would circumvent the attacks we found.
More generaly we are also interested in studying how to model integrity
properties for messages and communication flows in \proverif since it is one of
the biggest requirement for industrial protocols.

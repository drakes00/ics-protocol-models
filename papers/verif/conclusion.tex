We provide a formal verification of the new industry standard
communication protocol: \opcua, relying its official specifications
\cite{MLD09,opcua_part2,opcua_part4,opcua_part6}.
This work makes use of \proverif{} a tool for automatic cryptographic protocol
verification. 
Protocol modelings were tedious tasks since specifications are willingly elusive
to allow interoperability.
Particularly due to unclear statements on the use of cryptography with security
mode \sms, we studied the protocol with and without counter-measures and proved
the need of encryption for secrets to ensure messages integrity.
We also found attacks on authentication and provided counter-measures.

In the future, we aim at testing the attacks we found on official
implementations which are proprietary. This next work would analyze
if these implementations filled the gap as did {\em FreeOpcUa}, 
adopting the prudent engineering practices advised in~\cite{AN96} to
circumvent the attacks.
If they do, then our analysis formally proves the security of the protocol.
More generally, we are also interested in
studying how to model integrity properties for messages and
communication flows in \proverif{} since it is one of the requirements
of industrial protocols.

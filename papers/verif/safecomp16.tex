%\documentclass[a4paper]{article}
\documentclass{llncs}

\usepackage[T1]{fontenc}
\usepackage[utf8]{inputenc}

\usepackage{multirow}

\usepackage{graphicx}
\usepackage{xstring}
\usepackage{xspace}
\usepackage{amstext}

%\usepackage{showframe}
\usepackage{times}

\usepackage{auto-pst-pdf}
\usepackage{msc}

% ========== Don't touch ==========
\makeatletter
    \def\msc@frame{no}
    \def\msc@settitle{}
\makeatother

\newcommand{\dosmnone}[3]{#1}
\newcommand{\dosmsign}[3]{#1, $\left\{\mbox{h(#1)}\right\}_{\mbox{#3}}$}
\newcommand{\dosmsignandencrypt}[3]{$\left\{\mbox{#1}\right\}_{\mbox{#2}}$, $\left\{\mbox{h(#1)}\right\}_{\mbox{#3}}$}
\newcommand{\dosmhmacencrypt}[3]{$\left\{\mbox{#1}\right\}_{\mbox{#2}}$, MAC(\mbox{#3}, (\mbox{#1}))}

\newcommand{\smn}{None}
\newcommand{\sms}{Sign}
\newcommand{\smse}{SignAndEncrypt}
\newcommand{\smseshort}{SignEnc}

\newcommand{\sm}{\IfStrEq{\smname}{\smn}{\dosmnone}{\IfStrEq{\smname}{\sms}{\dosmsign}{\dosmsignandencrypt}}}

\newcommand{\ifsmnotnone}[1]{\IfStrEq{\smname}{\smn}{}{#1}}
\newcommand{\smname}{\smseshort}
% ========== Don't touch ==========

\newcommand{\UNSAFE}{{\color{red!50!black} UNSAFE}\xspace}
\newcommand{\SAFE}{{\color{green!50!black} SAFE}\xspace}
\newcommand{\NA}{{\color{blue!50!black} N/A}\xspace}
\newcommand{\TODO}{{\color{red}\bf TODO}\xspace}
\newcommand{\DiH}{Diffie-Hellman\xspace}
\newcommand{\proverif}{ProVerif\xspace}
\newcommand{\ofmc}{OFMC\xspace}
\newcommand{\spear}{SPEAR II\xspace}
\newcommand{\opcua}{OPC-UA\xspace}
\newcommand{\opc}{OPC\xspace}
\newcommand{\dcom}{DCOM\xspace}
\newcommand{\mms}{MMS\xspace}
\newcommand{\iec}[1]{IEC\xspace#1\xspace}
\newcommand{\iccp}{ICCP\xspace}
\newcommand{\modbus}{MODBUS\xspace}
\newcommand{\profinet}{PROFINET\xspace}
\newcommand{\etherip}{EtherNet/IP\xspace}
\newcommand{\dnp}{DNP3\xspace}
\newcommand{\eg}{{\em e.g.}\xspace}
\newcommand{\etal}{{\em et al.}\xspace}
\newcommand{\securechan}{{\em OpenSecureChannel}\xspace}
\newcommand{\session}{{\em CreateSession}\xspace}

\title{Formal Analysis of Security Properties on the \opcua SCADA Protocol}
\author{Maxime Puys$^1$ \and Marie-Laure Potet$^1$ \and Pascal Lafourcade$^{1,2}$}
\institute{(1) Verimag, University Grenoble Alpes,  Gi\`eres, France \\
  \texttt{firstname.lastname@imag.fr}\\
   (2) LIMOS, University Clermont Auvergne,  Campus des C\'ezeaux, Aubi\`ere, France\\
  \texttt{pascal.lafourcade@udamail.fr}
  \thanks{This work has been partially supported by the LabEx PERSYVAL-Lab
      (ANR-11-LABX-0025), the project {\em Programme Investissement d’Avenir
      FSN AAP Sécurité Numérique n\textsuperscript{o}3} ARAMIS (P3342-146798) and “Digital trust” Chair from the University of Auvergne
Foundation.}
}
\date{}

\begin{document}

\maketitle

\begin{abstract}
    Industrial systems are publicly the target of cyberattacks since
    Stuxnet \cite{Lan11}.  Nowadays they often communicate over
    insecure media such as Internet.  Due to their interaction with
    the physical world, it is crucial to prove the security of their
    protocols.  In this paper, we formally study the security of one
    of the most used industrial protocols: \opcua.  Using ProVerif, a
    well known cryptographic protocol verification tool, we are able
    to check secrecy and authentication properties.  We find several
    attacks on the protocols and provide countermeasures.
\end{abstract}

\section{Introduction}
Industrial systems also called SCADA (Supervisory Control And Data
Acquisition) have been known to be targeted by cyberattacks since the
famous Stuxnet case~\cite{Lan11} in 2010.  Due to the criticity of
their interaction with the physical world, these systems can
potentially be really harmful for humans and environment.  The
frequence of such attacks is increasing to become one of the priorities for
governmental agencies (\eg~\cite{SFS11} from the US National Institute of
Standards and Technology (NIST) or
\cite{ANSSI12_guide_securite_industrielle_en} from the French {\em Agence
nationale de la sécurité des systèmes d'information} (ANSSI)).
%As
%industrial systems historically have been physically isolated from the
%rest of the world, there were less exposed to cyber attacks and their
%security was less considered. Nowadays, such attacks become feasible
%because these systems are spreading geographically and communicating
%more and more through unsafe mediums like Internet.


%Industrial systems have specificities. First the isolation of such
%systems requires the attacker to be physically present in the
%system. Then compagnies focused more on the protection against natural
%deceases and human mistakes than cybersecurity (often call security).
%In the context of security there is an adversary willing to perform malicious
%actions.
Industrial systems differ from other systems because of
the really long lifetime of there devices and their difficulty to
patch in case of vulnerabilities.
Such specificities encourage to carefully check
standards and applications before deploying them.
%It also explain the number of legacy hosts.
%Moreover, most of industrial protocols are proprietary and
%provide a very low level of security if
%any, \eg \modbus~\cite{MODBUS}, \profinet~\cite{PROFINET}, \etherip~\cite{Bro01}
%or
%\dnp~\cite{CR04}.
%However, in 2006, the {\em OPC Foundation} (an industry consortium)
%released the first version of \opcua~\cite{MLD09}, which is presented
%as the new standard for industrial communications. %% This standard  and whose security
%% is quite closer to business IT's protocols such as TLS~\cite{DR08}.
%% This directly applies to communication protocols.
As it already appeared for business IT's protocols for twenty years,
automated verification is crucial in order to discover flaws in the
protocols' specifications before assessing implementations. However,
the lack of formal verification of industrial protocols has been
emphasized in 2006 by Igure et al.~\cite{ILW06} and in 2009 by
Patel \emph{et al.}~\cite{PBG09}.  They particularly argued that
automated protocol verification allows to understand most of the
vulnerabilities of a protocol before changing its standards which
helps at minimizing the number of revisions which costs time and
money.  Moreover, due to the combinational explosion of the number of
possible execution traces and the increasing complexity of the
systems, only automated verification can ensure the security of a
protocol in a given model.

\TODO Intruduire \opcua qq part


%\subsection{State-of-the-art}\label{sec:intro_sota}
\paragraph{State-of-the-art:}\label{sec:intro_sota}

%The security of industrial protocols becomes a hot topic and some works are
%present in the literature.
Most of the works on the security of industrial protocols only rely on
specifications written in human language rather than using formal methods.
In 2004, Dzung et al. proposed a detailed survey on the security in SCADA
systems including analysis on the security properties offered by \opc (Open
Platform Communications), \mms (Manufacturing Message Specification),
\iec{61850}, \iccp (Inter-Control Center Communications Protocol) and \etherip.
Still in 2004, Clarke et al.~\cite{CR04} discussed the security of \dnp
(Distributed Network Protocol) and \iccp.
In 2006, in the technical documentation of \opcua (OPC Unified Architecture) the
authors detailled the security measures of the protocol (specialy in part 2, 4
and 6).
In 2015, Wanying et al. summarizeed the security offered by \modbus, \dnp and
\opcua.

On the other hand, some works propose new versions of existing protocols to make
them secure against malicious adversaries.
In 2007, Patel et al.~\cite{PY07} studied the security of \dnp and proposed two
ways of enhancing it through digital signatures and challenge-response models.
In 2009, Fovino et al.~\cite{FCMT09} proposed a secure version of \modbus
relying on well-known cryptographic primitives such as RSA or SHA2.
%This version of the protocol also need to introduce new components in the system
%to allow existing devices to use these cryptographic primitives.
In 2013, Hayes et al.~\cite{HE13} designed another secure \modbus protocol using
hash-based message authentication codes (HMACs) and built on STCP (Stream
Transmission Control Protocol).

To the best of our knowledge, the only work directly making use of formal
methods to prove the security of industrial protocols or find attack against
them is from Graham et al.~\cite{GP05} in 2005.
They proposed a formal verification of \dnp using \ofmc~\cite{BMV03} and
\spear~\cite{SH01}.
%% \TODO Sadly we were not able to access their modelings to discuss them with
%% ours.
%In 2008, Dutertre~\cite{Dut08} detailed formal specifications of \modbus
%developed using PVS, a generic theorem prover in order to help proving the
%consistancy of an implementation with the standards.

%\subsection{Contributions}
\paragraph{Contributions:}

We propose a formal analysis of the security of
% standard \modbus protocol and
%the correction proposed in~\cite{FCMT09}.%,HE13}.
%We also assess the 
the sub-protocols involved in the \opcua handshake, namely \opcua
OpenSecureChannel and \opcua CreateSession.  We are able to find
attacks against both of them and provide some countermeasures when
needed.  This work makes use of one of the most efficient tools in the
domain of cryptographic protocol verification according
to \cite{LP15}, namely \proverif developed by Blanchet et al.~\cite{Bla01}.
It analyzes a protocol written either in Horn clauses format or using
a subset of the Pi-calculus for an unbounded number of sessions
considering the classical Dolev-Yao intruder model~\cite{DY81}.  This
powerful intruder controls the network, listens, stops, forges,
replays or modifies some messages according to its capabilities and
knowledge.  The perfect encryption hypothesis is assumed, meaning that
it is not possible to decrypt a ciphertext without its encryption key
or to forge a signature without knowing the secret key.  In this
context, verification tools are able to verify security properties of
a protocol such as secrecy and authentication.  The first property
ensures that a secret message cannot be discovered by an unauthorized
agent (including the intruder).  The authentication property means
that one participant of the protocol is guaranteed to communicate with
another one.

\paragraph{Outline:} In the Section~\ref{sec:secure_channel}, we analyse the
security of \opcua OpenSecureChannel and 
\opcua CreateSession in Section~\ref{sec:session}. Finally, we conclude in Section~\ref{sec:conclusion}.


\section{\opcua OpenSecureChannel}\label{sec:secure_channel}
%\newcommand{\gereq}{GetEndpointRequest}
%\newcommand{\geres}{GetEndpointResponse}
%\newcommand{\oscreq}{OpenSecureChannelRequest}
%\newcommand{\oscres}{OpenSecureChannelResponse}
\newcommand{\gereq}{GEReq}
\newcommand{\geres}{GERes}
\newcommand{\oscreq}{OSCReq}
\newcommand{\oscres}{OSCRes}


The {\em OpenSecureChannel} sub-protocol described in
Figure~\ref{fig:secure_channel_se} aims to authenticate a client and
a server and allows them to generate a shared key for the later communications.
\opcua can be used with three security modes, namely {\em \smn}, {\em \sms} and
{\em \smse}.
\vspace{-.5em}
\begin{itemize}
    \item {\em \smse}: claims to provide secrecy using symmetric and
      asymmetric encryption and both authentication and integrity
      through digital signatures (\eg $enc_{pkB}(m)$, $sig_{skA}(h(m))$).
  \item {\em \sms}: same as {\em \smse} but without any encryption.  Thus
      nonces are not used to generate a shared key but bring freshness
      to the messages (\eg $m$, $sig_{skA}(h(m))$).
  \item {\em \smn}: does not provide any security.  Using this mode, the
      {\em OpenSecureChannel} sub-protocol does not serve much
      purpose but is used for compatibility.
\end{itemize}

\vspace{-1em}
\begin{figure}[htb]
    \centering
    \renewcommand{\smname}{\smseshort}
    \resizebox{.8\linewidth}{!}{
        \begin{postscript}
            \begin{msc}{Secure channel creation}
                \setlength{\envinstdist}{1.5\envinstdist}
                \setlength{\instdist}{2.6\instdist}
                \setlength{\labeldist}{1.5\labeldist}

                \declinst{cli}{C}{}
                \declinst{intruder}{DiscoreryEndpoint}{}
                \declinst{sess}{S}{}

                \msccomment[-.5]{1.}{cli}
                \mess{\gereq}{cli}{intruder}
                \nextlevel[1.5]

                \msccomment[-.5]{2.}{cli}
                \mess{\geres, pk(S), \smname, SP, UP}{intruder}{cli}
                \nextlevel[1]
                
                %\action*{Validates pk(S)}{cli}
                %\nextlevel[2]
                
                \action*{Generates N$_{C}$}{cli}
                \nextlevel[3]

                \msccomment[-.5]{3.}{cli}
                \mess{pk(C), \sm{\oscreq, pk(C), $N_{C}$}{pk(S)}{sk(C)}}{cli}{sess}
                \nextlevel[1]

                %\action*{Validates pk(C) }{sess}
                %\nextlevel[2]

                \action*{Generates N$_{S}$ }{sess}
                \nextlevel[2.5]

                %\ifsmnotnone{%
                %    \action*{$(K_{SC}, KSig_{SC})$ = $P\_hash(N_{S}, N_{C})$ }{sess}
                %    \nextlevel[3.5]
                %}

                \msccomment[-.5]{4.}{cli}
                \mess{\sm{\oscres, $N_{S}$, ST, TTL}{pk(C)}{sk(S)}}{sess}{cli}
                %\nextlevel[1]

                %\ifsmnotnone{%
                %    \action*{$(K_{CS}, KSig_{CS})$ = $P\_hash(N_{C}, N_{S})$ }{cli}
                %    \nextlevel[2]
                %}
            \end{msc}
        \end{postscript}
    }
    \caption{\opcua \securechan sub-protocol}
    \label{fig:secure_channel_se}
\end{figure}
\vspace{-1em}

In message 1. $C$ requests information on $S$ with {\em GEReq} meaning
{\em GetEndpointRequest}.
In message 2. $DiscoveryEndpoint$ answers with these information with
{\em GERes} standing for {\em GetEndpointResponse}, {\em SP} for
{\em Security Policy} and {\em UP} for {\em UserPolicy}.
Both {\em SP} and {\em UP} are used for cryptographic primitive negociations.
In message 3. $C$ sends a nonce (here $N_{C}$) to $S$ with {\em OSReq}
standing for {\em OpenSecureChannelRequest}.
Finally in message 4. $S$ answers a nonce to $C$ (here $N_{S}$) with
{\em OSCRes} for {\em OpenSecureChannelResponse}, {\em ST} for
{\em SecurityToken} (a unique identifier for the channel) and {\em TTL} for
{\em TimeToLive} (its life-time).
The four terms {\em GEReq, GERes, OSCReq} and {\em OSCRes} indicate the purpose
of each message of the protocol.
At the end of this protocol, both $C$ and $S$ derive four keys using the nonces
using a function named $P\_hash$, similar to the one specified in
TLS~\cite{DR08}: $(K_{CS}, KSig_{CS})$ = $P\_hash(N_{C}, N_{S})$ and
$(K_{SC}, KSig_{SC})$ = $P\_hash(N_{S}, N_{C})$.

\subsection{Modeling}

Normally, a {\em GetEnpointRequest} would be answered by a list of
session endpoints with possibly different security modes. We suppose
that only one endpoint is answered and that the client will always
accept the security mode proposed.
Client's and server's certificates have been abstracted as their public
keys (which explains why $pk(C)$ appears twice in message 3.
Moreover, thanks to the perfect
encryption hypothesis, we can abstract the cryptographic primitives
used.  We consider an intruder whose public key would be accepted by a
legitimate client or server.  Such an intruder could for instance
represent a legitimate device that has been corrupted through a virus
or that is controlled by a malicious operator.
%Finally, it is said in
%\cite{MLD09} that ``\emph{The establishment of the Secure Channel is
%  mainly used for exchanging special secret information between
%  clients and servers. This secret is used for deriving Symmetric Keys
%  $[...]$}''.  To the best of our knowledge, no more precision is
%given in \cite{MLD09,opcua_part2,opcua_part4,opcua_part6} on which
%cryptographic primitives should be used for such purpose.  Thus we
%chose to model this key derivation using the \DiH mechanisms which
%relies on the commutativity of the exponentiation: $(g^a)^b =
%(g^b)^a$.
In this context, we consider the following security objectives: (i)
the secrecy of the keys obtained by C, (ii) the secrecy of the keys obtained
by S, (iii) the authentication of C on $N_{C}$ and (iv) the authentication of S
on $N_{S}$.

\subsection{Results}\label{sec:secure_channel_res}

We run \proverif on this protocol for of the three security
modes of \opcua for each objective proposed.
Results are shown in Table \ref{tab:secure_channel_results}.

\vspace{-1em}
\begin{table}[htb]
    \centering
    %\hspace{-1em}
    %\resizebox{1.05\columnwidth}{!}{
    \begin{tabular}{|c|c|c|c|c|}
        \hline
        \multirow{2}{*}{\opcua Security mode} & \multicolumn{4}{|c|}{Objectives} \\
        \cline{2-5}
                       & Sec $K_{cli}$ & Sec $K_{srv}$ & Auth $g^{N_{S}}$  & Auth $g^{N_{C}}$  \\
        \hline
        \smn           & \UNSAFE       & \UNSAFE       & \UNSAFE           & \UNSAFE           \\ 
        \hline
        \sms           & \UNSAFE       & \UNSAFE      & \UNSAFE           & \UNSAFE           \\ 
        \hline
        \smseshort     & \SAFE         & \SAFE         & \UNSAFE           & \UNSAFE           \\ 
        \hline
    \end{tabular}
    %}
    \caption{Results for textbook {\em OpenSecureChannel} sub-protocol}
    \label{tab:secure_channel_results}
\end{table}
\vspace{-2em}

Obviously, as the security mode \smn~does not provide any security,
all objectives can be attacked and as nonces are exchanged in plaintext in
security mode \sms, the keys are leaked.
Moreover, attacks on authentication
in the case of \sms~and \smse~implies the intruder rerouting messages.
Such manipulation differs from replaying a message as it does not
delay the message more than what a legitimate router would do (thus
avoiding replay protections such as timestamps).  Figure
\ref{fig:secure_channel_atk} shows an attack on the authentication of
C using $N_{C}$.  This attack is possible because the standard
\opcua protocol does not require explicitly to specify the identity of
the receiver of a message.  Thus it allows the intruder to send to S
the signed message C sent to him similarly as the attack on the
Needham-Schroeder protocol~\cite{Low96}.
%However, these attacks are
%not exploitable in security modes \sms~and \smse~since the secrecy of the keys
%is proven by the tool.  It means that even if an intruder can usurp a
%client when speaking to a server and vice-versa, he will not obtain
%the key derived by the protocol.
%Such property holds thanks to the \DiH key derivation mechanism that allows
%to exchange the nonces without revealing them (sending $g^{N_{C}}$ instead of 
%$N_{C}$).

\vspace{-1em}
\begin{figure}[htb]
    \hspace{-2.25em}
    \renewcommand{\smname}{\smse}
    \resizebox{1.1\columnwidth}{!}{
        \begin{postscript}
            \begin{msc}{Secure channel creation}
                \setlength{\envinstdist}{1.5\envinstdist}
                \setlength{\instdist}{5.1\instdist}
                \setlength{\labeldist}{1.5\labeldist}

                \declinst{cli}{C}{}
                \declinst{intruder}{I}{}
                \declinst{sess}{S}{}

                \mess{\gereq}{cli}{intruder}
                \nextlevel[1.5]

                \mess{\geres, pk(I), \smname, SP, UP}{intruder}{cli}
                \nextlevel[1]
                
                %\action*{Validates pk(I)}{cli}
                %\nextlevel[2]
                
                \action*{Generates N$_{C}$}{cli}
                \nextlevel[3]

                \mess{pk(C), \sm{\oscreq, pk(C), $N_{C}$}{pk(I)}{sk(C)}}{cli}{intruder}
                \nextlevel[2]

                \mess{pk(C), \sm{\oscreq, pk(C), $N_{C}$}{pk(S)}{sk(C)}}{intruder}{sess}
                \nextlevel[1]

                %\action*{Validates pk(C) }{sess}
                %\nextlevel[2]

                %\action*{Generates N$_{S}$ }{sess}
                %\nextlevel[2]

                %\ifsmnotnone{%
                %    \action*{$K_{srv}$ = $(g^{N_{C}})^{N_{S}}$ }{sess}
                %    \nextlevel[3]
                %}

                %\mess{\sm{\oscres, $g^{N_{S}}$, ST, TTL}{pk(C)}{sk(S)}}{sess}{intruder}
                %\nextlevel[1.5]
            \end{msc}
        \end{postscript}
    }
    \caption{Attack on $N_{C}$: I usurps C when speaking to S}
    \label{fig:secure_channel_atk}
\end{figure}

\vspace{1em}
\subsection{Fixed version}\label{sec:secure_channel_fix}

We propose a fixed version of the {\em OpenSecureChannel} sub-protocol.
%in Figure \ref{fig:secure_channel_fix_se}.
It consists in explicitly adding the identity of the receiver to the messages
and thus  avoids an intruder rerouting signed messages to usurp hosts as
presented in Section \ref{sec:secure_channel_res}.
More formally, message 3. and 4. of Figure~\ref{fig:secure_channel_se} are replaced
by:
%\vspace{-15pt}
\arraycolsep=1.4pt
{\small
\begin{flushleft}
    $\begin{array}{lrllll}
        3. & C & \rightarrow & S & : & \text{\sm{\oscreq, {\bf S}, pk(C), $N_{C}$}{pk(S)}{sk(C)}}\\
        4. & S & \rightarrow & C & : & \text{\sm{\oscres, {\bf C}, $N_{S}$, ST, TTL}{pk(C)}{sk(S)}}\\
    \end{array}$
\end{flushleft}
}
%\vspace{-.5em}

Note that instead of adding identity in these messages, one could add the user's
public keys (pk(C) instead of C for instance).
This is one of the classical counter-measures for communication protocols
proposed in \cite{AN96}.
Moreover, the \opcua standards~\cite{MLD09,opcua_part2,opcua_part4,opcua_part6}
advise the use of key wrapping~\cite{FLS11}.
Although the use of such mechanism in security mode \sms~remains unclear in the
standard, we take advantage of it in this fixed version to ensure the secrecy of
derived keys in security mode \sms.
More formally, occurrences of $N_{C}$ are replaced by
$\left\{\mbox{$N_{C}$}\right\}_{pk(S)}$ in message 3 and all occurrences of
$N_{S}$ in message 4 by $\left\{\mbox{$N_{S}$}\right\}_{pk(C)}$.
Thus all the messages is signed in security mode \sms but only the nonces are
encrypted.
We also use \proverif to check the security of this fixed version with each
counter-measure separately and together.
The results are presented in Table \ref{tab:secure_channel_fix_results} and show
that explicitly specifying the receiver of a message in its signature fixes
attacks found on authentication for security modes \sms~and \smse.
They also show that the use of key wrapping secure the secrecy of the keys for
security mode \sms~since the key is never sent in plaintext.

\vspace{-1em}
\begin{table}[htb]
    \centering
    %\hspace{-1em}
    %\resizebox{1.05\columnwidth}{!}{
    \begin{tabular}{|c|c|c|c|c|}
        \hline
        \multirow{2}{*}{\opcua Security mode} & \multicolumn{4}{|c|}{Objectives} \\
        \cline{2-5}
                       & Sec $K_{cli}$ & Sec $K_{srv}$ & Auth $g^{N_{S}}$  & Auth $g^{N_{C}}$  \\
        \hline
        \smn           & \UNSAFE       & \UNSAFE       & \UNSAFE           & \UNSAFE           \\ 
        \hline
        \sms           & \SAFE         & \SAFE         & \SAFE             & \SAFE             \\ 
        \hline
        \smseshort     & \SAFE         & \SAFE         & \SAFE             & \SAFE             \\ 
        \hline
    \end{tabular}
    %}
    \caption{Results for fixed {\em OpenSecureChannel} sub-protocol}
    \label{tab:secure_channel_fix_results}
\end{table}
\vspace{-3em}


\section{\opcua CreateSession}\label{sec:session}
%\newcommand{\csreq}{CreateSessionRequest}
%\newcommand{\csres}{CreateSessionResponse}
%\newcommand{\asreq}{ActivateSessionRequest}
%\newcommand{\asres}{ActivateSessionResponse}
\newcommand{\csreq}{CSReq}
\newcommand{\csres}{CSRes}
\newcommand{\asreq}{ASReq}
\newcommand{\asres}{ASRes}

%\begin{figure}[htb]
%    \renewcommand{\smname}{\smn}
%    \resizebox{\textwidth}{!}{
%        \begin{postscript}
%            \begin{msc}{Session creation}
%                \setlength{\envinstdist}{1.5\envinstdist}
%                \setlength{\instdist}{5.7\instdist}
%
%                \declinst{cli}{C}{}
%                \declinst{sess}{S}{}
%
%                \mess{\sm{\csreq}{$K_{cli}$}{sk(C)}}[b]{cli}{sess}
%                \nextlevel[1.5]
%
%                \mess{\sm{SSC, $N_{s2}$}{$K_{srv}$}{sk(S)}}[b]{sess}{cli}
%                \nextlevel[1.5]
%                
%                \action*{Validates SSC}{cli}
%                \nextlevel[2]
%                
%                \action*{Verifies the signature of $N_{s2}$}{cli}
%                \nextlevel[2]
%                
%                \mess{\sm{CSC, Login, Passwd}{$K_{cli}$}{sk(C)}}[b]{cli}{sess}
%                \nextlevel[2]
%                
%                \action*{Validates CSC}{sess}
%                \nextlevel[2]
%
%                \action*{Validates (Login, Passwd) }{sess}
%                \nextlevel[2]
%
%                \mess{\sm{\asres}{$K_{src}$}{sk(S)}}[b]{sess}{cli}
%                \nextlevel[2]
%            \end{msc}
%        \end{postscript}
%    }
%    \caption{OPC-UA Session None}
%\end{figure}


%\begin{figure}[htb]
%    \renewcommand{\smname}{\sms}
%    \resizebox{\textwidth}{!}{
%        \begin{postscript}
%            \begin{msc}{Session creation}
%                \setlength{\envinstdist}{1.5\envinstdist}
%                \setlength{\instdist}{5.7\instdist}
%
%                \declinst{cli}{C}{}
%                \declinst{sess}{S}{}
%
%                \mess{\sm{\csreq}{$K_{cli}$}{sk(C)}}[b]{cli}{sess}
%                \nextlevel[1.5]
%
%                \mess{\sm{SSC, $N_{s2}$}{$K_{srv}$}{sk(S)}}[b]{sess}{cli}
%                \nextlevel[1.5]
%                
%                \action*{Validates SSC}{cli}
%                \nextlevel[2]
%                
%                \action*{Verifies the signature of $N_{s2}$}{cli}
%                \nextlevel[2]
%                
%                \mess{\sm{CSC, Login, Passwd}{$K_{cli}$}{sk(C)}}[b]{cli}{sess}
%                \nextlevel[2]
%                
%                \action*{Validates CSC}{sess}
%                \nextlevel[2]
%
%                \action*{Validates (Login, Passwd) }{sess}
%                \nextlevel[2]
%
%                \mess{\sm{\asres}{$K_{src}$}{sk(S)}}[b]{sess}{cli}
%                \nextlevel[2]
%                \end{msc}
%        \end{postscript}
%    }
%    \caption{OPC-UA Session Sign}
%\end{figure}


The \opcua {\em CreateSession} sub-protocol allows a client to send credentials
(e.g.: a login and a password) over an already created Secure Channel.
This sub-protocol is presented in Figure \ref{fig:session_se}.

\begin{figure}[htb]
    \renewcommand{\smname}{\smse}
    \resizebox{\textwidth}{!}{
        \begin{postscript}
            \begin{msc}{Session creation}
                \setlength{\envinstdist}{1.5\envinstdist}
                \setlength{\instdist}{6\instdist}

                \declinst{cli}{C}{}
                \declinst{sess}{S}{}

                \mess{\sm{\csreq, pk(C), $N_{C}$}{$K_{cli}$}{sk(C)}}[b]{cli}{sess}
                \nextlevel[1.5]
                
                \action*{Validates pk(C)}{sess}
                \nextlevel[2]

                \mess{\sm{\csres, pk(S) $N_{S}$}{$K_{srv}$}{sk(S)}}[b]{sess}{cli}
                \nextlevel[1.5]
                
                \action*{Validates pk(S)}{cli}
                \nextlevel[2]
                
                \mess{\sm{\asreq, pk(C), Login, Passwd}{$K_{cli}$}{sk(C)}}[b]{cli}{sess}
                \nextlevel[2]
                
                \action*{Validates pk(C)}{sess}
                \nextlevel[2]

                \action*{Validates (Login, Passwd) }{sess}
                \nextlevel[2]

                \mess{\sm{\asres, $N_{S2}$}{$K_{src}$}{sk(S)}}[b]{sess}{cli}
                \nextlevel[2]
            \end{msc}
        \end{postscript}
    }
    \label{fig:session_se}
    \caption{OPC-UA CreateSession sub-protocol}
\end{figure}

For this protocol, a modeling choice appears.
We can either suppose that C respects security standards and would use different
credentials for each server or suppose he will always send the same login and
password for all servers.

We first suppose that C always sends the same credentials including in a session
with the intruder (possibly a rogue server).
%Thus, once the intruder obtained C's Login/Passwd, he can mount an 
%authentication attack in another session.
Results under this assumption are presented in Table \ref{tab:session_results}.

\begin{table}[htb]
    \centering
    \begin{tabular}{|c|c|c|c|c|}
        \hline
        \multicolumn{2}{|c}{\opcua Security modes} & \multicolumn{3}{|c|}{Objectives}   \\
        \hline
        C              & S              & Sec $K$       & Sec $Creds$   & Auth $Creds$  \\
        \hline
        None           & None           & \SAFE         & \UNSAFE       & \UNSAFE       \\ 
        \hline
        None           & Sign           & \SAFE         & \UNSAFE       & \SAFE         \\ 
        \hline
        None           & SignAndEncrypt & \SAFE         & \UNSAFE       & \SAFE         \\ 
        \hline
        Sign           & None           & \SAFE         & \UNSAFE       & \UNSAFE       \\ 
        \hline
        Sign           & Sign           & \SAFE         & \UNSAFE       & \UNSAFE       \\ 
        \hline
        Sign           & SignAndEncrypt & \SAFE         & \UNSAFE       & \SAFE         \\ 
        \hline
        SignAndEncrypt & None           & \SAFE         & \UNSAFE       & \UNSAFE       \\ 
        \hline
        SignAndEncrypt & Sign           & \SAFE         & \UNSAFE       & \UNSAFE       \\ 
        \hline
        SignAndEncrypt & SignAndEncrypt & \SAFE         & \UNSAFE       & \SAFE         \\ 
        \hline
    \end{tabular}
    \label{tab:session_results}
    \caption{Results for \opcua {\em CreateSession} sub-protocol}
\end{table}

\subsection{Fixed version}

We apply the same fix than in OpenSecureConversation, the resulting protocol is
displayed in Figure \ref{fig:session_fix}

\begin{figure}[htb]
    \renewcommand{\smname}{\smse}
    \resizebox{\textwidth}{!}{
        \begin{postscript}
            \begin{msc}{Session creation}
                \setlength{\envinstdist}{1.5\envinstdist}
                \setlength{\instdist}{6.25\instdist}

                \declinst{cli}{C}{}
                \declinst{sess}{S}{}

                \mess{\sm{\csreq, pk(C), $N_{C}$}{$K_{cli}$}{sk(C)}}[b]{cli}{sess}
                \nextlevel[1.5]
                
                \action*{Validates pk(C)}{sess}
                \nextlevel[2]

                \mess{\sm{\csres, pk(S) $N_{S}$}{$K_{srv}$}{sk(S)}}[b]{sess}{cli}
                \nextlevel[1.5]
                
                \action*{Validates pk(S)}{cli}
                \nextlevel[2]
                
                \mess{\sm{\asreq, S, pk(C), Login, Passwd}{$K_{cli}$}{sk(C)}}[b]{cli}{sess}
                \nextlevel[2]
                
                \action*{Validates pk(C)}{sess}
                \nextlevel[2]

                \action*{Validates (Login, Passwd) }{sess}
                \nextlevel[2]

                \mess{\sm{\asres, $N_{S2}$}{$K_{src}$}{sk(S)}}[b]{sess}{cli}
                \nextlevel[2]
            \end{msc}
        \end{postscript}
    }
    \label{fig:session_fix}
    \caption{OPC-UA fixed CreateSession sub-protocol}
\end{figure}

Then authentication becomes secure as shown in Table \ref{tab:session_fix_results}.

\begin{table}[htb]
    \centering
    \begin{tabular}{|c|c|c|c|c|}
        \hline
        \multicolumn{2}{|c}{\opcua Security modes} & \multicolumn{3}{|c|}{Objectives}   \\
        \hline
        C              & S              & Sec $K$       & Sec $Creds$   & Auth $Creds$  \\
        \hline
        None           & None           & \SAFE         & \UNSAFE       & \UNSAFE       \\ 
        \hline
        None           & Sign           & \SAFE         & \UNSAFE       & \SAFE         \\ 
        \hline
        None           & SignAndEncrypt & \SAFE         & \UNSAFE       & \SAFE         \\ 
        \hline
        Sign           & None           & \SAFE         & \UNSAFE       & \UNSAFE       \\ 
        \hline
        Sign           & Sign           & \SAFE         & \UNSAFE       & \SAFE         \\ 
        \hline
        Sign           & SignAndEncrypt & \SAFE         & \UNSAFE       & \SAFE         \\ 
        \hline
        SignAndEncrypt & None           & \SAFE         & \UNSAFE       & \UNSAFE       \\ 
        \hline
        SignAndEncrypt & Sign           & \SAFE         & \UNSAFE       & \SAFE         \\ 
        \hline
        SignAndEncrypt & SignAndEncrypt & \SAFE         & \UNSAFE       & \SAFE         \\ 
        \hline
    \end{tabular}
    \label{tab:session_fix_results}
    \caption{Results}
\end{table}

In a second time, we suppose that C would use different credentials for each server.
Then both normal and fixed CreateSession results in Table \ref{tab:session_uniq_creds_results}

\begin{table}[htb]
    \centering
    \begin{tabular}{|c|c|c|c|c|}
        \hline
        \multicolumn{2}{|c}{\opcua Security modes} & \multicolumn{3}{|c|}{Objectives}   \\
        \hline
        C              & S              & Sec $K$       & Sec $Creds$   & Auth $Creds$  \\
        \hline
        None           & None           & \SAFE         & \UNSAFE       & \SAFE         \\ 
        \hline
        None           & Sign           & \SAFE         & \UNSAFE       & \SAFE         \\ 
        \hline
        None           & SignAndEncrypt & \SAFE         & \UNSAFE       & \SAFE         \\ 
        \hline
        Sign           & None           & \SAFE         & \SAFE         & \SAFE         \\ 
        \hline
        Sign           & Sign           & \SAFE         & \UNSAFE       & \SAFE         \\ 
        \hline
        Sign           & SignAndEncrypt & \SAFE         & \UNSAFE       & \SAFE         \\ 
        \hline
        SignAndEncrypt & None           & \SAFE         & \SAFE         & \SAFE         \\ 
        \hline
        SignAndEncrypt & Sign           & \SAFE         & \SAFE         & \SAFE         \\ 
        \hline
        SignAndEncrypt & SignAndEncrypt & \SAFE         & \SAFE         & \SAFE         \\ 
        \hline
    \end{tabular}
    \label{tab:session_uniq_creds_results}
    \caption{Results}
\end{table}


\section{Conclusion}\label{sec:conclusion}
We provide a formal verification of the new industry standard
communication protocol: \opcua, relying its official specifications
\cite{MLD09,opcua_part2,opcua_part4,opcua_part6}.
This work makes use of \proverif{} a tool for automatic cryptographic protocol
verification. 
Protocol modelings were tedious tasks since specifications are willingly elusive
to allow interoperability.
Particularly due to unclear statements on the use of cryptography with security
mode \sms, we studied the protocol with and without counter-measures and proved
the need of encryption for secrets to ensure messages integrity.
We also found attacks on authentication and provided counter-measures.

In the future, we aim at testing the attacks we found on official
implementations which are proprietary. This next work would analyze
if these implementations filled the gap as did {\em FreeOpcUa}, 
adopting the prudent engineering practices advised in~\cite{AN96} to
circumvent the attacks.
If they do, then our analysis formally proves the security of the protocol.
More generally, we are also interested in
studying how to model integrity properties for messages and
communication flows in \proverif{} since it is one of the requirements
of industrial protocols.


\bibliographystyle{unsrt}
\bibliography{phdBiblio}

\end{document}

%\documentclass[a4paper]{article}
\documentclass[a4paper]{article}

\usepackage[T1]{fontenc}
\usepackage[utf8]{inputenc}

\usepackage{graphicx}
\usepackage{xstring}
\usepackage{xspace}
\usepackage{amstext}

%\usepackage{times}

\usepackage{auto-pst-pdf}
    \usepackage{msc}


% ========== Don't touch ==========
\makeatletter
    \def\msc@frame{no}
    \def\msc@settitle{}
\makeatother

\newcommand{\dosmnone}[3]{#1}
\newcommand{\dosmsign}[3]{#1, $\left\{\mbox{h(#1)}\right\}_{\mbox{#3}}$}
\newcommand{\dosmsignandencrypt}[3]{$\left\{\mbox{#1}\right\}_{\mbox{#2}}$, $\left\{\mbox{h(#1)}\right\}_{\mbox{#3}}$}

\newcommand{\smn}{None}
\newcommand{\sms}{Sign}
\newcommand{\smse}{SignAndEncrypt}
\newcommand{\smseshort}{SignEnc}

\newcommand{\sm}{\IfStrEq{\smname}{\smn}{\dosmnone}{\IfStrEq{\smname}{\sms}{\dosmsign}{\dosmsignandencrypt}}}

\newcommand{\ifsmnotnone}[1]{\IfStrEq{\smname}{\smn}{}{#1}}
\newcommand{\smname}{\smseshort}
% ========== Don't touch ==========

\newcommand{\UNSAFE}{{\color{red!50!black} UNSAFE}\xspace}
\newcommand{\SAFE}{{\color{green!50!black} SAFE}\xspace}
\newcommand{\NA}{{\color{blue!50!black} N/A}\xspace}
\newcommand{\TODO}{{\color{red}\bf TODO}\xspace}
\newcommand{\DiH}{Diffie-Hellman\xspace}
\newcommand{\proverif}{ProVerif\xspace}
\newcommand{\ofmc}{OFMC\xspace}
\newcommand{\spear}{SPEAR II\xspace}
\newcommand{\opcua}{OPC-UA\xspace}
\newcommand{\opc}{OPC\xspace}
\newcommand{\dcom}{DCOM\xspace}
\newcommand{\mms}{MMS\xspace}
\newcommand{\iec}[1]{IEC\xspace#1\xspace}
\newcommand{\iccp}{ICCP\xspace}
\newcommand{\modbus}{MODBUS\xspace}
\newcommand{\profinet}{PROFINET\xspace}
\newcommand{\etherip}{EtherNet/IP\xspace}
\newcommand{\dnp}{DNP3\xspace}
\newcommand{\eg}{{\it e.g.}\xspace}

\title{Formal Analysis of Security Properties on the \opcua SCADA Protocol}
\author{Pascal Lafourcade \and Marie-Laure Potet \and Maxime Puys}
\date{}

\begin{document}

\maketitle

\begin{abstract}
    Industrial systems are the target of cyberattacks since Stuxnet
    \cite{Lan11}.  Nowadays they communicate over unsecure media such
    as Internet.  Due to their interaction with the physical world, it
    is crucial to prove the security relying protocols.  In this
    paper, we propose a formal study of the security of one of the
    most used industrial protocols: \opcua.  Using
    Proverif, a well known cryptographic protocol verification tool,
    we are able to check secrecy and authentication properties.  We
    find several attacks on the protocols and provide countermeasures
    when needed.
\end{abstract}

\section{Introduction}
Industrial systems also called SCADA (Supervisory Control And Data
Acquisition) have been known to be targeted by cyberattacks since the
famous Stuxnet case~\cite{Lan11} in 2010.  Due to the criticity of
their interaction with the physical world, these systems can
potentially be really harmful for humans and environment.  The
frequence of such attacks is increasing to become one of the priorities for
governmental agencies (\eg~\cite{SFS11} from the US National Institute of
Standards and Technology (NIST) or
\cite{ANSSI12_guide_securite_industrielle_en} from the French {\em Agence
nationale de la sécurité des systèmes d'information} (ANSSI)).
%As
%industrial systems historically have been physically isolated from the
%rest of the world, there were less exposed to cyber attacks and their
%security was less considered. Nowadays, such attacks become feasible
%because these systems are spreading geographically and communicating
%more and more through unsafe mediums like Internet.


%Industrial systems have specificities. First the isolation of such
%systems requires the attacker to be physically present in the
%system. Then compagnies focused more on the protection against natural
%deceases and human mistakes than cybersecurity (often call security).
%In the context of security there is an adversary willing to perform malicious
%actions.
Industrial systems differ from other systems because of
the really long lifetime of there devices and their difficulty to
patch in case of vulnerabilities.
Such specificities encourage to carefully check
standards and applications before deploying them.
%It also explain the number of legacy hosts.
%Moreover, most of industrial protocols are proprietary and
%provide a very low level of security if
%any, \eg \modbus~\cite{MODBUS}, \profinet~\cite{PROFINET}, \etherip~\cite{Bro01}
%or
%\dnp~\cite{CR04}.
%However, in 2006, the {\em OPC Foundation} (an industry consortium)
%released the first version of \opcua~\cite{MLD09}, which is presented
%as the new standard for industrial communications. %% This standard  and whose security
%% is quite closer to business IT's protocols such as TLS~\cite{DR08}.
%% This directly applies to communication protocols.
As it already appeared for business IT's protocols for twenty years,
automated verification is crucial in order to discover flaws in the
protocols' specifications before assessing implementations. However,
the lack of formal verification of industrial protocols has been
emphasized in 2006 by Igure et al.~\cite{ILW06} and in 2009 by
Patel \emph{et al.}~\cite{PBG09}.  They particularly argued that
automated protocol verification allows to understand most of the
vulnerabilities of a protocol before changing its standards which
helps at minimizing the number of revisions which costs time and
money.  Moreover, due to the combinational explosion of the number of
possible execution traces and the increasing complexity of the
systems, only automated verification can ensure the security of a
protocol in a given model.

\TODO Intruduire \opcua qq part


%\subsection{State-of-the-art}\label{sec:intro_sota}
\paragraph{State-of-the-art:}\label{sec:intro_sota}

%The security of industrial protocols becomes a hot topic and some works are
%present in the literature.
Most of the works on the security of industrial protocols only rely on
specifications written in human language rather than using formal methods.
In 2004, Dzung et al. proposed a detailed survey on the security in SCADA
systems including analysis on the security properties offered by \opc (Open
Platform Communications), \mms (Manufacturing Message Specification),
\iec{61850}, \iccp (Inter-Control Center Communications Protocol) and \etherip.
Still in 2004, Clarke et al.~\cite{CR04} discussed the security of \dnp
(Distributed Network Protocol) and \iccp.
In 2006, in the technical documentation of \opcua (OPC Unified Architecture) the
authors detailled the security measures of the protocol (specialy in part 2, 4
and 6).
In 2015, Wanying et al. summarizeed the security offered by \modbus, \dnp and
\opcua.

On the other hand, some works propose new versions of existing protocols to make
them secure against malicious adversaries.
In 2007, Patel et al.~\cite{PY07} studied the security of \dnp and proposed two
ways of enhancing it through digital signatures and challenge-response models.
In 2009, Fovino et al.~\cite{FCMT09} proposed a secure version of \modbus
relying on well-known cryptographic primitives such as RSA or SHA2.
%This version of the protocol also need to introduce new components in the system
%to allow existing devices to use these cryptographic primitives.
In 2013, Hayes et al.~\cite{HE13} designed another secure \modbus protocol using
hash-based message authentication codes (HMACs) and built on STCP (Stream
Transmission Control Protocol).

To the best of our knowledge, the only work directly making use of formal
methods to prove the security of industrial protocols or find attack against
them is from Graham et al.~\cite{GP05} in 2005.
They proposed a formal verification of \dnp using \ofmc~\cite{BMV03} and
\spear~\cite{SH01}.
%% \TODO Sadly we were not able to access their modelings to discuss them with
%% ours.
%In 2008, Dutertre~\cite{Dut08} detailed formal specifications of \modbus
%developed using PVS, a generic theorem prover in order to help proving the
%consistancy of an implementation with the standards.

%\subsection{Contributions}
\paragraph{Contributions:}

We propose a formal analysis of the security of
% standard \modbus protocol and
%the correction proposed in~\cite{FCMT09}.%,HE13}.
%We also assess the 
the sub-protocols involved in the \opcua handshake, namely \opcua
OpenSecureChannel and \opcua CreateSession.  We are able to find
attacks against both of them and provide some countermeasures when
needed.  This work makes use of one of the most efficient tools in the
domain of cryptographic protocol verification according
to \cite{LP15}, namely \proverif developed by Blanchet et al.~\cite{Bla01}.
It analyzes a protocol written either in Horn clauses format or using
a subset of the Pi-calculus for an unbounded number of sessions
considering the classical Dolev-Yao intruder model~\cite{DY81}.  This
powerful intruder controls the network, listens, stops, forges,
replays or modifies some messages according to its capabilities and
knowledge.  The perfect encryption hypothesis is assumed, meaning that
it is not possible to decrypt a ciphertext without its encryption key
or to forge a signature without knowing the secret key.  In this
context, verification tools are able to verify security properties of
a protocol such as secrecy and authentication.  The first property
ensures that a secret message cannot be discovered by an unauthorized
agent (including the intruder).  The authentication property means
that one participant of the protocol is guaranteed to communicate with
another one.

\paragraph{Outline:} In the Section~\ref{sec:secure_channel}, we analyse the
security of \opcua OpenSecureChannel and 
\opcua CreateSession in Section~\ref{sec:session}. Finally, we conclude in Section~\ref{sec:conclusion}.


%\section{Misc}
%
%\paragraph{NonIdent} Considering a powerful intruder, anybody could play the role of the client and
%the role of the server, including the interder.
%In such model, a legitimate client could communicate with a rogue server.
%
%
%\paragraph{Ident} A more realistic model supposes that the client and the server know each other
%and would only communicate together.
%Obviously, these models are equivalent if the protocol does not provide any
%authentication method.
%Integrity only applies in this model since it abstracts hosts identity
%(integrity can be seen as a mono-session authentication).

%% \section{\modbus}
%% \modbus does not provide any security by default.
A message contains an applicative header, a function code and data.
Thus a modbus message can by seen represented as:

\begin{figure}[htb]
    \centering
    \renewcommand{\smname}{\smn}
    \resizebox{\columnwidth}{!}{
        \begin{postscript}
            \begin{msc}{\modbus}
                \setlength{\envinstdist}{1.5\envinstdist}
                \setlength{\instdist}{3\instdist}
                \setlength{\labeldist}{1.5\labeldist}

                \declinst{cli}{C}{}
                \declinst{srv}{S}{}

                \mess{\sm{m$_1$}{}{skC}}{cli}{srv}
                \nextlevel[1.5]

                \mess{\sm{m$_2$}{}{skS}}{srv}{cli}
                \nextlevel[1]
            \end{msc}
        \end{postscript}
    }
    \caption{\modbus request and response}
    \label{fig:modbus_none}
\end{figure}

However in \cite{FCMT09}, Fovino et al, proposed a way to secure \modbus using
classical cryptographic primitives such as hash function and digital signatures.

\begin{figure}[htb]
    \centering
    \renewcommand{\smname}{\sms}
    \resizebox{\columnwidth}{!}{
        \begin{postscript}
            \begin{msc}{\modbus}
                \setlength{\envinstdist}{1.5\envinstdist}
                \setlength{\instdist}{3\instdist}
                \setlength{\labeldist}{1.5\labeldist}

                \declinst{cli}{C}{}
                \declinst{srv}{S}{}

                \mess{\sm{TS$_1$, m$_1$}{}{skC}}{cli}{srv}
                \nextlevel[1.5]

                \mess{\sm{TS$_2$, m$_2$}{}{skS}}{srv}{cli}
                \nextlevel[1]
            \end{msc}
        \end{postscript}
    }
    \caption{Secure \modbus request and response}
    \label{fig:modbus_fix}
\end{figure}

According to their proposition, integrity is provided using a secure hashing
function.
Authentication is guaranted by signing the entire message with a digital public
key signature scheme.
In addition, they added a timestamp information to protect against replay
attacks.
As discrete time cannot be modelised using protocol verification tools (\TODO
Pascal ok ?) we abstract this counter-measure.

\subsection{Modeling}

\modbus has barely any interaction between the client and the server.
The client only sends a request and the server answers a response which can
sometimes be deduced from the request (\eg for WRITE requests).
However, in the general case the response and the request can be modelised as
two distinct messages.

\subsection{Results}

As neither default \modbus nor the version of \cite{FCMT09} provides secrecy,
all messages can be accessed by the intruder.

\begin{table}[htb]
    \centering
    \resizebox{\columnwidth}{!}{
    \begin{tabular}{|c|c|c|c|}
        \hline
        Protocol        & Secrecy   & Integrity & Authentication    \\
        \hline
        Std. \modbus    & \UNSAFE   & \UNSAFE   & \UNSAFE           \\
        \hline
        \cite{FCMT09}   & \UNSAFE   & \SAFE     & \UNSAFE           \\
        \hline
    \end{tabular}
    }
\end{table}

%Attack on integrity with secure modbus and model NonIdent:
%
%\begin{figure}[htb]
%    \centering
%    \renewcommand{\smname}{\sms}
%    \resizebox{.8\columnwidth}{!}{
%        \begin{postscript}
%            \begin{msc}{\modbus}
%                \setlength{\envinstdist}{1.5\envinstdist}
%                \setlength{\instdist}{1.5\instdist}
%                \setlength{\labeldist}{1.5\labeldist}
%
%                \declinst{cli}{C}{}
%                \declinst{i1}{I$_1$}{}
%                \declinst{i2}{I$_2$}{}
%                \declinst{srv}{S}{}
%
%                \mess{\sm{m$_1$}{}{skC}}{cli}{i1}
%                \mess{\sm{m$_2$}{}{skI$_2$}}{i2}{srv}
%                \nextlevel[1]
%            \end{msc}
%        \end{postscript}
%    }
%    \caption{Secure \modbus attack}
%    \label{fig:modbus_fix}
%\end{figure}



%\section{\opcua}\label{sec:opcua}

\section{\opcua OpenSecureChannel}\label{sec:opcuasc}
%\newcommand{\gereq}{GetEndpointRequest}
%\newcommand{\geres}{GetEndpointResponse}
%\newcommand{\oscreq}{OpenSecureChannelRequest}
%\newcommand{\oscres}{OpenSecureChannelResponse}
\newcommand{\gereq}{GEReq}
\newcommand{\geres}{GERes}
\newcommand{\oscreq}{OSCReq}
\newcommand{\oscres}{OSCRes}


The {\em OpenSecureConversation} sub-protocol, aims to authenticate a client and
a server and allow them to generate a shared key for the later communications.
\opcua can be used with three security modes, namely {\em None}, {\em Sign} and
{\em SignAndEncrypt}.

\begin{itemize}
    \item SignAndEncrypt: claims to provide secrecy using symetric and
    assymetric encryption and both authentication and integrity through digital
    signatures.
    It is described in Figure \ref{fig:secure_conv_se}.

    \item Sign: same as {\em SignAndEncrypt} but without any encryption.
    Thus nonces are not used to generate a shared key but bring freshness to the
    messages.
%    Figure \ref{fig:secure_conv_sign} demonstrates this variant.

    \item None: does not provide any security.
    Using this security mode, the {\em OpenSecureConversation} sub-protocol does
    not serve much purpose but is used for compatibility.
%    Figure \ref{fig:secure_conv_none} presents this variant of the sub-protocol.
\end{itemize}

%\begin{figure}[htb]
%    \renewcommand{\smname}{\smn}
%    \resizebox{\textwidth}{!}{
%        \begin{postscript}
%            \begin{msc}{Secure channel creation}
%                \setlength{\envinstdist}{1.5\envinstdist}
%                \setlength{\instdist}{3.5\instdist}
%
%                \declinst{cli}{C}{}
%                \declinst{intruder}{DiscoreryEndpoint}{}
%                \declinst{sess}{S}{}
%
%                \mess{GetEndpointsRequest}[b]{cli}{intruder}
%                \nextlevel[1.5]
%
%                \mess{SP, \smname, UP, pk(S)}[b]{intruder}{cli}
%                \nextlevel[1.5]
%                
%                \action*{Validates pk(S)}{cli}
%                \nextlevel[2]
%                
%                \action*{Generates N$_{C}$}{cli}
%                \nextlevel[2]
%
%                \mess{pk(C), \sm{SP, $g^{N_{C}}$}{pk(S)}{sk(C)}}[b]{cli}{sess}
%                \nextlevel[2]
%
%                \action*{Validates pk(C) }{sess}
%                \nextlevel[2]
%
%                \action*{Generates N$_{S}$ }{sess}
%                \nextlevel[2]
%
%                \ifsmnotnone{%
%                    \action*{$K_{srv}$ = $(g^{N_{C}})^{N_{S}}$ }{sess}
%                    \nextlevel[2]
%                }
%
%                \mess{\sm{$g^{N_{S}}$}{pk(C)}{sk(S)}}[b]{sess}{cli}
%                \nextlevel[2]
%
%                \ifsmnotnone{%
%                    \action*{$K_{cli}$ = $(g^{N_{S}})^{N_{C}}$}{cli}
%                    \nextlevel[2]
%                }
%            \end{msc}
%        \end{postscript}
%    }
%    \caption{\opcua Secure Conversation None}
%    \label{fig:secure_conv_none}
%\end{figure}
%
%\begin{figure}[htb]
%    \renewcommand{\smname}{\sms}
%    \resizebox{\textwidth}{!}{
%        \begin{postscript}
%            \begin{msc}{Secure channel creation}
%                \setlength{\envinstdist}{1.5\envinstdist}
%                \setlength{\instdist}{3.5\instdist}
%
%                \declinst{cli}{C}{}
%                \declinst{intruder}{DiscoreryEndpoint}{}
%                \declinst{sess}{S}{}
%
%                \mess{GetEndpointsRequest}[b]{cli}{intruder}
%                \nextlevel[1.5]
%
%                \mess{SP, \smname, UP, pk(S)}[b]{intruder}{cli}
%                \nextlevel[1.5]
%                
%                \action*{Validates pk(S)}{cli}
%                \nextlevel[2]
%                
%                \action*{Generates N$_{C}$}{cli}
%                \nextlevel[2]
%
%                \mess{pk(C), \sm{SP, $g^{N_{C}}$}{pk(S)}{sk(C)}}[b]{cli}{sess}
%                \nextlevel[2]
%
%                \action*{Validates pk(C) }{sess}
%                \nextlevel[2]
%
%                \action*{Generates N$_{S}$ }{sess}
%                \nextlevel[2]
%
%                \ifsmnotnone{%
%                    \action*{$K_{srv}$ = $(g^{N_{C}})^{N_{S}}$ }{sess}
%                    \nextlevel[2]
%                }
%
%                \mess{\sm{$g^{N_{S}}$}{pk(C)}{sk(S)}}[b]{sess}{cli}
%                \nextlevel[2]
%
%                \ifsmnotnone{%
%                    \action*{$K_{cli}$ = $(g^{N_{S}})^{N_{C}}$}{cli}
%                    \nextlevel[2]
%                }
%            \end{msc}
%        \end{postscript}
%    }
%    \caption{\opcua Secure Conversation Sign}
%    \label{fig:secure_conv_sign}
%\end{figure}

\begin{figure}[htb]
    \renewcommand{\smname}{\smse}
    \resizebox{\textwidth}{!}{
        \begin{postscript}
            \begin{msc}{Secure channel creation}
                \setlength{\envinstdist}{1.5\envinstdist}
                \setlength{\instdist}{3\instdist}

                \declinst{cli}{C}{}
                \declinst{intruder}{DiscoreryEndpoint}{}
                \declinst{sess}{S}{}

                \mess{\gereq}[b]{cli}{intruder}
                \nextlevel[1.5]

                \mess{\geres, pk(S), \smname, SP, UP}[b]{intruder}{cli}
                \nextlevel[1.5]
                
                \action*{Validates pk(S)}{cli}
                \nextlevel[2]
                
                \action*{Generates N$_{C}$}{cli}
                \nextlevel[2]

                \mess{pk(C), \sm{\oscreq, pk(C), $g^{N_{C}}$}{pk(S)}{sk(C)}}[b]{cli}{sess}
                \nextlevel[2]

                \action*{Validates pk(C) }{sess}
                \nextlevel[2]

                \action*{Generates N$_{S}$ }{sess}
                \nextlevel[2]

                \ifsmnotnone{%
                    \action*{$K_{srv}$ = $(g^{N_{C}})^{N_{S}}$ }{sess}
                    \nextlevel[2]
                }

                \mess{\sm{\oscres, $g^{N_{S}}$, ST, TTL}{pk(C)}{sk(S)}}[b]{sess}{cli}
                \nextlevel[2]

                \ifsmnotnone{%
                    \action*{$K_{cli}$ = $(g^{N_{S}})^{N_{C}}$}{cli}
                    \nextlevel[2]
                }
            \end{msc}
        \end{postscript}
    }
    \caption{\opcua OpenSecureConversation sub-protocol}
    \label{fig:secure_conv_se}
\end{figure}

\subsection{Modelisation}

Normaly, a {\em GetEnpointRequest} would be answered by a list of session
endpoints with possibly different security modes.
For simplicity, we suppose that only one endpoint is answered.
Thus, to ensure completeness of our analysis, we verify all combinations of
security modes (\eg a client configured in {\em None} speaking with a server
configured in {\em SignAndEncrypt}, which would not be possible normaly).
This way we can verify if an attacker can exploit the security of a peer as an
oracle to mount an attack against another.
Moreover, thanks to the perfect encryption hypothesis (\TODO expliquer qq part),
we can abstract the security policy messages as we do not differentiate which
cryptographic primitives are used.

Objectives:
\begin{enumerate}
    \item\label{item:sc_sec_cli} Secrecy of $K_{cli}$ obtained by C.
    \item\label{item:sc_sec_srv} Secrecy of $K_{src}$ obtained by S.
    \item\label{item:sc_auth_cli} Authentication on $g^{Nc}$.
    \item\label{item:sc_auth_srv} Authentication on $g^{Ns}$.
\end{enumerate}

\subsection{Results}

We run \proverif on this protocol for each combination of the three security
modes of \opcua for each objective proposed.
Results are shown in Table \ref{tab:secure_conv_results}.

\begin{table}[htb]
    \centering
    \begin{tabular}{|c|c|c|c|c|c|}
        \hline
        \multicolumn{2}{|c}{\opcua Security modes} & \multicolumn{4}{|c|}{Objectives} \\
        \hline
        C              & S              & Sec $K_{cli}$ & Sec $K_{srv}$ & Auth $g^{N_{S}}$  & Auth $g^{N_{C}}$  \\
        \hline                                                                                                 
        None           & None           & \UNSAFE       & \UNSAFE       & \UNSAFE           & \UNSAFE           \\ 
        \hline                                                                                                 
        None           & Sign           & \UNSAFE       & \SAFE         & \UNSAFE           & \SAFE             \\ 
        \hline                                                                                                 
        None           & SignAndEncrypt & \UNSAFE       & \SAFE         & \UNSAFE           & \SAFE             \\ 
        \hline                                                                                                 
        Sign           & None           & \SAFE         & \UNSAFE       & \UNSAFE           & \UNSAFE           \\ 
        \hline                                                                                                 
        Sign           & Sign           & \SAFE         & \SAFE         & \UNSAFE           & \UNSAFE           \\ 
        \hline                                                                                                 
        Sign           & SignAndEncrypt & \SAFE         & \SAFE         & \UNSAFE           & \UNSAFE           \\ 
        \hline                                                                                                 
        SignAndEncrypt & None           & \SAFE         & \UNSAFE       & \UNSAFE           & \UNSAFE           \\ 
        \hline                                                                                                 
        SignAndEncrypt & Sign           & \SAFE         & \SAFE         & \UNSAFE           & \UNSAFE           \\ 
        \hline                                                                                                 
        SignAndEncrypt & SignAndEncrypt & \SAFE         & \SAFE         & \UNSAFE           & \UNSAFE           \\ 
        \hline
    \end{tabular}
    \label{tab:secure_conv_results}
    \caption{Results}
\end{table}

Exemple d'attaque :
\begin{figure}[htb]
    \renewcommand{\smname}{\smse}
    \resizebox{\textwidth}{!}{
        \begin{postscript}
            \begin{msc}{Secure channel creation}
                \setlength{\envinstdist}{1.5\envinstdist}
                \setlength{\instdist}{5.1\instdist}

                \declinst{cli}{C}{}
                \declinst{intruder}{I}{}
                \declinst{sess}{S}{}

                \mess{\gereq}[b]{cli}{intruder}
                \nextlevel[1.5]

                \mess{\geres, pk(I), \smname, SP, UP}[b]{intruder}{cli}
                \nextlevel[1.5]
                
                \action*{Validates pk(I)}{cli}
                \nextlevel[2]
                
                \action*{Generates N$_{C}$}{cli}
                \nextlevel[2]

                \mess{pk(C), \sm{\oscreq, pk(C), $g^{N_{C}}$}{pk(I)}{sk(C)}}[b]{cli}{intruder}
                \nextlevel[2]

                \mess{pk(C), \sm{\oscreq, pk(C), $g^{N_{C}}$}{pk(S)}{sk(C)}}[b]{intruder}{sess}
                \nextlevel[2]

                \action*{Validates pk(C) }{sess}
                \nextlevel[2]

                \action*{Generates N$_{S}$ }{sess}
                \nextlevel[2]

                \ifsmnotnone{%
                    \action*{$K_{srv}$ = $(g^{N_{C}})^{N_{S}}$ }{sess}
                    \nextlevel[2]
                }

                \mess{\sm{\oscres, $g^{N_{S}}$, ST, TTL}{pk(C)}{sk(S)}}[b]{sess}{intruder}
                \nextlevel[1.5]
            \end{msc}
        \end{postscript}
    }
    \caption{\opcua Secure Conversation SignAndEncrypt}
\end{figure}

\subsection{Fixed version}

\begin{figure}[htb]
    \renewcommand{\smname}{\smse}
    \resizebox{\textwidth}{!}{
        \begin{postscript}
            \begin{msc}{Secure channel creation}
                \setlength{\envinstdist}{1.5\envinstdist}
                \setlength{\instdist}{3\instdist}

                \declinst{cli}{C}{}
                \declinst{intruder}{DiscoreryEndpoint}{}
                \declinst{sess}{S}{}

                \mess{\gereq}[b]{cli}{intruder}
                \nextlevel[1.5]

                \mess{\geres, pk(S), \smname, SP, UP}[b]{intruder}{cli}
                \nextlevel[1.5]
                
                \action*{Validates pk(S)}{cli}
                \nextlevel[2]
                
                \action*{Generates N$_{C}$}{cli}
                \nextlevel[2]

                \mess{pk(C), \sm{\oscreq, S, pk(C), $g^{N_{C}}$}{pk(S)}{sk(C)}}[b]{cli}{sess}
                \nextlevel[2]

                \action*{Validates pk(C) }{sess}
                \nextlevel[2]

                \action*{Generates N$_{S}$ }{sess}
                \nextlevel[2]

                \ifsmnotnone{%
                    \action*{$K_{srv}$ = $(g^{N_{C}})^{N_{S}}$ }{sess}
                    \nextlevel[2]
                }

                \mess{\sm{\oscres, C, $g^{N_{S}}$, ST, TTL}{pk(C)}{sk(S)}}[b]{sess}{cli}
                \nextlevel[2]

                \ifsmnotnone{%
                    \action*{$K_{cli}$ = $(g^{N_{S}})^{N_{C}}$}{cli}
                    \nextlevel[2]
                }
            \end{msc}
        \end{postscript}
    }
    \caption{\opcua OpenSecureConversation sub-protocol}
    \label{fig:secure_conv_se}
\end{figure}

\begin{table}[htb]
    \centering
    \begin{tabular}{|c|c|c|c|c|c|}
        \hline
        \multicolumn{2}{|c}{\opcua Security modes} & \multicolumn{4}{|c|}{Objectives} \\
        \hline
        C              & S              & Sec $K_{cli}$ & Sec $K_{srv}$ & Auth $g^{N_{S}}$  & Auth $g^{N_{C}}$  \\
        \hline                                                                                                  
        None           & None           & \UNSAFE       & \UNSAFE       & \UNSAFE           & \UNSAFE           \\ 
        \hline                                                                                                  
        None           & Sign           & \UNSAFE       & \SAFE         & \UNSAFE           & \SAFE             \\ 
        \hline                                                                                                  
        None           & SignAndEncrypt & \UNSAFE       & \SAFE         & \UNSAFE           & \SAFE             \\ 
        \hline                                                                                                  
        Sign           & None           & \SAFE         & \UNSAFE       & \SAFE             & \UNSAFE           \\ 
        \hline                                                                                                  
        Sign           & Sign           & \SAFE         & \SAFE         & \SAFE             & \SAFE             \\ 
        \hline                                                                                                  
        Sign           & SignAndEncrypt & \SAFE         & \SAFE         & \SAFE             & \SAFE             \\ 
        \hline                                                                                                  
        SignAndEncrypt & None           & \SAFE         & \UNSAFE       & \SAFE             & \UNSAFE           \\ 
        \hline                                                                                                  
        SignAndEncrypt & Sign           & \SAFE         & \SAFE         & \SAFE             & \SAFE             \\ 
        \hline                                                                                                  
        SignAndEncrypt & SignAndEncrypt & \SAFE         & \SAFE         & \SAFE             & \SAFE             \\ 
        \hline
    \end{tabular}
    \label{tab:secure_conv_results}
    \caption{Results}
\end{table}


\section{\opcua CreateSession}\label{sec:opcuacs}
%\newcommand{\csreq}{CreateSessionRequest}
%\newcommand{\csres}{CreateSessionResponse}
%\newcommand{\asreq}{ActivateSessionRequest}
%\newcommand{\asres}{ActivateSessionResponse}
\newcommand{\csreq}{CSReq}
\newcommand{\csres}{CSRes}
\newcommand{\asreq}{ASReq}
\newcommand{\asres}{ASRes}

%\begin{figure}[htb]
%    \renewcommand{\smname}{\smn}
%    \resizebox{\textwidth}{!}{
%        \begin{postscript}
%            \begin{msc}{Session creation}
%                \setlength{\envinstdist}{1.5\envinstdist}
%                \setlength{\instdist}{5.7\instdist}
%
%                \declinst{cli}{C}{}
%                \declinst{sess}{S}{}
%
%                \mess{\sm{\csreq}{$K_{cli}$}{sk(C)}}[b]{cli}{sess}
%                \nextlevel[1.5]
%
%                \mess{\sm{SSC, $N_{s2}$}{$K_{srv}$}{sk(S)}}[b]{sess}{cli}
%                \nextlevel[1.5]
%                
%                \action*{Validates SSC}{cli}
%                \nextlevel[2]
%                
%                \action*{Verifies the signature of $N_{s2}$}{cli}
%                \nextlevel[2]
%                
%                \mess{\sm{CSC, Login, Passwd}{$K_{cli}$}{sk(C)}}[b]{cli}{sess}
%                \nextlevel[2]
%                
%                \action*{Validates CSC}{sess}
%                \nextlevel[2]
%
%                \action*{Validates (Login, Passwd) }{sess}
%                \nextlevel[2]
%
%                \mess{\sm{\asres}{$K_{src}$}{sk(S)}}[b]{sess}{cli}
%                \nextlevel[2]
%            \end{msc}
%        \end{postscript}
%    }
%    \caption{OPC-UA Session None}
%\end{figure}


%\begin{figure}[htb]
%    \renewcommand{\smname}{\sms}
%    \resizebox{\textwidth}{!}{
%        \begin{postscript}
%            \begin{msc}{Session creation}
%                \setlength{\envinstdist}{1.5\envinstdist}
%                \setlength{\instdist}{5.7\instdist}
%
%                \declinst{cli}{C}{}
%                \declinst{sess}{S}{}
%
%                \mess{\sm{\csreq}{$K_{cli}$}{sk(C)}}[b]{cli}{sess}
%                \nextlevel[1.5]
%
%                \mess{\sm{SSC, $N_{s2}$}{$K_{srv}$}{sk(S)}}[b]{sess}{cli}
%                \nextlevel[1.5]
%                
%                \action*{Validates SSC}{cli}
%                \nextlevel[2]
%                
%                \action*{Verifies the signature of $N_{s2}$}{cli}
%                \nextlevel[2]
%                
%                \mess{\sm{CSC, Login, Passwd}{$K_{cli}$}{sk(C)}}[b]{cli}{sess}
%                \nextlevel[2]
%                
%                \action*{Validates CSC}{sess}
%                \nextlevel[2]
%
%                \action*{Validates (Login, Passwd) }{sess}
%                \nextlevel[2]
%
%                \mess{\sm{\asres}{$K_{src}$}{sk(S)}}[b]{sess}{cli}
%                \nextlevel[2]
%                \end{msc}
%        \end{postscript}
%    }
%    \caption{OPC-UA Session Sign}
%\end{figure}


The \opcua {\em CreateSession} sub-protocol allows a client to send credentials
(e.g.: a login and a password) over an already created Secure Channel.
This sub-protocol is presented in Figure \ref{fig:session_se}.

\begin{figure}[htb]
    \renewcommand{\smname}{\smse}
    \resizebox{\textwidth}{!}{
        \begin{postscript}
            \begin{msc}{Session creation}
                \setlength{\envinstdist}{1.5\envinstdist}
                \setlength{\instdist}{6\instdist}

                \declinst{cli}{C}{}
                \declinst{sess}{S}{}

                \mess{\sm{\csreq, pk(C), $N_{C}$}{$K_{cli}$}{sk(C)}}[b]{cli}{sess}
                \nextlevel[1.5]
                
                \action*{Validates pk(C)}{sess}
                \nextlevel[2]

                \mess{\sm{\csres, pk(S) $N_{S}$}{$K_{srv}$}{sk(S)}}[b]{sess}{cli}
                \nextlevel[1.5]
                
                \action*{Validates pk(S)}{cli}
                \nextlevel[2]
                
                \mess{\sm{\asreq, pk(C), Login, Passwd}{$K_{cli}$}{sk(C)}}[b]{cli}{sess}
                \nextlevel[2]
                
                \action*{Validates pk(C)}{sess}
                \nextlevel[2]

                \action*{Validates (Login, Passwd) }{sess}
                \nextlevel[2]

                \mess{\sm{\asres, $N_{S2}$}{$K_{src}$}{sk(S)}}[b]{sess}{cli}
                \nextlevel[2]
            \end{msc}
        \end{postscript}
    }
    \label{fig:session_se}
    \caption{OPC-UA CreateSession sub-protocol}
\end{figure}

For this protocol, a modeling choice appears.
We can either suppose that C respects security standards and would use different
credentials for each server or suppose he will always send the same login and
password for all servers.

We first suppose that C always sends the same credentials including in a session
with the intruder (possibly a rogue server).
%Thus, once the intruder obtained C's Login/Passwd, he can mount an 
%authentication attack in another session.
Results under this assumption are presented in Table \ref{tab:session_results}.

\begin{table}[htb]
    \centering
    \begin{tabular}{|c|c|c|c|c|}
        \hline
        \multicolumn{2}{|c}{\opcua Security modes} & \multicolumn{3}{|c|}{Objectives}   \\
        \hline
        C              & S              & Sec $K$       & Sec $Creds$   & Auth $Creds$  \\
        \hline
        None           & None           & \SAFE         & \UNSAFE       & \UNSAFE       \\ 
        \hline
        None           & Sign           & \SAFE         & \UNSAFE       & \SAFE         \\ 
        \hline
        None           & SignAndEncrypt & \SAFE         & \UNSAFE       & \SAFE         \\ 
        \hline
        Sign           & None           & \SAFE         & \UNSAFE       & \UNSAFE       \\ 
        \hline
        Sign           & Sign           & \SAFE         & \UNSAFE       & \UNSAFE       \\ 
        \hline
        Sign           & SignAndEncrypt & \SAFE         & \UNSAFE       & \SAFE         \\ 
        \hline
        SignAndEncrypt & None           & \SAFE         & \UNSAFE       & \UNSAFE       \\ 
        \hline
        SignAndEncrypt & Sign           & \SAFE         & \UNSAFE       & \UNSAFE       \\ 
        \hline
        SignAndEncrypt & SignAndEncrypt & \SAFE         & \UNSAFE       & \SAFE         \\ 
        \hline
    \end{tabular}
    \label{tab:session_results}
    \caption{Results for \opcua {\em CreateSession} sub-protocol}
\end{table}

\subsection{Fixed version}

We apply the same fix than in OpenSecureConversation, the resulting protocol is
displayed in Figure \ref{fig:session_fix}

\begin{figure}[htb]
    \renewcommand{\smname}{\smse}
    \resizebox{\textwidth}{!}{
        \begin{postscript}
            \begin{msc}{Session creation}
                \setlength{\envinstdist}{1.5\envinstdist}
                \setlength{\instdist}{6.25\instdist}

                \declinst{cli}{C}{}
                \declinst{sess}{S}{}

                \mess{\sm{\csreq, pk(C), $N_{C}$}{$K_{cli}$}{sk(C)}}[b]{cli}{sess}
                \nextlevel[1.5]
                
                \action*{Validates pk(C)}{sess}
                \nextlevel[2]

                \mess{\sm{\csres, pk(S) $N_{S}$}{$K_{srv}$}{sk(S)}}[b]{sess}{cli}
                \nextlevel[1.5]
                
                \action*{Validates pk(S)}{cli}
                \nextlevel[2]
                
                \mess{\sm{\asreq, S, pk(C), Login, Passwd}{$K_{cli}$}{sk(C)}}[b]{cli}{sess}
                \nextlevel[2]
                
                \action*{Validates pk(C)}{sess}
                \nextlevel[2]

                \action*{Validates (Login, Passwd) }{sess}
                \nextlevel[2]

                \mess{\sm{\asres, $N_{S2}$}{$K_{src}$}{sk(S)}}[b]{sess}{cli}
                \nextlevel[2]
            \end{msc}
        \end{postscript}
    }
    \label{fig:session_fix}
    \caption{OPC-UA fixed CreateSession sub-protocol}
\end{figure}

Then authentication becomes secure as shown in Table \ref{tab:session_fix_results}.

\begin{table}[htb]
    \centering
    \begin{tabular}{|c|c|c|c|c|}
        \hline
        \multicolumn{2}{|c}{\opcua Security modes} & \multicolumn{3}{|c|}{Objectives}   \\
        \hline
        C              & S              & Sec $K$       & Sec $Creds$   & Auth $Creds$  \\
        \hline
        None           & None           & \SAFE         & \UNSAFE       & \UNSAFE       \\ 
        \hline
        None           & Sign           & \SAFE         & \UNSAFE       & \SAFE         \\ 
        \hline
        None           & SignAndEncrypt & \SAFE         & \UNSAFE       & \SAFE         \\ 
        \hline
        Sign           & None           & \SAFE         & \UNSAFE       & \UNSAFE       \\ 
        \hline
        Sign           & Sign           & \SAFE         & \UNSAFE       & \SAFE         \\ 
        \hline
        Sign           & SignAndEncrypt & \SAFE         & \UNSAFE       & \SAFE         \\ 
        \hline
        SignAndEncrypt & None           & \SAFE         & \UNSAFE       & \UNSAFE       \\ 
        \hline
        SignAndEncrypt & Sign           & \SAFE         & \UNSAFE       & \SAFE         \\ 
        \hline
        SignAndEncrypt & SignAndEncrypt & \SAFE         & \UNSAFE       & \SAFE         \\ 
        \hline
    \end{tabular}
    \label{tab:session_fix_results}
    \caption{Results}
\end{table}

In a second time, we suppose that C would use different credentials for each server.
Then both normal and fixed CreateSession results in Table \ref{tab:session_uniq_creds_results}

\begin{table}[htb]
    \centering
    \begin{tabular}{|c|c|c|c|c|}
        \hline
        \multicolumn{2}{|c}{\opcua Security modes} & \multicolumn{3}{|c|}{Objectives}   \\
        \hline
        C              & S              & Sec $K$       & Sec $Creds$   & Auth $Creds$  \\
        \hline
        None           & None           & \SAFE         & \UNSAFE       & \SAFE         \\ 
        \hline
        None           & Sign           & \SAFE         & \UNSAFE       & \SAFE         \\ 
        \hline
        None           & SignAndEncrypt & \SAFE         & \UNSAFE       & \SAFE         \\ 
        \hline
        Sign           & None           & \SAFE         & \SAFE         & \SAFE         \\ 
        \hline
        Sign           & Sign           & \SAFE         & \UNSAFE       & \SAFE         \\ 
        \hline
        Sign           & SignAndEncrypt & \SAFE         & \UNSAFE       & \SAFE         \\ 
        \hline
        SignAndEncrypt & None           & \SAFE         & \SAFE         & \SAFE         \\ 
        \hline
        SignAndEncrypt & Sign           & \SAFE         & \SAFE         & \SAFE         \\ 
        \hline
        SignAndEncrypt & SignAndEncrypt & \SAFE         & \SAFE         & \SAFE         \\ 
        \hline
    \end{tabular}
    \label{tab:session_uniq_creds_results}
    \caption{Results}
\end{table}


%\section{Flow integrity}
%%TODO \input{flow}
%Idée: Modéliser l'envoie et la réception de messages jusqu'à fermeture du cannal.
%A la fin envoyer les messages envoyés et reçus à un oracle qui les compare.
%
%\section{TODO list}
%
%\begin{itemize}
%    \item Comment s'articulent la signature et le chiffrement (signature d'un haché, d'un chiffré, ...) ?
%    \item Logjam $\Rightarrow$ Oui, vulnérable si hypothèse de DH OK
%    \item Quel type de chiffrement (CBC, ...) ?
%    \item Est-ce que tout est réellement signé (constantes publiques) ?
%\end{itemize}

\bibliographystyle{unsrt}
\bibliography{phdBiblio}

\end{document}

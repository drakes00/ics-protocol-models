%\documentclass[a4paper]{article}
\documentclass[a4paper, twocolumn]{article}

\usepackage[T1]{fontenc}
\usepackage[utf8]{inputenc}

\usepackage{graphicx}
\usepackage{xstring}
\usepackage{xspace}

\usepackage{auto-pst-pdf}
    \usepackage{msc}


% ========== Don't touch ==========
\makeatletter
    \def\msc@frame{no}
    \def\msc@settitle{}
\makeatother

\newcommand{\dosmnone}[3]{#1}
\newcommand{\dosmsign}[3]{#1, $\left\{\mbox{h(#1)}\right\}_{\mbox{#3}}$}
\newcommand{\dosmsignandencrypt}[3]{$\left\{\mbox{#1}\right\}_{\mbox{#2}}$, $\left\{\mbox{h(#1)}\right\}_{\mbox{#3}}$}

\newcommand{\smn}{None}
\newcommand{\sms}{Sign}
\newcommand{\smse}{SignAndEncrypt}
\newcommand{\smseshort}{SignEnc}

\newcommand{\sm}{\IfStrEq{\smname}{\smn}{\dosmnone}{\IfStrEq{\smname}{\sms}{\dosmsign}{\dosmsignandencrypt}}}

\newcommand{\ifsmnotnone}[1]{\IfStrEq{\smname}{\smn}{}{#1}}
\newcommand{\smname}{\smn}
% ========== Don't touch ==========

\newcommand{\UNSAFE}{{\color{red!50!black} UNSAFE}\xspace}
\newcommand{\SAFE}{{\color{green!50!black} SAFE}\xspace}
\newcommand{\TODO}{{\color{red}\bf TODO}\xspace}
\newcommand{\proverif}{ProVerif\xspace}
\newcommand{\ofmc}{OFMC\xspace}
\newcommand{\spear}{SPEAR II\xspace}
\newcommand{\opcua}{OPC-UA\xspace}
\newcommand{\opc}{OPC\xspace}
\newcommand{\dcom}{DCOM\xspace}
\newcommand{\mms}{MMS\xspace}
\newcommand{\iec}[1]{IEC\xspace#1\xspace}
\newcommand{\iccp}{ICCP\xspace}
\newcommand{\modbus}{MODBUS\xspace}
\newcommand{\profinet}{PROFINET\xspace}
\newcommand{\etherip}{EtherNet/IP\xspace}
\newcommand{\dnp}{DNP3\xspace}
\newcommand{\eg}{{\it e.g.}\xspace}

\title{Formal Analysis of Security Properties on SCADA Protocols: \opcua and \modbus}
\author{Pascal Lafourcade \and Marie-Laure Potet \and Maxime Puys}
\date{}

\begin{document}

\maketitle

\begin{abstract}
    Industrial systems are the target of cyberattacks since Stuxnet \cite{Lan11}.
    They now communicate over unsecure media such as Internet and due to their
    interaction with the physical world, it is crucial to prove the security of
    their dedicated protocols.
    In this paper, we propose a formal study of the security of two of the most
    used industrial protocols : \modbus and \opcua.
    Using famous cryptographic protocol verification tools, we are able to check
    secrecy and authentication properties.
    We find attacks on both protocols and provide countermeasure when needed.
\end{abstract}

\section{Introduction}
\begin{itemize}
    \item La plus part des filtres sont stateless.
    \item Problématique car exemple sectionneur (+exemple aurora ?)
    \item Cependant stateful implique single point of passage == single point of failure
\end{itemize}

\paragraph{Contributions:} Stateful filtering with worst-case bandwidth.

\paragraph{Outline:} 


%\section{Misc}
%
%\paragraph{NonIdent} Considering a powerful intruder, anybody could play the role of the client and
%the role of the server, including the interder.
%In such model, a legitimate client could communicate with a rogue server.
%
%
%\paragraph{Ident} A more realistic model supposes that the client and the server know each other
%and would only communicate together.
%Obviously, these models are equivalent if the protocol does not provide any
%authentication method.
%Integrity only applies in this model since it abstracts hosts identity
%(integrity can be seen as a mono-session authentication).

\section{\modbus}
\modbus does not provide any security by default.
A message contains an applicative header, a function code and data.
Thus a modbus message can by seen represented as:

\begin{figure}[htb]
    \centering
    \renewcommand{\smname}{\smn}
    \resizebox{\columnwidth}{!}{
        \begin{postscript}
            \begin{msc}{\modbus}
                \setlength{\envinstdist}{1.5\envinstdist}
                \setlength{\instdist}{3\instdist}
                \setlength{\labeldist}{1.5\labeldist}

                \declinst{cli}{C}{}
                \declinst{srv}{S}{}

                \mess{\sm{m$_1$}{}{skC}}{cli}{srv}
                \nextlevel[1.5]

                \mess{\sm{m$_2$}{}{skS}}{srv}{cli}
                \nextlevel[1]
            \end{msc}
        \end{postscript}
    }
    \caption{\modbus request and response}
    \label{fig:modbus_none}
\end{figure}

However in \cite{FCMT09}, Fovino et al, proposed a way to secure \modbus using
classical cryptographic primitives such as hash function and digital signatures.

\begin{figure}[htb]
    \centering
    \renewcommand{\smname}{\sms}
    \resizebox{\columnwidth}{!}{
        \begin{postscript}
            \begin{msc}{\modbus}
                \setlength{\envinstdist}{1.5\envinstdist}
                \setlength{\instdist}{3\instdist}
                \setlength{\labeldist}{1.5\labeldist}

                \declinst{cli}{C}{}
                \declinst{srv}{S}{}

                \mess{\sm{TS$_1$, m$_1$}{}{skC}}{cli}{srv}
                \nextlevel[1.5]

                \mess{\sm{TS$_2$, m$_2$}{}{skS}}{srv}{cli}
                \nextlevel[1]
            \end{msc}
        \end{postscript}
    }
    \caption{Secure \modbus request and response}
    \label{fig:modbus_fix}
\end{figure}

According to their proposition, integrity is provided using a secure hashing
function.
Authentication is guaranted by signing the entire message with a digital public
key signature scheme.
In addition, they added a timestamp information to protect against replay
attacks.
As discrete time cannot be modelised using protocol verification tools (\TODO
Pascal ok ?) we abstract this counter-measure.

\subsection{Modeling}

\modbus has barely any interaction between the client and the server.
The client only sends a request and the server answers a response which can
sometimes be deduced from the request (\eg for WRITE requests).
However, in the general case the response and the request can be modelised as
two distinct messages.

\subsection{Results}

As neither default \modbus nor the version of \cite{FCMT09} provides secrecy,
all messages can be accessed by the intruder.

\begin{table}[htb]
    \centering
    \resizebox{\columnwidth}{!}{
    \begin{tabular}{|c|c|c|c|}
        \hline
        Protocol        & Secrecy   & Integrity & Authentication    \\
        \hline
        Std. \modbus    & \UNSAFE   & \UNSAFE   & \UNSAFE           \\
        \hline
        \cite{FCMT09}   & \UNSAFE   & \SAFE     & \UNSAFE           \\
        \hline
    \end{tabular}
    }
\end{table}

%Attack on integrity with secure modbus and model NonIdent:
%
%\begin{figure}[htb]
%    \centering
%    \renewcommand{\smname}{\sms}
%    \resizebox{.8\columnwidth}{!}{
%        \begin{postscript}
%            \begin{msc}{\modbus}
%                \setlength{\envinstdist}{1.5\envinstdist}
%                \setlength{\instdist}{1.5\instdist}
%                \setlength{\labeldist}{1.5\labeldist}
%
%                \declinst{cli}{C}{}
%                \declinst{i1}{I$_1$}{}
%                \declinst{i2}{I$_2$}{}
%                \declinst{srv}{S}{}
%
%                \mess{\sm{m$_1$}{}{skC}}{cli}{i1}
%                \mess{\sm{m$_2$}{}{skI$_2$}}{i2}{srv}
%                \nextlevel[1]
%            \end{msc}
%        \end{postscript}
%    }
%    \caption{Secure \modbus attack}
%    \label{fig:modbus_fix}
%\end{figure}



\section{\opcua}\label{sec:opcua}
\subsection{\opcua OpenSecureChannel}
\begin{figure}[h!]
    \renewcommand{\smname}{\smn}
    \resizebox{\textwidth}{!}{
        \begin{postscript}
            \begin{msc}{Secure channel creation}
                \setlength{\envinstdist}{1.5\envinstdist}
                \setlength{\instdist}{3.5\instdist}

                \declinst{cli}{C}{}
                \declinst{intruder}{DiscoreryEndpoint}{}
                \declinst{sess}{S}{}

                \mess{GetEndpointsRequest}[b]{cli}{intruder}
                \nextlevel[1.5]

                \mess{SP, \smname, UP, pk(S)}[b]{intruder}{cli}
                \nextlevel[1.5]
                
                \action*{Validates pk(S)}{cli}
                \nextlevel[2]
                
                \action*{Generates N$_{C}$}{cli}
                \nextlevel[2]

                \mess{pk(C), \sm{SP, $g^{N_{C}}$}{pk(S)}{sk(C)}}[b]{cli}{sess}
                \nextlevel[2]

                \action*{Validates pk(C) }{sess}
                \nextlevel[2]

                \action*{Generates N$_{S}$ }{sess}
                \nextlevel[2]

                \ifsmnotnone{%
                    \action*{$K_{srv}$ = $(g^{N_{C}})^{N_{S}}$ }{sess}
                    \nextlevel[2]
                }

                \mess{\sm{$g^{N_{S}}$}{pk(C)}{sk(S)}}[b]{sess}{cli}
                \nextlevel[2]

                \ifsmnotnone{%
                    \action*{$K_{cli}$ = $(g^{N_{S}})^{N_{C}}$}{cli}
                    \nextlevel[2]
                }
            \end{msc}
        \end{postscript}
    }
    \caption{OPC-UA Secure Channel None}
\end{figure}


\begin{figure}[h!]
    \renewcommand{\smname}{\sms}
    \resizebox{\textwidth}{!}{
        \begin{postscript}
            \begin{msc}{Secure channel creation}
                \setlength{\envinstdist}{1.5\envinstdist}
                \setlength{\instdist}{3.5\instdist}

                \declinst{cli}{C}{}
                \declinst{intruder}{DiscoreryEndpoint}{}
                \declinst{sess}{S}{}

                \mess{GetEndpointsRequest}[b]{cli}{intruder}
                \nextlevel[1.5]

                \mess{SP, \smname, UP, pk(S)}[b]{intruder}{cli}
                \nextlevel[1.5]
                
                \action*{Validates pk(S)}{cli}
                \nextlevel[2]
                
                \action*{Generates N$_{C}$}{cli}
                \nextlevel[2]

                \mess{pk(C), \sm{SP, $g^{N_{C}}$}{pk(S)}{sk(C)}}[b]{cli}{sess}
                \nextlevel[2]

                \action*{Validates pk(C) }{sess}
                \nextlevel[2]

                \action*{Generates N$_{S}$ }{sess}
                \nextlevel[2]

                \ifsmnotnone{%
                    \action*{$K_{srv}$ = $(g^{N_{C}})^{N_{S}}$ }{sess}
                    \nextlevel[2]
                }

                \mess{\sm{$g^{N_{S}}$}{pk(C)}{sk(S)}}[b]{sess}{cli}
                \nextlevel[2]

                \ifsmnotnone{%
                    \action*{$K_{cli}$ = $(g^{N_{S}})^{N_{C}}$}{cli}
                    \nextlevel[2]
                }
            \end{msc}
        \end{postscript}
    }
    \caption{OPC-UA Secure Channel Sign}
\end{figure}


\begin{figure}[h!]
    \renewcommand{\smname}{\smse}
    \resizebox{\textwidth}{!}{
        \begin{postscript}
            \begin{msc}{Secure channel creation}
                \setlength{\envinstdist}{1.5\envinstdist}
                \setlength{\instdist}{3.5\instdist}

                \declinst{cli}{C}{}
                \declinst{intruder}{DiscoreryEndpoint}{}
                \declinst{sess}{S}{}

                \mess{GetEndpointsRequest}[b]{cli}{intruder}
                \nextlevel[1.5]

                \mess{SP, \smname, UP, pk(S)}[b]{intruder}{cli}
                \nextlevel[1.5]
                
                \action*{Validates pk(S)}{cli}
                \nextlevel[2]
                
                \action*{Generates N$_{C}$}{cli}
                \nextlevel[2]

                \mess{pk(C), \sm{SP, $g^{N_{C}}$}{pk(S)}{sk(C)}}[b]{cli}{sess}
                \nextlevel[2]

                \action*{Validates pk(C) }{sess}
                \nextlevel[2]

                \action*{Generates N$_{S}$ }{sess}
                \nextlevel[2]

                \ifsmnotnone{%
                    \action*{$K_{srv}$ = $(g^{N_{C}})^{N_{S}}$ }{sess}
                    \nextlevel[2]
                }

                \mess{\sm{$g^{N_{S}}$}{pk(C)}{sk(S)}}[b]{sess}{cli}
                \nextlevel[2]

                \ifsmnotnone{%
                    \action*{$K_{cli}$ = $(g^{N_{S}})^{N_{C}}$}{cli}
                    \nextlevel[2]
                }
            \end{msc}
        \end{postscript}
    }
    \caption{OPC-UA Secure Channel SignAndEncrypt}
\end{figure}

Objectives:
\begin{enumerate}
    \item\label{item:sc_sec_cli} Secrecy of $K_{cli}$ shared between C and S
    \item\label{item:sc_sec_srv} Secrecy of $K_{src}$ shared between C and S
    \item\label{item:sc_auth_cli} Authentication on $g^{Nc}$
    \item\label{item:sc_auth_srv} Authentication on $g^{Ns}$
\end{enumerate}

\subsection{Results}

\begin{table}[h!]
    \centering
    \begin{tabular}{|c|c|c|c|c|c|}
        \hline
        \multicolumn{2}{|c}{Security modes} & \multicolumn{4}{|c|}{Objectives} \\
        \hline
        Client          & S   & \ref{item:sc_sec_cli} & \ref{item:sc_sec_srv} & \ref{item:sc_auth_cli}    & \ref{item:sc_auth_srv}    \\
        \hline
        None            & None              & \UNSAFE{}               & \UNSAFE{}               & \UNSAFE{}                   & \UNSAFE{}                   \\ 
        \hline
        None            & Sign              & \UNSAFE{}               & \SAFE{}                 & \SAFE{}                     & \UNSAFE{}                   \\ 
        \hline
        None            & SignAndEncrypt    & \UNSAFE{}               & \SAFE{}                 & \SAFE{}                     & \UNSAFE{}                   \\ 
        \hline
        Sign            & None              & \SAFE{}                 & \UNSAFE{}               & \UNSAFE{}                   & \UNSAFE{}                   \\ 
        \hline
        Sign            & Sign              & \SAFE{}                 & \SAFE{}                 & \UNSAFE{}                   & \UNSAFE{}                   \\ 
        \hline
        Sign            & SignAndEncrypt    & \SAFE{}                 & \SAFE{}                 & \UNSAFE{}                   & \UNSAFE{}                   \\ 
        \hline
        SignAndEncrypt  & None              & \SAFE{}                 & \UNSAFE{}               & \UNSAFE{}                   & \UNSAFE{}                   \\ 
        \hline
        SignAndEncrypt  & Sign              & \SAFE{}                 & \SAFE{}                 & \UNSAFE{}                   & \UNSAFE{}                   \\ 
        \hline
        SignAndEncrypt  & SignAndEncrypt    & \SAFE{}                 & \SAFE{}                 & \UNSAFE{}                   & \UNSAFE{}                   \\ 
        \hline
    \end{tabular}
    \caption{Results}
\end{table}

Exemple d'attaque :
\begin{figure}[h!]
    \renewcommand{\smname}{\smse}
    \resizebox{\textwidth}{!}{
        \begin{postscript}
            \begin{msc}{Secure channel creation}
                \setlength{\envinstdist}{1.5\envinstdist}
                \setlength{\instdist}{3.5\instdist}

                \declinst{cli}{C}{}
                \declinst{intruder}{I}{}
                \declinst{sess}{S}{}

                \mess{GetEndpointsRequest}[b]{cli}{intruder}
                \nextlevel[1.5]

                \mess{SP, \smname, UP, pk(I)}[b]{intruder}{cli}
                \nextlevel[1.5]
                
                \action*{Validates pk(I)}{cli}
                \nextlevel[2]
                
                \action*{Generates N$_{C}$}{cli}
                \nextlevel[2]

                \mess{pk(C), \sm{SP, $g^{N_{C}}$}{pk(I)}{sk(C)}}[b]{cli}{intruder}
                \mess{pk(C), \sm{SP, $g^{N_{C}}$}{pk(S)}{sk(C)}}[b]{intruder}{sess}
                \nextlevel[2]

                \action*{Validates pk(C) }{sess}
                \nextlevel[2]

                \action*{Generates N$_{S}$ }{sess}
                \nextlevel[2]

                \ifsmnotnone{%
                    \action*{$K_{srv}$ = $(g^{N_{C}})^{N_{S}}$ }{sess}
                    \nextlevel[2]
                }

                \mess{\sm{$g^{N_{S}}$}{pk(C)}{sk(S)}}[b]{sess}{intruder}
            \end{msc}
        \end{postscript}
    }
    \caption{OPC-UA Secure Channel SignAndEncrypt}
\end{figure}

NB: Dans la plus part des attaques, le client parle avec un autre client, problème ?\\
L'open secure channel request devrait contenir la pk du serveur dans la signature.
\begin{itemize}
    \item Le client atteste vouloir parler au serveur.
\end{itemize}


\subsection{\opcua CreateSession}
%\newcommand{\csreq}{CreateSessionRequest}
%\newcommand{\csres}{CreateSessionResponse}
%\newcommand{\asreq}{ActivateSessionRequest}
%\newcommand{\asres}{ActivateSessionResponse}
\newcommand{\csreq}{CSReq}
\newcommand{\csres}{CSRes}
\newcommand{\asreq}{ASReq}
\newcommand{\asres}{ASRes}


The \opcua {\em CreateSession} sub-protocol allows a client to send credentials
(\emph{e.g.}: a login and a password) over an already created Secure Channel.
It  automatically follows the security mode that was chosen during the
{\em OpenSecureChannel} sub-protocol.
This sub-protocol is presented in Figure \ref{fig:session_se}.

\begin{figure}[htb]
    \renewcommand{\smname}{\smse}
    \resizebox{\columnwidth}{!}{
        \begin{postscript}
            \begin{msc}{Session creation}
                \setlength{\envinstdist}{1.5\envinstdist}
                \setlength{\instdist}{5.5\instdist}
                \setlength{\labeldist}{1.5\labeldist}

                \declinst{cli}{C}{}
                \declinst{sess}{S}{}

                \mess{\sm{\csreq, pk(C), $N_{C}$}{$K_{CS}$}{sk(C)}}{cli}{sess}
                \nextlevel[1.5]
                
                %\action*{Validates pk(C)}{sess}
                %\nextlevel[2.5]

                \mess{\sm{\csres, pk(S), $N_{C}$, $N_{S}$}{$K_{CS}$}{sk(S)}}{sess}{cli}
                \nextlevel[1.5]
                
                %\action*{Validates pk(S)}{cli}
                %\nextlevel[3]
                
                \mess{\sm{\asreq, pk(C), Login, Passwd}{$K_{CS}$}{sk(C)}}{cli}{sess}
                \nextlevel[1.5]
                
                %\action*{Validates pk(C)}{sess}
                %\nextlevel[2]

                \action*{Validates (Login, Passwd) }{sess}
                \nextlevel[2.5]

                \mess{\sm{\asres, $N_{S2}$}{$K_{CS}$}{sk(S)}}{sess}{cli}
                \nextlevel[.5]
            \end{msc}
        \end{postscript}
    }
    \caption{OPC-UA CreateSession sub-protocol}
    \label{fig:session_se}
\end{figure}

\subsection{Modeling}

%For this protocol, a modeling choice appears.
%We can either suppose that C respects security standards and would use different
%credentials for each server or suppose he will always send the same login and
%password for all servers.
As this protocol involves login and passwords, we have to make modeling choices
on how they are used.
Here we suppose that C uses a different password for each server he speaks with.
%the same credentials including in a session
%with the intruder (possibly a rogue server controled by the intruder).
\TODO Explain modelization.

Moreover in our modeling, we assume that the intruder can share a symmetric key
with any other participant as a result of playing the {\em OpenSecureChannel}
sub-protocol.
However, as we said in Section \ref{sec:secure_channel_res}, he will not obtain
the derived key if he usurps another participant.
This means that if the intruder shares a symmetric key with another participant,
then this key is attached to a session in which he is not usurping someone's
identity.
We consider for security objectives: (i) the secrecy of the credentials, (ii)
the authentication of C on his credentials and (iii) the authentication of S on
$g^{N_{S}}$.

\subsection{Results}

Results under this assumption are presented in Table \ref{tab:session_results}.
Again obviously, all objectives are attacked in security mode \smn.
Also the secrecy of the password can not hold even in security mode \sms~since
it will be revealed by the client to the server during a legitimate exchange.
However, using a different password for each server ensures the authentication
of the client with other security modes.
Attack on the athentication on $N_{S}$ is similar to the attack presented in
Figure \ref{fig:secure_channel_atk}.
The intruder starts a session with the server and retoutes the packets to the
client.
However, such kind of attacks are circumvented in security mode \smse~because
as we previously said, the intruder does not possess the symmetric key of a
session where he is usurping someone's identity preventing him to communicate.
Attacks on the secrecy of the credentials occur because the client will accept
the intruder's public key (and signatures) according to our intruder model.
Thus as the client always uses the same credentials, then he will reveal them to
the intruder just by playing the sub-protocol.

\begin{table}[htb]
    \centering
    %\resizebox{1.05\columnwidth}{!}{
    \begin{tabular}{|c|c|c|c|}
        \hline
        \multirow{2}{*}{\opcua Security mode} & \multicolumn{3}{|c|}{Objectives} \\
        \cline{2-4}
                       & Sec $Creds$   & Auth $N_S$    & Auth $Creds$   \\
        \hline
        \smn           & \UNSAFE       & \UNSAFE       & \UNSAFE        \\ 
        \hline
        \sms           & \UNSAFE       & \UNSAFE       & \SAFE          \\ 
        \hline
        \smseshort     & \SAFE         & \SAFE         & \SAFE          \\ 
        \hline
    \end{tabular}
    %}
    \caption{Results for \opcua {\em CreateSession} sub-protocol}
    \label{tab:session_results}
\end{table}

\subsection{Fixed version}

We apply the same correction than in {\em OpenSecureChannel} (explicitly
specifying the identities of recievers in messages).
%, the resulting protocol is displayed in Figure \ref{fig:session_fix}
Then authentication on $N_{S}$ becomes secure as shown in Table
\ref{tab:session_fix_results}.

%\begin{figure}[htb]
%    \renewcommand{\smname}{\smse}
%    \resizebox{\columnwidth}{!}{
%        \begin{postscript}
%            \begin{msc}{Session creation}
%                \setlength{\envinstdist}{1.5\envinstdist}
%                \setlength{\instdist}{6.25\instdist}
%                \setlength{\labeldist}{1.5\labeldist}
%
%                \declinst{cli}{C}{}
%                \declinst{sess}{S}{}
%
%                \mess{\sm{\csreq, pk(C), $N_{C}$}{$K_{CS}$}{sk(C)}}{cli}{sess}
%                \nextlevel[1]
%                
%                \action*{Validates pk(C)}{sess}
%                \nextlevel[2.5]
%
%                \mess{\sm{\csres, {\bf C}, pk(S) $N_{S}$}{$K_{CS}$}{sk(S)}}{sess}{cli}
%                \nextlevel[1]
%                
%                \action*{Validates pk(S)}{cli}
%                \nextlevel[3]
%                
%                \mess{\sm{\asreq, {\bf S}, pk(C), Login, Passwd}{$K_{CS}$}{sk(C)}}{cli}{sess}
%                \nextlevel[1]
%                
%                \action*{Validates pk(C)}{sess}
%                \nextlevel[2]
%
%                \action*{Validates (Login, Passwd) }{sess}
%                \nextlevel[2.5]
%
%                \mess{\sm{\asres, $N_{S2}$}{$K_{CS}$}{sk(S)}}{sess}{cli}
%                \nextlevel[2]
%            \end{msc}
%        \end{postscript}
%    }
%    \caption{OPC-UA fixed CreateSession sub-protocol}
%    \label{fig:session_fix}
%\end{figure}

\begin{table}[htb]
    \centering
    %\resizebox{1.05\columnwidth}{!}{
    \begin{tabular}{|c|c|c|c|}
        \hline
        \multirow{2}{*}{\opcua Security mode} & \multicolumn{3}{|c|}{Objectives} \\
        \cline{2-4}
                       & Sec $Creds$   & Auth $N_S$    & Auth $Creds$   \\
        \hline                                                                          
        \smn           & \UNSAFE       & \UNSAFE       & \UNSAFE        \\ 
        \hline                                         
        \sms           & \UNSAFE       & \SAFE         & \SAFE          \\ 
        \hline                                         
        \smseshort     & \SAFE         & \SAFE         & \SAFE          \\ 
        \hline
    \end{tabular}
    %}
    \caption{Results for fixed \opcua {\em CreateSession} sub-protocol}
    \label{tab:session_fix_results}
\end{table}

%In a second time, we suppose that C would use different credentials for each
%server.
%Then both normal and fixed {\em CreateSession} results in Table
%\ref{tab:session_uniq_creds_results}.
%
%\begin{table}[htb]
%    \centering
%    %\resizebox{1.05\columnwidth}{!}{
%    \begin{tabular}{|c|c|c|c|}
%        \hline
%        \multirow{2}{*}{\opcua Security mode} & \multicolumn{3}{|c|}{Objectives} \\
%        \cline{2-4}
%                       & Sec $Creds$   & Auth $N_S$    & Auth $Creds$   \\
%        \hline                                                                          
%        \smn           & \UNSAFE       & \UNSAFE       & \UNSAFE       \\ 
%        \hline                                         
%        \sms           & \UNSAFE       & \UNSAFE/\SAFE & \SAFE         \\ 
%        \hline                                         
%        \smseshort     & \SAFE         & \SAFE         & \SAFE         \\ 
%        \hline
%    \end{tabular}
%    %}
%    \caption{Results for \opcua {\em CreateSession} sub-protocol with uniq credentials}
%    \label{tab:session_uniq_creds_results}
%\end{table}
%
%\TODO dire que creds uniques pour chaque serveur se comportent comme un secret partagé permettant une auth.
%
%\begin{figure}[htb]
%    \renewcommand{\smname}{\smn}
%    \resizebox{\columnwidth}{!}{
%        \begin{postscript}
%            \begin{msc}{Session creation}
%                \setlength{\envinstdist}{1.5\envinstdist}
%                \setlength{\instdist}{3\instdist}
%                \setlength{\labeldist}{1.5\labeldist}
%
%                \declinst{cli}{C}{(C,S)}
%                \declinst{int}{I}{}
%                \declinst{sess}{S}{(C,S)}
%
%                \mess{\sm{\csreq, pk(C), $N_{C}$}{$K_{CS}$}{sk(C)}}{cli}{int}
%                \nextlevel[1.5]
%                
%                %\action*{Validates pk(C)}{sess}
%                %\nextlevel[2.5]
%
%                \mess{\sm{\csres, pk(S), $N_{C}$, $N_{I}$}{$K_{CS}$}{sk(S)}}{int}{cli}
%                \nextlevel[1.5]
%                
%                %\action*{Validates pk(S)}{cli}
%                %\nextlevel[3]
%                
%                \msccomment{begin(C,S,Passwd(C,S))}{cli}
%                \mess{\sm{\asreq, pk(C), Login(C,S), Passwd(C,S)}{$K_{CS}$}{sk(C)}}{cli}{int}
%                \nextlevel[1.5]
%
%                \mess{\sm{\csreq, pk(C), $N_{C}$}{$K_{CS}$}{sk(C)}}{int}{sess}
%                \nextlevel[1.5]
%
%                \mess{\sm{\csres, pk(S), $N_{C}$, $N_{S}$}{$K_{CS}$}{sk(S)}}{sess}{int}
%                \nextlevel[1.5]
%                
%                \msccomment[r]{end(C,S,Passwd(C,S))}{sess}
%                \mess{\sm{\asreq, pk(C), Login(C,S), Passwd(C,S)}{$K_{CS}$}{sk(C)}}{int}{sess}
%                \nextlevel[1.5]
%                
%                %\action*{Validates pk(C)}{sess}
%                %\nextlevel[2]
%
%                \action*{Validates (Login, Passwd) }{sess}
%                \nextlevel[2.5]
%
%                \mess{\sm{\asres, $N_{S2}$}{$K_{CS}$}{sk(S)}}{sess}{int}
%                \nextlevel[.5]
%            \end{msc}
%        \end{postscript}
%    }
%    \caption{OPC-UA CreateSession sub-protocol}
%    \label{fig:session_se}
%\end{figure}


%\section{Flow integrity}
%%TODO \input{flow}
%Idée: Modéliser l'envoie et la réception de messages jusqu'à fermeture du cannal.
%A la fin envoyer les messages envoyés et reçus à un oracle qui les compare.
%
%\section{TODO list}
%
%\begin{itemize}
%    \item Comment s'articulent la signature et le chiffrement (signature d'un haché, d'un chiffré, ...) ?
%    \item Logjam $\Rightarrow$ Oui, vulnérable si hypothèse de DH OK
%    \item Quel type de chiffrement (CBC, ...) ?
%    \item Est-ce que tout est réellement signé (constantes publiques) ?
%\end{itemize}

\bibliographystyle{unsrt}
\bibliography{phdBiblio}

\end{document}

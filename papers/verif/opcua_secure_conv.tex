%\newcommand{\gereq}{GetEndpointRequest}
%\newcommand{\geres}{GetEndpointResponse}
%\newcommand{\oscreq}{OpenSecureChannelRequest}
%\newcommand{\oscres}{OpenSecureChannelResponse}
\newcommand{\gereq}{GEReq}
\newcommand{\geres}{GERes}
\newcommand{\oscreq}{OSCReq}
\newcommand{\oscres}{OSCRes}


The {\em OpenSecureConversation} sub-protocol described in
Figure~\ref{fig:secure_conv_se} aims to authenticate a client and
a server and allows them to generate a shared key for the later communications.
\opcua can be used with three security modes, namely {\em \smn}, {\em \sms} and
{\em \smse}.

\begin{itemize}
    \item \smse: claims to provide secrecy using symetric and
      assymetric encryption and both authentication and integrity
      through digital signatures.
    \item \sms: same as {\em \smse} but without any encryption.  Thus
      nonces are not used to generate a shared key but bring freshness
      to the messages.
    \item \smn: does not provide any security.  Using this mode, the
      {\em OpenSecureConversation} sub-protocol does not serve much
      purpose but is used for compatibility.
\end{itemize}

\begin{figure}[htb]
    \centering
    \renewcommand{\smname}{\smseshort}
    \resizebox{.8\linewidth}{!}{
        \begin{postscript}
            \begin{msc}{Secure channel creation}
                \setlength{\envinstdist}{1.5\envinstdist}
                \setlength{\instdist}{2.6\instdist}
                \setlength{\labeldist}{1.5\labeldist}

                \declinst{cli}{C}{}
                \declinst{intruder}{DiscoreryEndpoint}{}
                \declinst{sess}{S}{}

                \msccomment[-.5]{1.}{cli}
                \mess{\gereq}{cli}{intruder}
                \nextlevel[1.5]

                \msccomment[-.5]{2.}{cli}
                \mess{\geres, pk(S), \smname, SP, UP}{intruder}{cli}
                \nextlevel[1]
                
                %\action*{Validates pk(S)}{cli}
                %\nextlevel[2]
                
                \action*{Generates N$_{C}$}{cli}
                \nextlevel[3]

                \msccomment[-.5]{3.}{cli}
                \mess{pk(C), \sm{\oscreq, pk(C), $g^{N_{C}}$}{pk(S)}{sk(C)}}{cli}{sess}
                \nextlevel[1]

                %\action*{Validates pk(C) }{sess}
                %\nextlevel[2]

                \action*{Generates N$_{S}$ }{sess}
                \nextlevel[2]

                \ifsmnotnone{%
                    \action*{$K_{srv}$ = $(g^{N_{C}})^{N_{S}}$ }{sess}
                    \nextlevel[3.5]
                }

                \msccomment[-.5]{4.}{cli}
                \mess{\sm{\oscres, $g^{N_{S}}$, ST, TTL}{pk(C)}{sk(S)}}{sess}{cli}
                \nextlevel[1]

                \ifsmnotnone{%
                    \action*{$K_{cli}$ = $(g^{N_{S}})^{N_{C}}$}{cli}
                    \nextlevel[2]
                }
            \end{msc}
        \end{postscript}
    }
    \caption{\opcua OpenSecureConversation sub-protocol}
    \label{fig:secure_conv_se}
\end{figure}

\subsection{Modeling}

Normaly, a {\em GetEnpointRequest} would be answered by a list of
session endpoints with possibly different security modes.  For
simplicity, we suppose that only one endpoint is answered and that the
client will always accept the security mode proposed.  Moreover,
thanks to the perfect encryption hypothesis, we can abstract the
cryptographic suite negociation.  We consider an intruder whose public
key would be accepted by a legitimate client or server.  Such an
intruder could for instance represent a legitimate device that has
been corrupted through a virus or that is controlled by a malicious
opperator.  Finally, it is said in \cite{MLD09} that <<The
establishement of the Secure Channel is mainly used for exchanging
special secret information between clients and servers. This secret is
used for deriving Symetric Keys $[...]$>>.  To the best of our
knowledge, no more precision is given in
\cite{MLD09,opcua_part2,opcua_part4,opcua_part6} on which
cryptographic primitives should be used for such purpose.  Thus we
chose to model this key derivation using the \DiH mecanisms which
relies on the commutativity of the exponentiation: $(g^a)^b =
(g^b)^a$.  In this context, we consider the following four security
objectives:
\begin{enumerate}
    \item\label{item:sc_sec_cli} Secrecy of $K_{cli}$ obtained by C.
    \item\label{item:sc_sec_srv} Secrecy of $K_{src}$ obtained by S.
    \item\label{item:sc_auth_cli} Authentication of C on $g^{Nc}$.
    \item\label{item:sc_auth_srv} Authentication of S on $g^{Ns}$.
\end{enumerate}

\subsection{Results}\label{sec:secure_conv_res}

We run \proverif on this protocol for each combination of the three security
modes of \opcua for each objective proposed.
Results are shown in Table \ref{tab:secure_conv_results}.

\begin{table}[htb]
    \centering
    %\hspace{-1em}
    %\resizebox{1.05\columnwidth}{!}{
    \begin{tabular}{|c|c|c|c|c|}
        \hline
        \multirow{2}{*}{\opcua Security mode} & \multicolumn{4}{|c|}{Objectives} \\
        \cline{2-5}
                       & Sec $K_{cli}$ & Sec $K_{srv}$ & Auth $g^{N_{S}}$  & Auth $g^{N_{C}}$  \\
        \hline
        \smn           & \UNSAFE       & \UNSAFE       & \UNSAFE           & \UNSAFE           \\ 
        \hline
        \sms           & \SAFE         & \SAFE         & \UNSAFE           & \UNSAFE           \\ 
        \hline
        \smseshort     & \SAFE         & \SAFE         & \UNSAFE           & \UNSAFE           \\ 
        \hline
    \end{tabular}
    %}
    \caption{Results for textbook {\em OpenSecureChannel} sub-protocol}
    \label{tab:secure_conv_results}
\end{table}

Obviously, as the security mode \smn~does not provide any security, all
objectives can be attacked.
Moreover, attacks on authentication in the case of \sms~and \smse~implies the
intruder rerouting messages.
Such manipulation differs from replaying a message as it does not delays the
message more than what a legitimaete router would (thus avoiding replay
protections such as timestamps).
Figure \ref{fig:secure_conv_atk} shows an attack on the authentication of C
using $g^{N_{C}}$.
This attack is possible because the standard \opcua protocol does not requires
explicitly specifying the identity of the receiver of a message.
Thus allowing the intruder to send to S the signed message C sent to him
similarily as the attack on the Needham-Schroeder protocol \cite{Low96}.

\begin{figure}[htb]
    \hspace{-2.25em}
    \renewcommand{\smname}{\smse}
    \resizebox{1.1\columnwidth}{!}{
        \begin{postscript}
            \begin{msc}{Secure channel creation}
                \setlength{\envinstdist}{1.5\envinstdist}
                \setlength{\instdist}{5.1\instdist}
                \setlength{\labeldist}{1.5\labeldist}

                \declinst{cli}{C}{}
                \declinst{intruder}{I}{}
                \declinst{sess}{S}{}

                \mess{\gereq}{cli}{intruder}
                \nextlevel[1.5]

                \mess{\geres, pk(I), \smname, SP, UP}{intruder}{cli}
                \nextlevel[1]
                
                %\action*{Validates pk(I)}{cli}
                %\nextlevel[2]
                
                \action*{Generates N$_{C}$}{cli}
                \nextlevel[3]

                \mess{pk(C), \sm{\oscreq, pk(C), $g^{N_{C}}$}{pk(I)}{sk(C)}}{cli}{intruder}
                \nextlevel[2]

                \mess{pk(C), \sm{\oscreq, pk(C), $g^{N_{C}}$}{pk(S)}{sk(C)}}{intruder}{sess}
                \nextlevel[1]

                %\action*{Validates pk(C) }{sess}
                %\nextlevel[2]

                %\action*{Generates N$_{S}$ }{sess}
                %\nextlevel[2]

                %\ifsmnotnone{%
                %    \action*{$K_{srv}$ = $(g^{N_{C}})^{N_{S}}$ }{sess}
                %    \nextlevel[3]
                %}

                %\mess{\sm{\oscres, $g^{N_{S}}$, ST, TTL}{pk(C)}{sk(S)}}{sess}{intruder}
                %\nextlevel[1.5]
            \end{msc}
        \end{postscript}
    }
    \caption{Attack on $g^{N_{C}}$: I impersonnates C when speaking to S}
    \label{fig:secure_conv_atk}
\end{figure}

\subsection{Fixed version}

We propose a fixed version of the {\em OpenSecureConversation} sub-protocol.
%in Figure \ref{fig:secure_conv_fix_se}.
It consists in adding the identity of the reciever to the messages and thus 
avoids an intruder rerouting signed messages to impersonnate hosts as presented
in Section \ref{sec:secure_conv_res}.
More formaly, message 3. and 4. of Figure~\ref{fig:secure_conv_se} are replaced
by:

\arraycolsep=1.4pt
\begin{flushleft}
    \[\begin{array}{lrllll}
        3. & C & \rightarrow & S & : & \text{\small\sm{\oscreq, {\bf S}, pk(C), $g^{N_{C}}$}{pk(S)}{sk(C)}}\\
        4. & S & \rightarrow & C & : & \text{\small\sm{\oscres, {\bf C}, $g^{N_{S}}$, ST, TTL}{pk(C)}{sk(S)}}\\
    \end{array}\]
\end{flushleft}

%\begin{figure}[htb]
%    \hspace{-1em}
%    \renewcommand{\smname}{\smseshort}
%    \resizebox{1.05\columnwidth}{!}{
%        \begin{postscript}
%            \begin{msc}{Secure channel creation}
%                \setlength{\envinstdist}{1.5\envinstdist}
%                \setlength{\instdist}{2.8\instdist}
%                \setlength{\labeldist}{1.5\labeldist}
%
%                \declinst{cli}{C}{}
%                \declinst{intruder}{DiscoreryEndpoint}{}
%                \declinst{sess}{S}{}
%
%                \mess{\gereq}{cli}{intruder}
%                \nextlevel[1.5]
%
%                \mess{\geres, pk(S), \smname, SP, UP}{intruder}{cli}
%                \nextlevel[1]
%                
%                \action*{Validates pk(S)}{cli}
%                \nextlevel[2]
%                
%                \action*{Generates N$_{C}$}{cli}
%                \nextlevel[3]
%
%                \mess{pk(C), \sm{\oscreq, {\bf S}, pk(C), $g^{N_{C}}$}{pk(S)}{sk(C)}}{cli}{sess}
%                \nextlevel[1]
%
%                \action*{Validates pk(C) }{sess}
%                \nextlevel[2]
%
%                \action*{Generates N$_{S}$ }{sess}
%                \nextlevel[2]
%
%                \ifsmnotnone{%
%                    \action*{$K_{srv}$ = $(g^{N_{C}})^{N_{S}}$ }{sess}
%                    \nextlevel[3.5]
%                }
%
%                \mess{\sm{\oscres, {\bf C}, $g^{N_{S}}$, ST, TTL}{pk(C)}{sk(S)}}{sess}{cli}
%                \nextlevel[1]
%
%                \ifsmnotnone{%
%                    \action*{$K_{cli}$ = $(g^{N_{S}})^{N_{C}}$}{cli}
%                    \nextlevel[2]
%                }
%            \end{msc}
%        \end{postscript}
%    }
%    \caption{\opcua OpenSecureConversation sub-protocol}
%    \label{fig:secure_conv_fix_se}
%\end{figure}

We tested this fixed version under the same circumstances than the standard
version.
Results are presented in Table \ref{tab:secure_conv_fix_results}.
We can see that the attacks found on authentication are now fixed.

\begin{table}[htb]
    \centering
    %\hspace{-1em}
    %\resizebox{1.05\columnwidth}{!}{
    \begin{tabular}{|c|c|c|c|c|}
        \hline
        \multirow{2}{*}{\opcua Security mode} & \multicolumn{4}{|c|}{Objectives} \\
        \cline{2-5}
                       & Sec $K_{cli}$ & Sec $K_{srv}$ & Auth $g^{N_{S}}$  & Auth $g^{N_{C}}$  \\
        \hline
        \smn           & \UNSAFE       & \UNSAFE       & \UNSAFE           & \UNSAFE           \\ 
        \hline
        \sms           & \SAFE         & \SAFE         & \SAFE             & \SAFE             \\ 
        \hline
        \smseshort     & \SAFE         & \SAFE         & \SAFE             & \SAFE             \\ 
        \hline
    \end{tabular}
    %}
    \caption{Results for fixed {\em OpenSecureChannel} sub-protocol}
    \label{tab:secure_conv_fix_results}
\end{table}

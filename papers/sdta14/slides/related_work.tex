\begin{frame} \frametitle{Related works (smart card code robustness evaluation)} 
\vfill
\begin{block}{Test-based approach}
\begin{itemize}
\item embedded low level simulation (NOP fault model) [BBC+14] % Essos'14 Morpho
\item JavaCard byte-code fault injection [SLIC11]  %(Lanet and all 2011) (XLIM)
\item NOP permanent fault attacks at the C level [KT12] % as a coverage criterion for low level fault
%injections (Berthom\'e, Heydemann, Kauffman-Tourkestansky 2012) (Oberthur)
\end{itemize}
$\rightarrow$ target single fault, with no robustness guarantee \dots
\end{block}
%Lazart approach cons:  more complete fault models (particularly for low level code)
%Lazart approach pro: a static approach potentially ensuring proof of absence of attacks
\vfill
\begin{block}{Verification-based approach
[CCGV13] (Gemalto/Trusted Lab)}  
\begin{itemize}
\item C mutants encoding a possible simple fault injection on data 
\item robustness proof %(oracle: if fault!=0 then final-state==error)
\item implemented as a Frama-C plug-in (interacting with Jessie)
\end{itemize}
$\rightarrow$ targets non-volatile fault, limitations of proof-based approach

%Lazart approach pros:
%\begin{itemize}
%\item very similar encoding of fault injection (but permanent one)
%\item symbolic execution  more adapted than proofs (attack/absence of attack) 
%\item concolic approach allows to treat complex computation (like cryptographic functions in the openssh case study)
%\end{itemize}
\end{block}
\end{frame}


%\documentclass[a4paper]{article}
\documentclass{llncs}

\usepackage[T1]{fontenc}
\usepackage[utf8]{inputenc}

\usepackage{multirow}

\usepackage{xstring}
\usepackage{xspace}
\usepackage{amstext}

%\usepackage{showframe}
%\usepackage{times}

\usepackage{listings}
\usepackage{xcolor}
\definecolor{Green}{rgb}{0.1,0.5,0.1}
\lstset{%configuration de listings 
	float=hbp,% 
	language={Python},%
	morekeywords={assert, then},%
	columns=flexible, % 
	keepspaces=true,  % keeps spaces in text, useful for keeping indentation of code
	escapeinside={<@}{@>},%
	tabsize=4, % 
	frame=l, % 
	frameround=tttt, % 
	extendedchars=true, % 
	showspaces=false, % 
	showstringspaces=false, % 
	%numbers=left, % 
	numbersep=5pt,
	breaklines=true, % 
	xleftmargin=15pt, %
	breakautoindent=true, % 
	captionpos=b,% 
    framesep=2pt,
	keywordstyle=\color{blue}, % 
	commentstyle=\color{Green}, % 
} 


% Tikz
\usepackage{tikz}
\usetikzlibrary{arrows,snakes,backgrounds,patterns,matrix,shapes,fit,calc,shadows,plotmarks,intersections,shapes.geometric}

% Include paths
\usepackage{graphicx}
\graphicspath{{assets/}}
\makeatletter
    \def\input@path{{assets/}}
\makeatother

\newcommand{\UNSAFE}{{\color{red!50!black} UNSAFE}\xspace}
\newcommand{\SAFE}{{\color{green!50!black} SAFE}\xspace}
\newcommand{\NA}{{\color{blue!50!black} N/A}\xspace}
\newcommand{\TODO}{{\color{red}\bf TODO}\xspace}
\newcommand{\DiH}{Diffie-Hellman\xspace}
\newcommand{\proverif}{ProVerif\xspace}
\newcommand{\ofmc}{OFMC\xspace}
\newcommand{\spear}{SPEAR II\xspace}
\newcommand{\opcua}{OPC-UA\xspace}
\newcommand{\opc}{OPC\xspace}
\newcommand{\dcom}{DCOM\xspace}
\newcommand{\mms}{MMS\xspace}
\newcommand{\iec}[1]{IEC\xspace#1\xspace}
\newcommand{\iccp}{ICCP\xspace}
\newcommand{\modbus}{MODBUS\xspace}
\newcommand{\profinet}{PROFINET\xspace}
\newcommand{\etherip}{EtherNet/IP\xspace}
\newcommand{\dnp}{DNP3\xspace}
\newcommand{\eg}{{\em e.g.:}\xspace}
\newcommand{\etal}{{\em et al.}\xspace}
\newcommand{\securechan}{{\em OpenSecureChannel}\xspace}
\newcommand{\session}{{\em CreateSession}\xspace}

\title{Short paper: Domain Specific Filtering With Worst-Case Bandwidth}
\author{Maxime Puys \and Jean-Louis Roch \and Marie-Laure Potet}
\institute{Verimag, University Grenoble Alpes,  Gi\`eres, France \\
  \texttt{firstname.lastname@imag.fr}
  \thanks{This work has been partially supported by the LabEx PERSYVAL-Lab
      (ANR-11-LABX-0025) and the project {\em Programme Investissement d’Avenir
      FSN AAP Sécurité Numérique n\textsuperscript{o}3} ARAMIS (P3342-146798).}
}
\date{}

\begin{document}

\renewcommand{\thelstlisting}{\arabic{lstlisting}}

\maketitle

\begin{abstract}
    Industrial systems are publicly the target of cyberattacks since
    Stuxnet \cite{Lan11}.  Nowadays they are increasingly communicating over
    insecure media such as Internet.  Due to their interaction with
    the real world, it is crucial to ensure their security. In this paper, we
    propose a stateful type of domain specific filtering able to keep track
    of the value of predetermined variables. Such filter allows to express rules
    depending on the context of the system. Moreover, it must guarantee bounded
    memory and execution time to be resilient against malicious adversaries.
    Our approach is illustrated on an example.
\end{abstract}

\section{Introduction}\label{sec:intro}
\begin{itemize}
    \item La plus part des filtres sont stateless.
    \item Problématique car exemple sectionneur (+exemple aurora ?)
    \item Cependant stateful implique single point of passage == single point of failure
\end{itemize}

\paragraph{Contributions:} Stateful filtering with worst-case bandwidth.

\paragraph{Outline:} 


\section{Stateful Filtering}\label{sec:stateful}
In this Section, we discuss shortly what it stateful filtering and its
shortcomings.
Stateful filtering consists in keeping track of the value of predertemined
variables of servers.
The filter saves the value of the variable when they go through.
As we said in Section~\ref{sec:intro} this implies the filter to be the single
point of passage of all messages.
This implies that the filter must be hardened to resist against attacks.
It also requires it to run in bounded memory and execution time to not delay
real time message (for instance filters that necessitate sorting) or crash the
filter when processing a memory-worst-case message.
Same applies to evaluation of arithmetic operations which uses a pushdown
automaton which consumes memory depending on the size of the operation.
Thus a long enough message could fill the memory of the filter.
%
Morover, no decision can be taken for a variable if it have not been seen once
before.
For this sake, one might want to use three values logic such as Kleene's
logic~\cite{Kle52}.
This also holds if the server can update variables on his own (such as
temparature, pression, etc) and they are not read frequently enough.
Three values logics introduce a value neither true or false, called
{\em unknown} or {\em irrelevant} and extend classic logic operators to handle
such value.
Thus a default policy is needed when the filter is not able to take a decision.

\paragraph{Attacker model:} We consider an attacker that is able to communicate
with the server through a filter using network.
The intruder has access to the rules configured for the filter.
We also consider the attacker having access to their implementations.
This can be illustrated by an attacker purchasing a copy of the filter and
reverse engineering it.
Finaly we make the hypothesis that the attacker has no physical access to the
filter and cannot for instance power it off.


\section{Proposed Implementation}\label{sec:impl}

In this Section, we present our filtering model taking the limitations presented
in Section~\ref{sec:stateful} into account.
Then we illustrate it on an example.

\subsection{Description}\label{sec:impl_desc}
In this section, we explain how the the stateful filtering mechanism we propose
works.

\paragraph{Variables:} All variables of present on a server should be known by
filter. Thus a variable represented by a numerical identifier is associated to a server
(associated to a protocol), a data type (\eg uint32 or double) and the path on 
the server to access it (\eg a \modbus address or an \opcua node).
Variables can also have a sequence of dimensions (\eg the length of an array or
the dimensions of a matrix).
Their definition is shown in Listing~\ref{lst:var}.

\begin{lstlisting}[label=lst:var,caption=Variable definition example]
# A <@{\color{Green} \modbus}@> server
Declare Server 1 Protocol Modbus Addr 10.0.0.1 Port 502
<@\vspace{-.8em}@>
# A <@{\color{Green} \modbus}@> coil
Declare Variable 1 Server 1 Type Boolean Addr coils:0x1000
<@\vspace{-.8em}@>
# An <@{\color{Green} \opcua}@> server
Declare Server 2 Protocol OpcUa Addr 10.0.0.2 Port 48010
<@\vspace{-.8em}@>
# An <@{\color{Green} \opcua}@> unsigned integer 5<@{\color{Green} $\times$}@> 10 matrix
Declare Variable 2 Server 2 Type UInt32 Addr numeric:5000 Dims 5 10
\end{lstlisting}

\paragraph{Monitors:} A monitor represents a local copy of a declared variable.
To ensure bounded memory, it only applies to one cell of the variable when
multidimensional.
We use the {\em Index} keyword followed by enough valid array keys to obtain a
scalar value.
Such constraint can be lifted if and only if the size and dimensions of a
variable cannot be modified once set.
The value of a monitor shall be updated when a message containing the value
goes through the filter.
It applies for example to read responses and write requests.
Updates on write requests must be reversible since the request can possibly be
rejected by the server.
Thus the new value shall be stored in a temporary variable until server confirms
the success of the request.
Other services can be used to update monitors depending on the protocols such
as \opcua's publish status.
An example of the definition of monitors is shown in Listing~\ref{lst:mon}.

\begin{lstlisting}[label=lst:mon,caption=Monitor definition example]
# A monitor on a <@{\color{Green} \modbus}@> coil
Declare Monitor 1 Variable 1
<@\vspace{-.8em}@>
# A monitor a cell of an <@{\color{Green} \opcua}@> unsigned integer 5<@{\color{Green} $\times$}@> 10 matrix
Declare Monitor 2 Variable 2 Index 3 4
\end{lstlisting}

\paragraph{Rules:} Finally rules can be set on variables using the previously
declared monitors.
They can target either a whole variable or a subrange when multidimensional.
They take the form of disjunctive normal forms (DNF) which are basically a sum of
product with addition and multiplication being respectively logical OR and AND.
Multiple rules can be expressed in DNF and they have to all be verified to 
authorize a message, thus the set of all rules becomes a conjunctive normal
form.
The predicates involved in the rules are boolean functions taking two arguments.
These functions implement boolean conditions such as equality, integer
relations, etc.
Arguments take the form of constant numbers.
We also introduce two keywords: (i) {\em NewVal} designating the value to be
written in a write request (possibly restricted to a single cell if the rule
targets a multidimensional variable) and (ii) {\em LocalVal} designating a
monitor by its identifier.
A rule can be either an assertion that will block a message when violated or a
warning that will authorize the message but log the violation for a later
decision.
An example of the definition of rules is shown in Listing~\ref{lst:rule}.

\begin{lstlisting}[label=lst:rule,caption=Rule definition example]
# Variable 1 should never been set to its current value
# (<@{\color{Green} \eg}@> opening a currently opened circuit breaker)
Declare Rule Variable 1 Assert NotEqual(NewVal, LocalVal[1])
<@\vspace{-.8em}@>
# The first three rows of variable 2 must remain between 0 and 100.
Declare Rule Variable 2 Range 0-5 0-2 Warning \
    GreaterThan(NewVal,0) AND LessThan(NewVal,100)
\end{lstlisting}

All of our predicate can be verified in $\mathcal{O}(1)$ complexity.
So processing one command only depends on the number of rules.


\subsection{Use-Case Example: an Electrical Disconnector}\label{sec:impl_example}
To illustrate our stateful filtering proccess, we propose the following simple
example.
An electrical disconector $D$ separates three eletrical networks such as
networks 2 and 3 are connected to the same input of $D$.
As we told in Section~\ref{sec:intro}, a disconector cannot be manipulated while
current is passing to avoid the creation of an electric arc.
To ensure safety, three circuit breakers $B_1$, $B_2$ and $B_3$ are placed
between $D$ and each eletrical network.
Figure~\ref{fig:example} describes this setup.

\begin{figure}[htb]
    \centering
    \resizebox{.5\textwidth}{!}{
        \begin{tikzpicture}[font=\Large,
    arrow/.style={thick,<->,shorten >=2pt,shorten <=2pt,>=stealth},
]
    \draw (1,1) rectangle (2,2) node [pos=.5] {$B_1$};
    \draw (3.5,1.5) circle (.5) node {$D$};

    \draw (6,0) rectangle (7,1) node [pos=.5] {$B_3$};
    \draw (6,2) rectangle (7,3) node [pos=.5] {$B_2$};
    
    \draw (0,1.5) -- (1,1.5); 
    \draw (2,1.5) -- (3,1.5); 
    \draw (4,1.5) -- (5,1.5);

    \draw (5,.5) -- (5,2.5);

    \draw (5,.5) -- (6,.5); 
    \draw (5,2.5) -- (6,2.5);

    \draw (7,.5) -- (8,.5);
    \draw (7,2.5) -- (8,2.5);
\end{tikzpicture}

    }
    \caption{Example infrastrucure}
    \label{fig:example}
\end{figure}

Within a \modbus server, $D$, $B_1$, $B_2$ and $B_3$ can be represented as coil
(i.e.: read/write booleans) with openned state represented by {\em False}.
In this example, $D$ can be manipulated if and only if either $B_1$ if openned
or if both $B_2$ and $B_3$ are open.
Thus the configuration presented in Listing~\ref{lst:example} is enough to
describe this rule.
Note that in the rule definition, the AND operator has priority on the OR
operator.

\begin{lstlisting}[label=lst:example,caption=Example configuration]
Declare Server 1 Protocol Modbus Addr 10.0.0.1 Port 502
<@\vspace{-.8em}@>
Declare Variable 1 Server 1 Type Boolean Addr coils:0x1001 # <@{\color{Green} $B_1$}@>
Declare Variable 2 Server 1 Type Boolean Addr coils:0x1002 # <@{\color{Green} $B_2$}@>
Declare Variable 3 Server 1 Type Boolean Addr coils:0x1003 # <@{\color{Green} $B_3$}@>
Declare Variable 4 Server 1 Type Boolean Addr coils:0x1004 # <@{\color{Green} $D$}@>
<@\vspace{-.8em}@>
Declare Monitor 1 Variable 1 # Monitor on <@{\color{Green} $B_1$}@>
Declare Monitor 1 Variable 2 # Monitor on <@{\color{Green} $B_2$}@>
Declare Monitor 1 Variable 3 # Monitor on <@{\color{Green} $B_3$}@>
<@\vspace{-.8em}@>
Declare Rule Variable 4 Assert    \
    Equal(LocalVal[1], False) OR \
    Equal(LocalVal[2], False) AND Equal(LocalVal[3], False)
\end{lstlisting}


\section{Conclusion}\label{sec:concl}
In this paper we present a stateful type of domain specific filtering able to
keep track of the value of predertemined variables.
It guarantees bounded memory and execution time to be resilient against
malicious adversaries since processing one command only depends on the number of
rules and memory to store monitor is controled by only monitoring scalar
variables (or cells when multidimentional).
\TODO Future works


\bibliographystyle{unsrt}
\bibliography{phdBiblio}

\end{document}

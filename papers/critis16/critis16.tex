%\documentclass[a4paper]{article}
\documentclass{llncs}

\usepackage[T1]{fontenc}
\usepackage[utf8]{inputenc}

\usepackage{multirow}

\usepackage{graphicx}
\usepackage{xstring}
\usepackage{xspace}
\usepackage{amstext}

%\usepackage{showframe}
\usepackage{times}

\newcommand{\UNSAFE}{{\color{red!50!black} UNSAFE}\xspace}
\newcommand{\SAFE}{{\color{green!50!black} SAFE}\xspace}
\newcommand{\NA}{{\color{blue!50!black} N/A}\xspace}
\newcommand{\TODO}{{\color{red}\bf TODO}\xspace}
\newcommand{\DiH}{Diffie-Hellman\xspace}
\newcommand{\proverif}{ProVerif\xspace}
\newcommand{\ofmc}{OFMC\xspace}
\newcommand{\spear}{SPEAR II\xspace}
\newcommand{\opcua}{OPC-UA\xspace}
\newcommand{\opc}{OPC\xspace}
\newcommand{\dcom}{DCOM\xspace}
\newcommand{\mms}{MMS\xspace}
\newcommand{\iec}[1]{IEC\xspace#1\xspace}
\newcommand{\iccp}{ICCP\xspace}
\newcommand{\modbus}{MODBUS\xspace}
\newcommand{\profinet}{PROFINET\xspace}
\newcommand{\etherip}{EtherNet/IP\xspace}
\newcommand{\dnp}{DNP3\xspace}
\newcommand{\eg}{{\em e.g.}\xspace}
\newcommand{\etal}{{\em et al.}\xspace}
\newcommand{\securechan}{{\em OpenSecureChannel}\xspace}
\newcommand{\session}{{\em CreateSession}\xspace}

\title{Short paper: Domain Specific Filtering With Worst-Case Bandwidth}
\author{Maxime Puys \and Jean-Louis Roch \and Marie-Laure Potet}
\institute{Verimag, University Grenoble Alpes,  Gi\`eres, France \\
  \texttt{firstname.lastname@imag.fr}
  \thanks{This work has been partially supported by the LabEx PERSYVAL-Lab
      (ANR-11-LABX-0025) and the project {\em Programme Investissement d’Avenir
      FSN AAP Sécurité Numérique n\textsuperscript{o}3} ARAMIS (P3342-146798).}
}
\date{}

\begin{document}

\maketitle

\begin{abstract}
    Industrial systems are publicly the target of cyberattacks since
    Stuxnet \cite{Lan11}.  Nowadays they are increasingly communicating over
    insecure media such as Internet.  Due to their interaction with
    the real world, it is crucial to ensure their security. In this paper, we
    propose a stateful type of domain specific filtering able to keep track
    of the value of predertemined variables. Such filter allows to express rules
    depending on the context of the system. Moreover, it must guarantee bounded
    memory and execution time to be resilient against malicious adversaries.
    Our approach is illusrated on an example.
\end{abstract}

\section{Introduction}\label{sec:intro}
\begin{itemize}
    \item La plus part des filtres sont stateless.
    \item Problématique car exemple sectionneur (+exemple aurora ?)
    \item Cependant stateful implique single point of passage == single point of failure
\end{itemize}

\paragraph{Contributions:} Stateful filtering with worst-case bandwidth.

\paragraph{Outline:} 


\section{Stateful Filtering}\label{sec:stateful}
\subsection{Description}\label{sec:stateful_desc}
Stateful = garder en mémoire la valeur de certaines variables.
Limites:
\begin{itemize}
    \item Mémoire et temps d'exécusion borné (cas des automates à piles ou du tri).
    \item Décision locale implique logique ternaire.
\end{itemize}

Modèles de l'adversaire.


\subsection{Proposed Implementation}\label{sec:stateful_impl}
\input{stateful_impl}

\section{Use-Case Example: an Electrical Disconector}\label{sec:example}
\begin{tikzpicture}[font=\Large,
    arrow/.style={thick,<->,shorten >=2pt,shorten <=2pt,>=stealth},
]
    \draw (1,1) rectangle (2,2) node [pos=.5] {$B_1$};
    \draw (3.5,1.5) circle (.5) node {$D$};

    \draw (6,0) rectangle (7,1) node [pos=.5] {$B_3$};
    \draw (6,2) rectangle (7,3) node [pos=.5] {$B_2$};
    
    \draw (0,1.5) -- (1,1.5); 
    \draw (2,1.5) -- (3,1.5); 
    \draw (4,1.5) -- (5,1.5);

    \draw (5,.5) -- (5,2.5);

    \draw (5,.5) -- (6,.5); 
    \draw (5,2.5) -- (6,2.5);

    \draw (7,.5) -- (8,.5);
    \draw (7,2.5) -- (8,2.5);
\end{tikzpicture}


\section{Conclusion}\label{sec:concl}
In this paper we present a stateful type of domain specific filtering able to
keep track of the value of predertemined variables.
It guarantees bounded memory and execution time to be resilient against
malicious adversaries since processing one command only depends on the number of
rules and memory to store monitor is controled by only monitoring scalar
variables (or cells when multidimentional).
\TODO Future works


\bibliographystyle{unsrt}
\bibliography{phdBiblio}

\end{document}

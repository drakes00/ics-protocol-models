Industrial systems also called SCADA (\emph{Supervisory Control And
Data Acquisition}) are the target of cyberattacks since the Stuxnet
worm~\cite{Lan11} in 2010.
Nowadays, these systems control nuclear power plants, water purification or
power distribution.
Due to the criticality of their interaction with the real world, they can
potentially be really harmful for humans and environment.
The frequency of attacks againt these systems is increasing to become one of the
priorities for governmental agencies, \eg~\cite{SFS11} from the US
National Institute of Standards and Technology (NIST) or
\cite{ANSSI12_guide_securite_industrielle_en} from the French {\em Agence
Nationale de la Sécurité des Systèmes d'Information} (ANSSI).

To face such adversaries, industrial systems can be protected by intrusion
detection systems~\cite{StoneGate13,Pax99,Suricata,Snort} or intrusion
protection systems~\cite{DESIRE10,StoneGate11,rWeb13,DZNetwork14}.
Those kind of filters are usualy {\em stateless}, meaning that judge the 
legitimacy of a message based only on the message itself.
However, a lot of attacks will occur because a sequence of messages is recieved
in a certain order or in a certain amount of time, each message being legitimate
on its own.
Such attack has been demonstrated through the Aurora project~\cite{Aurora}, lead
by the US National Idaho Laboratory in 2007 (and classified until 2014).
In order to test a diesel generator against cyberattacks, researchers rapidly 
sent openning and closing commands to circuit breakers.
The frequency of orders being to high, it caused the generator to explode.
Electrical disconectors also require to be manipulated by commands in a precise
order.
If any electric current runs through a disconector while it is manipulated, an
eletric arc will appear harming humans around and damaging equipment.

\paragraph{Contributions:} We propose a stateful type of domain specific
filtering able to keep track of the value of predertemined variables.
While filtering messages it will save the value of some variable when they go
through.
Such filtering process is tedious since the filter must be the single point of
passage of all commands to not miss any.
One could think of several filter at different positions in the architecture,
synchronizing all together.
However, this easily becomes an instance of the distributed consensus problem
over an unreliable channel which has been proven impossible in the general
case~\cite{FLP85}.
Thus, having a single point of passage for commands also means a single point of
failure.
To be resilient against malicious adversaries, we designed our filtering process
to guarentee worst-case bandwidth and memory.


\paragraph{Outline:} First, Section~\ref{sec:stateful} explains more deeply
stateful filtering and its pros and cons.
Then Section~\ref{sec:impl} describes our filtering model and illustrates it on
an example.
Finaly, Section~\ref{sec:concl} concludes.

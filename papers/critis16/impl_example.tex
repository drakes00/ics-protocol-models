To illustrate our stateful filtering proccess, we propose the following simple
example.
An electrical disconector $D$ separates three eletrical networks such as
networks 2 and 3 are connected to the same input of $D$.
As we told in Section~\ref{sec:intro}, a disconector cannot be manipulated while
current is passing to avoid the creation of an electric arc.
To ensure safety, three circuit breakers $B_1$, $B_2$ and $B_3$ are placed
between $D$ and each eletrical network.
Figure~\ref{fig:example} describes this setup.

\begin{figure}[htb]
    \centering
    \resizebox{.5\textwidth}{!}{
        \begin{tikzpicture}[font=\Large,
    arrow/.style={thick,<->,shorten >=2pt,shorten <=2pt,>=stealth},
]
    \draw (1,1) rectangle (2,2) node [pos=.5] {$B_1$};
    \draw (3.5,1.5) circle (.5) node {$D$};

    \draw (6,0) rectangle (7,1) node [pos=.5] {$B_3$};
    \draw (6,2) rectangle (7,3) node [pos=.5] {$B_2$};
    
    \draw (0,1.5) -- (1,1.5); 
    \draw (2,1.5) -- (3,1.5); 
    \draw (4,1.5) -- (5,1.5);

    \draw (5,.5) -- (5,2.5);

    \draw (5,.5) -- (6,.5); 
    \draw (5,2.5) -- (6,2.5);

    \draw (7,.5) -- (8,.5);
    \draw (7,2.5) -- (8,2.5);
\end{tikzpicture}

    }
    \caption{Example infrastrucure}
    \label{fig:example}
\end{figure}

Within a \modbus server, $D$, $B_1$, $B_2$ and $B_3$ can be represented as coil
(i.e.: read/write booleans) with openned state represented by {\em False}.
In this example, $D$ can be manipulated if and only if either $B_1$ if openned
or if both $B_2$ and $B_3$ are open.
Thus the configuration presented in Listing~\ref{lst:example} is enough to
describe this rule.
Note that in the rule definition, the AND operator has priority on the OR
operator.

\begin{lstlisting}[label=lst:example,caption=Example configuration]
Declare Server 1 Protocol Modbus Addr 10.0.0.1 Port 502
<@\vspace{-.8em}@>
Declare Variable 1 Server 1 Type Boolean Addr coils:0x1001 # <@{\color{Green} $B_1$}@>
Declare Variable 2 Server 1 Type Boolean Addr coils:0x1002 # <@{\color{Green} $B_2$}@>
Declare Variable 3 Server 1 Type Boolean Addr coils:0x1003 # <@{\color{Green} $B_3$}@>
Declare Variable 4 Server 1 Type Boolean Addr coils:0x1004 # <@{\color{Green} $D$}@>
<@\vspace{-.8em}@>
Declare Monitor 1 Variable 1 # Monitor on <@{\color{Green} $B_1$}@>
Declare Monitor 1 Variable 2 # Monitor on <@{\color{Green} $B_2$}@>
Declare Monitor 1 Variable 3 # Monitor on <@{\color{Green} $B_3$}@>
<@\vspace{-.8em}@>
Declare Rule Variable 4 Assert    \
    Equal(LocalVal[1], False) OR \
    Equal(LocalVal[2], False) AND Equal(LocalVal[3], False)
\end{lstlisting}

Industrial systems also called SCADA (\emph{Supervisory Control And
Data Acquisition}) have been known to be targeted by cyberattacks
since the famous Stuxnet case~\cite{Lan11} in 2010.  Due to the
criticality of their interaction with the real world, these
systems can potentially be really harmful for humans and environment.
The frequency of such attacks is increasing to become one of the
priorities for governmental agencies, \eg~\cite{SFS11} from the US
National Institute of Standards and Technology (NIST) or
\cite{ANSSI12_guide_securite_industrielle_en} from the French {\em Agence
Nationale de la Sécurité des Systèmes d'Information} (ANSSI).
%As
%industrial systems historically have been physically isolated from the
%rest of the world, there were less exposed to cyber attacks and their
%security was less considered. Nowadays, such attacks become feasible
%because these systems are spreading geographically and communicating
%more and more through unsafe mediums like Internet.


%Industrial systems have specificities. First the isolation of such
%systems requires the attacker to be physically present in the
%system. Then compagnies focused more on the protection against natural
%deceases and human mistakes than cybersecurity (often call security).
%In the context of security there is an adversary willing to perform malicious
%actions.
Industrial systems differ from other systems because of
the  long lifetime of the devices and their difficulty to
be patched in case of vulnerabilities.
Such specificities encourage to carefully check
standards and applications before deploying them.
%It also explain the number of legacy hosts.
%Moreover, most of industrial protocols are proprietary and
%provide a very low level of security if
%any, \eg \modbus~\cite{MODBUS}, \profinet~\cite{PROFINET}, \etherip~\cite{Bro01}
%or
%\dnp~\cite{CR04}.
%However, in 2006, the {\em OPC Foundation} (an industry consortium)
%released the first version of \opcua~\cite{MLD09}, which is presented
%as the new standard for industrial communications. %% This standard  and whose security
%% is quite closer to business IT's protocols such as TLS~\cite{DR08}.
%% This directly applies to communication protocols.
As it already appeared for business IT's protocols for twenty years,
automated verification is crucial in order to discover flaws in the
protocols' specifications before assessing implementations. However,
the lack of formal verification of industrial protocols has been
emphasized in 2006 by Igure \etal~\cite{ILW06} and in 2009 by
Patel \emph{\etal}~\cite{PBG09}.  They particularly argued that
automated protocol verification help to understand most of the
vulnerabilities of a protocol before changing its standards in order
to minimize the number of revisions which costs time and money.
%Moreover, due to the combinational explosion of the number of
%possible execution traces and the increasing complexity of the
%systems, only automated verification can ensure the security of a
%protocol in a given model.

%\subsection{State-of-the-art}\label{sec:intro_sota}
\paragraph{State-of-the-art:}\label{sec:intro_sota}

%The security of industrial protocols becomes a hot topic and some works are
%present in the literature.
Most of the works on the security of industrial protocols only rely on
specifications written in human language rather than using formal
methods.  In 2004, Clarke \etal~\cite{CR04} discussed the security
of \dnp (\emph{Distributed Network Protocol}) and \iccp
(\emph{Inter-Control Center Communications Protocol}).  In 2005,
Dzung \etal~\cite{DNHC05} proposed a detailed survey on the security
in SCADA systems including informal analysis on the security
properties offered by various industrial protocols: \opc (\emph{Open
Platform Communications}),
\mms (\emph{Manufacturing Message Specification}),
\iec{61850}, \iccp and \etherip.
In 2006, in the technical documentation of \opcua (\emph{OPC Unified
Architecture}) the authors detailed the security measures of the
protocol (specially in part 2, 4 and 6).  In 2015, Wanying \etal
summarized the security offered by \modbus, \dnp and
\opcua.

On the other hand, some works propose new versions of existing protocols to make
them secure against malicious adversaries.
In 2007, Patel \etal~\cite{PY07} studied the security of \dnp and proposed two
ways of enhancing it through digital signatures and challenge-response models.
In 2009, Fovino \etal~\cite{FCMT09} proposed a secure version of \modbus
relying on well-known cryptographic primitives such as RSA or SHA2.
%This version of the protocol also need to introduce new components in the system
%to allow existing devices to use these cryptographic primitives.
In 2013, Hayes \etal~\cite{HE13} designed another secure \modbus
protocol using hash-based message authentication codes and built on
STCP (\emph{Stream Transmission Control Protocol}).
%
To the best of our knowledge, Graham \etal~\cite{GP05} is the only
work directly using formal methods to prove the security of industrial
protocols or find attack against them.  They proposed a formal
verification of \dnp using \ofmc~\cite{BMV03} (Open-Source Fixed-Point
Model-Checker) and
\spear~\cite{SH01} (Security Protocol Engineering and Analysis Resource).
%% \TODO Sadly we were not able to access their modelings to discuss them with
%% ours.
%In 2008, Dutertre~\cite{Dut08} detailed formal specifications of \modbus
%developed using PVS, a generic theorem prover in order to help proving the
%consistancy of an implementation with the standards.

%\subsection{Contributions}
\paragraph{Contributions:}

We propose a formal analysis of the security of the sub-protocols
involved in the \opcua handshake, namely \opcua{}
\securechan and \opcua{} \session. 
These sub-protocols are crucial for the security since the first aims at
authenticating a client and a server and deriving secret keys while the second
allows the client to send his credentials to the server.

To perform our security analysis, we use one of the most efficient
tools in the domain of cryptographic protocol verification according
to \cite{LP15}, namely \proverif developed by
Blanchet \etal~\cite{Bla01}.  It considers the classical Dolev-Yao
intruder model~\cite{DY81} who controls the network, listens, stops,
forges, replays or modifies some messages according to its
capabilities and knowledge.  The perfect encryption hypothesis is
assumed, meaning that it is not possible to decrypt a ciphertext
without its encryption key or to forge a signature without knowing the
secret key.  \proverif{} can verify security properties of a protocol
such as secrecy and authentication.  The first property ensures that a
secret message cannot be discovered by an unauthorized agent
(including the intruder).  The authentication property means that one
participant of the protocol is guaranteed to communicate with another
one.  Modeling credential in \proverif{} is not common and requires to
understand the assumptions made in the protocol in order to model it
correctly.  We follow the official \opcua standards in our modellings
and checked it against a free implementation: {\em FreeOpcUa}
\footnote{https://freeopcua.github.io/}. Finally, using \proverif{}, we automatically find attacks against both of them and provide
simple realistic countermeasures.

\paragraph{Outline:} In Section~\ref{sec:secure_channel}, we analyze the
security of \opcua{} \securechan and \opcua{} \session in
Section~\ref{sec:session}.  Finally, we conclude in
Section~\ref{sec:conclusion}.


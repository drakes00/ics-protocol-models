\documentclass{article}

%\usepackage{fullpage}

\usepackage[T1]{fontenc}
\usepackage[utf8]{inputenc}

\usepackage[french]{babel}

\usepackage{authblk}

\usepackage{listings}
\usepackage{caption}
\usepackage{subcaption}

\usepackage{graphicx}
\usepackage{xstring}

\usepackage{auto-pst-pdf}
\ifpdf
    \usepackage{todonotes}
    \usepackage{tikz}
    \usetikzlibrary{arrows,automata}
\else
    \usepackage{msc}
\fi

% ========== Don't touch ==========
\makeatletter
    \def\msc@frame{no}
    \def\msc@settitle{}
\makeatother

\newcommand{\dosmnone}[3]{#1}
\newcommand{\dosmsign}[3]{#1, $\left\{\mbox{#1}\right\}_{\mbox{#3}}$}
\newcommand{\dosmsignandencrypt}[3]{$\left\{\mbox{#1}\right\}_{\mbox{#2}}$, $\left\{\mbox{#1}\right\}_{\mbox{#3}}$}

\newcommand{\smn}{None}
\newcommand{\sms}{Sign}
\newcommand{\smse}{SignAndEncrypt}

\newcommand{\sm}{\IfStrEq{\smname}{\smn}{\dosmnone}{\IfStrEq{\smname}{\sms}{\dosmsign}{\dosmsignandencrypt}}}

\newcommand{\ifsmnotnone}[1]{\IfStrEq{\smname}{\smn}{}{#1}}
\newcommand{\smname}{\smn}
% ========== Don't touch ==========

\newcommand{\UNSAFE}{{\color{red!50!black} UNSAFE}}
\newcommand{\SAFE}{{\color{green!50!black} SAFE}}
\newcommand{\proverif}{ProVerif}

\title{Verification formelle de protocoles industriels}

\author[1]{Maxime Puys}
\author[1]{Marie-Laure Potet}
\author[2]{Jean-Louis ?}
\author[3]{Pascal ?}

\affil[1]{
    Univ. Grenoble Alpes, VERIMAG, F-38000 Grenoble, France\\
    CNRS, VERIMAG, F-38000 Grenoble, France
}

\date{}

\begin{document}

\maketitle

%\begin{abstract}
%    aa
%\end{abstract}

\section{Introduction}

Les protocoles sont un ingrédient essentiel des communications réseaux de notre époque.
Ces communications pouvant être attaquées soit en espionnant le trafic, soit en le modifiant,
il apparait depuis plusieurs décennies que la preuve formelle de la sécurité de ces protocoles
est un enjeu majeur.
Un exemple connu étant {\em mi.TLS} \cite{BFKPS13}, une implémentation prouvée du protocole TLS \cite{DR08}
servant par exemple au paiement en ligne.
Cependant, les protocoles de communication des systèmes industriels servant à envoyer des commandes
et recevoir des informations n'ont jamais été vérifiés de cette façon.
Ces protocoles régissant parfois jusqu'à des centrales nucléaires, des barrages ou la distribution d'énergie,
il est vital de s'assurer de leur sécurité.

Les traveaux d'analyse de protocoles hors du cadre industriel ont commencé avec la contribution de
G. Lowe \cite{Low96}, brisant le protocole {\em Needham-Schoeder} en 1995.
Depuis, les outils d'analyse et les travaux de modélisation se sont multipliés.
Ces outils reposent sur des approches théoriques variées telles que la réécriture,
la résolution de contrainte, les clauses de Horn ou les {\em SAT-solvers}.
Ces outils considèrent un intrus dit de Dolev-Yao \cite{DY81} qui contrôle le réseau,
espionne, stoppe, forge, modifie, entrelace ou rejoue des messages en utilisant la connaissance
des messages qu'ils appris précédemment.
En jouant en parallèle plusieurs sessions avec les différents rôles du protocole,
l'intrus tente de violer des propriétés telles que l'{\em autentification} et le {\em secret}.
La première propriété signifie qu'un participant est conviancu qu'il parle avec un autre.
Le secret assure qu'un agent non autorisé (y compris l'intrus) n'accède pas au message désigné.
Enfin, les outils se basent sur l'hypothèse du chiffrement parfait, selon laquelle
il n'est pas possible de déchiffrer un message sans la chef de chiffrement ou d'usurper une signature.

En particulier, \proverif{} \cite{Bla01,PROVERIF14_manual}, développé par B. Blanchet et al. analyse un protocole écrit
soit en clauses de Horn soit dans un sous ensemble du Pi-Calcul pour un nombre non borné
de sessions.
Via des techniques de sur-approximation, il est capable de prouver la sécurité d'un protocole
mais les attaques qu'il trouve peuvent être des faux-positifs.
Lorsqu'une attaque est trouvée, l'outil reconstruit une trace montrant sa réalisation.

\section{Le protocol OPC-UA}

OPC-UA \cite{opc-ua} est un protocole de communication qui devient un standard dans le milieu industriel.
Ce nouveau protocole très complexe de part la diversité des données qu'il peut transférer
se divise en plusieurs couches. 
Parmi elles, la couche {\em Secure Conversation} est en charge de la sécurité en termes d'authentification,
de contrôle d'accès, d'intégrité et de chiffrement.
Trois niveaux de sécurité peuvent être utilisés suivant la configuration des clients et des serveurs.

Plusieurs sous-protocoles atomiques interviennent dans cette couche :
\begin{enumerate}
    \item Le sous-protocole {\em Secure Channel Establishment} se sert des
        clefs assymétriques des participants pour établir une clef symétrique partagée.
    \item Ensuite, le sous-protocole {\em Session Establishment} utilise la clef symétrique partagée
        par les participants pour que le client envoie ses identifiants au serveur.
    \item Le client peut alors envoyer des requêtes et le serveur les réponses associées
        en utilisant la clef symétrique partagée.
    \item Enfin, le sous-protocole {\em Secure Conversation Closing} clot la session et le {\em Secure Channel}.
\end{enumerate}

\section{Travaux de recherche}

Nous souhaitons tester formellement la sécurité d'OPC-UA à l'aide d'outils de vérification de protocoles.
Une attaque pouvant être découverte pourrait être la compromission de la clef symétrique partagée
(qui révèlerait le contenu des communication).
D'autre part, un intrus pourrait se faire passer pour le serveur aux yeux du client ou beaucoup plus grave, se faire passer
pour le client aux yeux du serveur.
Ce scénario gravissime autoriserait l'intrus à lancer des commandes ayant un réel impact sur le monde extérieur
pouvant par exemple égaler les conséquenses du virus Stuxnet en Iran \cite{Lan11}.

Afin de découvrir des attaques ou de prouver formellement la sécurité du protocole, nous cherchons tout d'abord à l'exprimer
sous forme de diagrammess de séquence.
La figure \ref{fig:opcua_diagram} montre par exemple une représentation du sous-protocole {\em Secure Channel Establishment}.

\begin{figure}[htb]
    %\newcommand{\gereq}{GetEndpointRequest}
%\newcommand{\geres}{GetEndpointResponse}
%\newcommand{\oscreq}{OpenSecureChannelRequest}
%\newcommand{\oscres}{OpenSecureChannelResponse}
\newcommand{\gereq}{GEReq}
\newcommand{\geres}{GERes}
\newcommand{\oscreq}{OSCReq}
\newcommand{\oscres}{OSCRes}


The {\em OpenSecureChannel} sub-protocol described in
Figure~\ref{fig:secure_channel_se} aims to authenticate a client and
a server and allows them to generate a shared key for the later communications.
\opcua can be used with three security modes, namely {\em \smn}, {\em \sms} and
{\em \smse}.
\vspace{-.5em}
\begin{itemize}
    \item {\em \smse}: claims to provide secrecy using symmetric and
      asymmetric encryption and both authentication and integrity
      through digital signatures (\eg $enc_{pkB}(m)$, $sig_{skA}(h(m))$).
  \item {\em \sms}: same as {\em \smse} but without any encryption.  Thus
      nonces are not used to generate a shared key but bring freshness
      to the messages (\eg $m$, $sig_{skA}(h(m))$).
  \item {\em \smn}: does not provide any security.  Using this mode, the
      {\em OpenSecureChannel} sub-protocol does not serve much
      purpose but is used for compatibility.
\end{itemize}

\vspace{-1em}
\begin{figure}[htb]
    \centering
    \renewcommand{\smname}{\smseshort}
    \resizebox{.8\linewidth}{!}{
        \begin{postscript}
            \begin{msc}{Secure channel creation}
                \setlength{\envinstdist}{1.5\envinstdist}
                \setlength{\instdist}{2.6\instdist}
                \setlength{\labeldist}{1.5\labeldist}

                \declinst{cli}{C}{}
                \declinst{intruder}{DiscoreryEndpoint}{}
                \declinst{sess}{S}{}

                \msccomment[-.5]{1.}{cli}
                \mess{\gereq}{cli}{intruder}
                \nextlevel[1.5]

                \msccomment[-.5]{2.}{cli}
                \mess{\geres, pk(S), \smname, SP, UP}{intruder}{cli}
                \nextlevel[1]
                
                %\action*{Validates pk(S)}{cli}
                %\nextlevel[2]
                
                \action*{Generates N$_{C}$}{cli}
                \nextlevel[3]

                \msccomment[-.5]{3.}{cli}
                \mess{pk(C), \sm{\oscreq, pk(C), $N_{C}$}{pk(S)}{sk(C)}}{cli}{sess}
                \nextlevel[1]

                %\action*{Validates pk(C) }{sess}
                %\nextlevel[2]

                \action*{Generates N$_{S}$ }{sess}
                \nextlevel[2.5]

                %\ifsmnotnone{%
                %    \action*{$(K_{SC}, KSig_{SC})$ = $P\_hash(N_{S}, N_{C})$ }{sess}
                %    \nextlevel[3.5]
                %}

                \msccomment[-.5]{4.}{cli}
                \mess{\sm{\oscres, $N_{S}$, ST, TTL}{pk(C)}{sk(S)}}{sess}{cli}
                %\nextlevel[1]

                %\ifsmnotnone{%
                %    \action*{$(K_{CS}, KSig_{CS})$ = $P\_hash(N_{C}, N_{S})$ }{cli}
                %    \nextlevel[2]
                %}
            \end{msc}
        \end{postscript}
    }
    \caption{\opcua \securechan sub-protocol}
    \label{fig:secure_channel_se}
\end{figure}
\vspace{-1em}

In message 1. $C$ requests information on $S$ with {\em GEReq} meaning
{\em GetEndpointRequest}.
In message 2. $DiscoveryEndpoint$ answers with these information with
{\em GERes} standing for {\em GetEndpointResponse}, {\em SP} for
{\em Security Policy} and {\em UP} for {\em UserPolicy}.
Both {\em SP} and {\em UP} are used for cryptographic primitive negociations.
In message 3. $C$ sends a nonce (here $N_{C}$) to $S$ with {\em OSReq}
standing for {\em OpenSecureChannelRequest}.
Finally in message 4. $S$ answers a nonce to $C$ (here $N_{S}$) with
{\em OSCRes} for {\em OpenSecureChannelResponse}, {\em ST} for
{\em SecurityToken} (a unique identifier for the channel) and {\em TTL} for
{\em TimeToLive} (its life-time).
The four terms {\em GEReq, GERes, OSCReq} and {\em OSCRes} indicate the purpose
of each message of the protocol.
At the end of this protocol, both $C$ and $S$ derive four keys using the nonces
using a function named $P\_hash$, similar to the one specified in
TLS~\cite{DR08}: $(K_{CS}, KSig_{CS})$ = $P\_hash(N_{C}, N_{S})$ and
$(K_{SC}, KSig_{SC})$ = $P\_hash(N_{S}, N_{C})$.

\subsection{Modeling}

Normally, a {\em GetEnpointRequest} would be answered by a list of
session endpoints with possibly different security modes. We suppose
that only one endpoint is answered and that the client will always
accept the security mode proposed.
Client's and server's certificates have been abstracted as their public
keys (which explains why $pk(C)$ appears twice in message 3.
Moreover, thanks to the perfect
encryption hypothesis, we can abstract the cryptographic primitives
used.  We consider an intruder whose public key would be accepted by a
legitimate client or server.  Such an intruder could for instance
represent a legitimate device that has been corrupted through a virus
or that is controlled by a malicious operator.
%Finally, it is said in
%\cite{MLD09} that ``\emph{The establishment of the Secure Channel is
%  mainly used for exchanging special secret information between
%  clients and servers. This secret is used for deriving Symmetric Keys
%  $[...]$}''.  To the best of our knowledge, no more precision is
%given in \cite{MLD09,opcua_part2,opcua_part4,opcua_part6} on which
%cryptographic primitives should be used for such purpose.  Thus we
%chose to model this key derivation using the \DiH mechanisms which
%relies on the commutativity of the exponentiation: $(g^a)^b =
%(g^b)^a$.
In this context, we consider the following security objectives: (i)
the secrecy of the keys obtained by C, (ii) the secrecy of the keys obtained
by S, (iii) the authentication of C on $N_{C}$ and (iv) the authentication of S
on $N_{S}$.

\subsection{Results}\label{sec:secure_channel_res}

We run \proverif on this protocol for of the three security
modes of \opcua for each objective proposed.
Results are shown in Table \ref{tab:secure_channel_results}.

\vspace{-1em}
\begin{table}[htb]
    \centering
    %\hspace{-1em}
    %\resizebox{1.05\columnwidth}{!}{
    \begin{tabular}{|c|c|c|c|c|}
        \hline
        \multirow{2}{*}{\opcua Security mode} & \multicolumn{4}{|c|}{Objectives} \\
        \cline{2-5}
                       & Sec $K_{cli}$ & Sec $K_{srv}$ & Auth $g^{N_{S}}$  & Auth $g^{N_{C}}$  \\
        \hline
        \smn           & \UNSAFE       & \UNSAFE       & \UNSAFE           & \UNSAFE           \\ 
        \hline
        \sms           & \UNSAFE       & \UNSAFE      & \UNSAFE           & \UNSAFE           \\ 
        \hline
        \smseshort     & \SAFE         & \SAFE         & \UNSAFE           & \UNSAFE           \\ 
        \hline
    \end{tabular}
    %}
    \caption{Results for textbook {\em OpenSecureChannel} sub-protocol}
    \label{tab:secure_channel_results}
\end{table}
\vspace{-2em}

Obviously, as the security mode \smn~does not provide any security,
all objectives can be attacked and as nonces are exchanged in plaintext in
security mode \sms, the keys are leaked.
Moreover, attacks on authentication
in the case of \sms~and \smse~implies the intruder rerouting messages.
Such manipulation differs from replaying a message as it does not
delay the message more than what a legitimate router would do (thus
avoiding replay protections such as timestamps).  Figure
\ref{fig:secure_channel_atk} shows an attack on the authentication of
C using $N_{C}$.  This attack is possible because the standard
\opcua protocol does not require explicitly to specify the identity of
the receiver of a message.  Thus it allows the intruder to send to S
the signed message C sent to him similarly as the attack on the
Needham-Schroeder protocol~\cite{Low96}.
%However, these attacks are
%not exploitable in security modes \sms~and \smse~since the secrecy of the keys
%is proven by the tool.  It means that even if an intruder can usurp a
%client when speaking to a server and vice-versa, he will not obtain
%the key derived by the protocol.
%Such property holds thanks to the \DiH key derivation mechanism that allows
%to exchange the nonces without revealing them (sending $g^{N_{C}}$ instead of 
%$N_{C}$).

\vspace{-1em}
\begin{figure}[htb]
    \hspace{-2.25em}
    \renewcommand{\smname}{\smse}
    \resizebox{1.1\columnwidth}{!}{
        \begin{postscript}
            \begin{msc}{Secure channel creation}
                \setlength{\envinstdist}{1.5\envinstdist}
                \setlength{\instdist}{5.1\instdist}
                \setlength{\labeldist}{1.5\labeldist}

                \declinst{cli}{C}{}
                \declinst{intruder}{I}{}
                \declinst{sess}{S}{}

                \mess{\gereq}{cli}{intruder}
                \nextlevel[1.5]

                \mess{\geres, pk(I), \smname, SP, UP}{intruder}{cli}
                \nextlevel[1]
                
                %\action*{Validates pk(I)}{cli}
                %\nextlevel[2]
                
                \action*{Generates N$_{C}$}{cli}
                \nextlevel[3]

                \mess{pk(C), \sm{\oscreq, pk(C), $N_{C}$}{pk(I)}{sk(C)}}{cli}{intruder}
                \nextlevel[2]

                \mess{pk(C), \sm{\oscreq, pk(C), $N_{C}$}{pk(S)}{sk(C)}}{intruder}{sess}
                \nextlevel[1]

                %\action*{Validates pk(C) }{sess}
                %\nextlevel[2]

                %\action*{Generates N$_{S}$ }{sess}
                %\nextlevel[2]

                %\ifsmnotnone{%
                %    \action*{$K_{srv}$ = $(g^{N_{C}})^{N_{S}}$ }{sess}
                %    \nextlevel[3]
                %}

                %\mess{\sm{\oscres, $g^{N_{S}}$, ST, TTL}{pk(C)}{sk(S)}}{sess}{intruder}
                %\nextlevel[1.5]
            \end{msc}
        \end{postscript}
    }
    \caption{Attack on $N_{C}$: I usurps C when speaking to S}
    \label{fig:secure_channel_atk}
\end{figure}

\vspace{1em}
\subsection{Fixed version}\label{sec:secure_channel_fix}

We propose a fixed version of the {\em OpenSecureChannel} sub-protocol.
%in Figure \ref{fig:secure_channel_fix_se}.
It consists in explicitly adding the identity of the receiver to the messages
and thus  avoids an intruder rerouting signed messages to usurp hosts as
presented in Section \ref{sec:secure_channel_res}.
More formally, message 3. and 4. of Figure~\ref{fig:secure_channel_se} are replaced
by:
%\vspace{-15pt}
\arraycolsep=1.4pt
{\small
\begin{flushleft}
    $\begin{array}{lrllll}
        3. & C & \rightarrow & S & : & \text{\sm{\oscreq, {\bf S}, pk(C), $N_{C}$}{pk(S)}{sk(C)}}\\
        4. & S & \rightarrow & C & : & \text{\sm{\oscres, {\bf C}, $N_{S}$, ST, TTL}{pk(C)}{sk(S)}}\\
    \end{array}$
\end{flushleft}
}
%\vspace{-.5em}

Note that instead of adding identity in these messages, one could add the user's
public keys (pk(C) instead of C for instance).
This is one of the classical counter-measures for communication protocols
proposed in \cite{AN96}.
Moreover, the \opcua standards~\cite{MLD09,opcua_part2,opcua_part4,opcua_part6}
advise the use of key wrapping~\cite{FLS11}.
Although the use of such mechanism in security mode \sms~remains unclear in the
standard, we take advantage of it in this fixed version to ensure the secrecy of
derived keys in security mode \sms.
More formally, occurrences of $N_{C}$ are replaced by
$\left\{\mbox{$N_{C}$}\right\}_{pk(S)}$ in message 3 and all occurrences of
$N_{S}$ in message 4 by $\left\{\mbox{$N_{S}$}\right\}_{pk(C)}$.
Thus all the messages is signed in security mode \sms but only the nonces are
encrypted.
We also use \proverif to check the security of this fixed version with each
counter-measure separately and together.
The results are presented in Table \ref{tab:secure_channel_fix_results} and show
that explicitly specifying the receiver of a message in its signature fixes
attacks found on authentication for security modes \sms~and \smse.
They also show that the use of key wrapping secure the secrecy of the keys for
security mode \sms~since the key is never sent in plaintext.

\vspace{-1em}
\begin{table}[htb]
    \centering
    %\hspace{-1em}
    %\resizebox{1.05\columnwidth}{!}{
    \begin{tabular}{|c|c|c|c|c|}
        \hline
        \multirow{2}{*}{\opcua Security mode} & \multicolumn{4}{|c|}{Objectives} \\
        \cline{2-5}
                       & Sec $K_{cli}$ & Sec $K_{srv}$ & Auth $g^{N_{S}}$  & Auth $g^{N_{C}}$  \\
        \hline
        \smn           & \UNSAFE       & \UNSAFE       & \UNSAFE           & \UNSAFE           \\ 
        \hline
        \sms           & \SAFE         & \SAFE         & \SAFE             & \SAFE             \\ 
        \hline
        \smseshort     & \SAFE         & \SAFE         & \SAFE             & \SAFE             \\ 
        \hline
    \end{tabular}
    %}
    \caption{Results for fixed {\em OpenSecureChannel} sub-protocol}
    \label{tab:secure_channel_fix_results}
\end{table}
\vspace{-3em}

    \caption{Diagramme de séquence pour OPC-UA}
    \label{fig:opcua_diagram}
\end{figure}
    
%\begin{figure}[htb]
%    \lstinputlisting[language=caml]{opcua_proverif.pv}
%    \caption{Exemple de modélisation \proverif{}}
%    \label{fig:opcua_proverif}
%\end{figure}

A partir de représentations, nous pouvons modéliser le rôle de chaque agent dans le langage de l'outil de vérification.
L'outil nous répondra alors soit que le protocole est sûre pour les propriétés que nous avons exprimé,
soit par une attaque contre l'une des propriétés.

\section{Conclusion}

Les défis de ce type d'approche sont principalement la difficulté de modéliser le protocole en captant les propriétés
de sécurité qu'il revendique.
L'outil pouvant produire des faux positifs, il faut également comprendre le attaques trouvée et si elles sont fausses,
changer la modélisation pour qu'elles n'apparaissent plus.
Certains outils permettent également à l'utilisateur d'ajouter des théories equationelles afin d'augmenter
le pouvoir de l'intrus.
Il également possible de modéliser des propriétés de sécurité plus complexes telles que la non-répudiation
(impossibilité de nier l'envoie ou la réception d'un message) ou l'intégrité du flux (vérifier que les messages
reçu par un participant sont exactement ceux envoyés par un autre en tenant compte de l'ordre).

\section{Auteurs}

Biographie ... ?

\bibliographystyle{plain}
\bibliography{phdBiblio}

\end{document}

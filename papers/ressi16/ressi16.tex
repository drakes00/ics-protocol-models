\documentclass{article}

%\usepackage{fullpage}

\usepackage[T1]{fontenc}
\usepackage[utf8]{inputenc}

\usepackage[french]{babel}

\usepackage{authblk}

\usepackage{listings}
\usepackage{caption}
\usepackage{subcaption}

\usepackage{graphicx}
\graphicspath{{assets/}}
\makeatletter
    \def\input@path{{assets/}}
\makeatother

\usepackage{xstring}

\usepackage{times}

\usepackage{auto-pst-pdf}
\ifpdf
    \usepackage{todonotes}
    \usepackage{tikz}
    \usetikzlibrary{arrows,automata}
\else
    \usepackage{msc}
\fi

% ========== Don't touch ==========
\makeatletter
    \def\msc@frame{no}
    \def\msc@settitle{}
\makeatother

\newcommand{\dosmnone}[3]{#1}
\newcommand{\dosmsign}[3]{#1, $\left\{\mbox{#1}\right\}_{\mbox{#3}}$}
\newcommand{\dosmsignandencrypt}[3]{$\left\{\mbox{#1}\right\}_{\mbox{#2}}$, $\left\{\mbox{#1}\right\}_{\mbox{#3}}$}

\newcommand{\smn}{None}
\newcommand{\sms}{Sign}
\newcommand{\smse}{SignAndEncrypt}

\newcommand{\sm}{\IfStrEq{\smname}{\smn}{\dosmnone}{\IfStrEq{\smname}{\sms}{\dosmsign}{\dosmsignandencrypt}}}

\newcommand{\ifsmnotnone}[1]{\IfStrEq{\smname}{\smn}{}{#1}}
\newcommand{\smname}{\smn}
% ========== Don't touch ==========

\newcommand{\UNSAFE}{{\color{red!50!black} UNSAFE}}
\newcommand{\SAFE}{{\color{green!50!black} SAFE}}
\newcommand{\proverif}{ProVerif}
\newcommand{\aramis}{ARAMIS}

\title{La sécurité informatique dans les systèmes d'informations industriels}

\author{
    Maxime Puys\\
    Univ. Grenoble Alpes, VERIMAG, F-38000 Grenoble, France\\
    CNRS, VERIMAG, F-38000 Grenoble, France
    \thanks{Ce travail a été partiellement financé par le projet PIA Sécurité Numérique ARAMIS.}
}

\date{}

\begin{document}

\maketitle

%\begin{abstract}
%    aa
%\end{abstract}

\section{Introduction}

De plus en plus d'attaques informatiques contre les systèmes industriels sont
présentées par les médias.
Ces systèmes tendent à devenir géographiquement distribués et communiquer via
des reseaux vulnérables tels qu'Internet.
Depuis son apparition, l'informatique industrielle a toujours été physiquement
isolée du reste du monde.
Les attaques nécessitaient donc d'être présent sur le site et étaient peu
probables.
L'industrie a donc préféré se focaliser sur la protection contre les risques
naturels et les erreurs de manipulation, appelée {\em sûreté}.
Ainsi, les systèmes infdustriels ont grossi au point de devenir omniprésents
dans nos vies tandis que leur sécurité informatique a toujours été presque
inexistante.
La particularité de la sécurité informatique face à la sûreté est la volonté
de nuire de l'attaquant, doublée de sa capacité à réfléchir et à apprendre de
ses erreurs, suivant l'état de l'art de la sécurité informatique.
Aujourd'hui, ces systèmes régissent aussi bien la production et la distribution
d'énergie, les barrages ou le nucléaire et il apparait que leur protection
contre les attaques informatiques doit se mettre à l'état de l'art afin d'éviter
des attaques massives telles que Stuxnet \cite{Lan11}.

Une des difficulté en terme de sécurité est de combiner les propriétés des
protocoles de communication et les attendus métiers en terme de flux.
Pour ce faire, la Section \ref{sec:aramis} expliquera comment filtrer les
messages en tenant compte des aspects métiers.
La Section \ref{sec:protocols} traitera de la vérification formelle des
propriétés des protocoles de communication industriels.
Enfin la Section \ref{sec:models} proposera un modèle pour décrire les attaques
informatiques en fonction de paramètres tels que les objectifs, les capacités
des attaquants et leur positions dans une architecture réseau.

\section{Filtrage des communications}\label{sec:aramis}

Un premier axe repose sur le projet \aramis{} \cite{aramis} qui consiste à
cloisonner physiquement les réseaux et à filtrer les échanges pour rejeter tout
flux identifié comme non autorisé, donc potentiellement malveillant.
La spécificité de ce projet est de fournir simultanément ces deux
fonctionnalités dans un même dispositif robuste, dont l'ajout n'impacte pas le
système existant.
Ceci, tout en le protégeant d'agressions qu'elles soient issues du monde
extérieur ou internes.
%La figure \ref{fig:aramis} montre le placement du dispositif \aramis{} dans un
%système industriel.
Les communications se résument à des requêtes envoyées par des clients vers des
systèmes SCADA (système de contrôle et d'acquisition de données est un système
de télégestion à grande échelle).
Ces requêtes sont principalement des lectures ou des écritures sur des variables
afin de surveiller l'état du processus.
La spécificité des messages nous permet donc d'effectuer un filtrage applicatif
prenant en compte les besoins liés au corps de métier du système industriel à
protéger.
De plus, les systèmes industriels acceptant une latence maximale de l'ordre du
centième de seconde, le filtrage doit être pensé pour une efficacité maximale.

%\begin{figure}
%    \resizebox{\textwidth}{!}{
%        \begin{tikzpicture}[
    arrow/.style={thick,<-,shorten >=2pt,shorten <=2pt,>=stealth},
]

    \draw[dashed] (-.5,-.5) rectangle (6.5,4.5) node [below=.3,left] {Zone de confiance};
    \fill[white] (6.25,1.5) rectangle (6.75,2.5);

    \draw (0,0) rectangle (3,1) node [pos=.5] {$SCADA_{3}$};
    \draw (0,1.5) rectangle (3,2.5) node [pos=.5] {$SCADA_{2}$};
    \draw (0,3) rectangle (3,4) node [pos=.5] {$SCADA_{1}$};
    
    \draw (5,1.5) rectangle (8,2.5) node [pos=.5] {$Aramis$};
    
    \draw (10,0) rectangle (13,1) node [pos=.5] {$Client$};
    \draw (10,1.5) rectangle (13,2.5) node [pos=.5] {$Client_{corrompu}$};
    \draw (10,3) rectangle (13,4) node [pos=.5] {$Intrus$};

    \draw[arrow,green] (3,.5) -- (5,1.5);
    \draw[arrow,green] (3,2) -- (5,2);
    \draw[arrow,green] (3,3.5) -- (5,2.5);

    \draw[arrow,green] (8,1.5) -- (10,.5);
    \draw[arrow,red] (8,2) -- (10,2);
    \draw[arrow,red] (8,2.5) -- (10,3.5);
\end{tikzpicture}

%    }
%    \caption{Placement du dispositif \aramis{}}
%    \label{fig:aramis}
%\end{figure}

\section{Vérification formelle de protocoles industriels}\label{sec:protocols}

%Les protocoles sont un ingrédient essentiel des communications réseaux de notre
%époque.
%Ces communications pouvant être attaquées soit en espionnant le trafic, soit en
%le modifiant,
Il apparait depuis plusieurs décennies que la preuve formelle de
la sécurité de ces protocoles est un enjeu majeur.
Un exemple connu est {\em mi.TLS} \cite{BFKPS13}, une implémentation prouvée
du protocole TLS \cite{DR08} servant par exemple au paiement en ligne.
%Les traveaux d'analyse de protocoles ont commencé avec la contribution de
%G. Lowe \cite{Low96}, brisant le protocole {\em Needham-Schoeder} en 1995.
%Depuis, les outils d'analyse et les travaux de modélisation se sont multipliés.
%Ces outils reposent sur des approches théoriques variées telles que la
%réécriture, la résolution de contrainte, les clauses de Horn ou les
%{\em SAT-solvers}.
%Ces outils considèrent un intrus dit de Dolev-Yao \cite{DY81} qui contrôle le
%réseau, espionne, stoppe, forge, modifie, entrelace ou rejoue des messages en
%utilisant la connaissance des messages qu'ils appris précédemment.
Il existe plusieurs outils permettant de modéliser des protocoles afin d'en
tester la sécurité.
Ces outils considèrent des propriétés telles que l'{\em autentification}
et le {\em secret}.
%La première propriété signifie qu'un participant est conviancu qu'il parle avec
%un autre.
%Le secret assure qu'un agent non autorisé (y compris l'intrus) n'accède pas au
%message désigné.
%Enfin, les outils se basent sur l'hypothèse du chiffrement parfait, selon
%laquelle il n'est pas possible de déchiffrer un message sans la chef de
%chiffrement ou d'usurper une signature.

Cependant, les protocoles de communication des systèmes industriels n'ont jamais
été vérifiés de cette façon.
Nous souhaitons donc tester formellement la sécurité de ces protocoles à l'aide
d'outils de vérification de protocoles.
%En violant le secret, un intrus pourrait compromettre une clef de chiffrement,
%ce qui révèlerait le contenu des communication.
%D'autre part, en attaquant l'autentification, un intrus pourrait se faire passer
%pour un serveur aux yeux d'un client ou beaucoup plus grave, se faire passer
%pour un client aux yeux d'un serveur.
%Ce scénario gravissime autoriserait l'intrus à lancer des commandes ayant un
%réel impact sur le monde extérieur.
Les défis de ce type d'approche appliquée aux protocoles industriels viennent
de la particularité des propriétés de sécurité à considérer qui demandent une
modélisation adaptée des protocoles.
Par exemple, vérifier l'intégrité du flux stipulant que les messages reçu par
un participant sont exactement ceux envoyés par un autre en tenant compte de
l'ordre.

\section{Modèles d'attaques}\label{sec:models}

Enfin, nous souhaitons tirer parti des travaux sur les méthodes d'analyse
de risques telles que EBIOS \cite{EBIOS} et MEHARI \cite{MEHARI}.
Ces méthodes servent à générer automatiquement des scénarios d'attaques en
fonction des objectifs des attaquants et des biens que l'on souhaite protéger.
De la même façon, nous modélisons les attaques informatiques sur les systèmes
industriels en fonction des capacités des attaquants (placements dans le
système, protocoles et possibilités de nuisance) et de leurs objectifs possibles
(atteinte à la configentialité ou  l'intrité d'un message, de la configuration
d'un hôte, etc).
En suite, à l'aide d'une approche déductive tenant compte de l’analyse métier
d’un système industriel, nous calculons s'il existe une configuration où un
attaquant positionné à un endroit de l'architecture, avec des capacités
dépendant des propriétés de sécurité des protocoles de communication pourrait
atteindre l'un de ses objectifs d'attaque.
Ainsi, nous sommes en mesure de spécifier quels sont scénarii d’attaques sont à
prendre en compte ou non pour faciliter la phase de test d'un système.
Cette approche vise à être automatisée dans un outil.

\section{Auteur}

Maxime Puys est doctorant en deuxième année au laboratoire Verimag de
l'Université Grenoble Alpes.
Sous la direction de Marie-Laure Potet (Prof. Grenoble INP) et Jean-Louis Roch
(MCF. Grenoble INP), il travaille sur à la sécurité informatique dans les
systèmes d'informations industriels.
Diplômé en 2014 du {\em Master Sécurité, Audit et Informatique Légal} de
l'Université Grenoble Alpes, il a partcipé pendant deux ans à des recherches sur
la sécurité des cartes-à-puce contre les attaques par fautes dans le laboratoire
Verimag et l'entreprise SAFRAN Morpho.
Intéressé par les méthodes formelles dans le domaine de la sécurité, il
participe à les intégrer dans les projets concrets des industriels.

\bibliographystyle{plain}
\bibliography{phdBiblio}

\end{document}

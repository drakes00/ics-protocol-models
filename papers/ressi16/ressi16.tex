\documentclass{article}

\usepackage[T1]{fontenc}
\usepackage[utf8]{inputenc}

\usepackage[french]{babel}

\usepackage{authblk}

\usepackage{graphicx}
\usepackage{xstring}

\usepackage{auto-pst-pdf}
\ifpdf
    \usepackage{todonotes}
    \usepackage{tikz}
    \usetikzlibrary{arrows,automata}
\else
    \usepackage{msc}
\fi

% ========== Don't touch ==========
\makeatletter
    \def\msc@frame{no}
    \def\msc@settitle{}
\makeatother

\newcommand{\dosmnone}[3]{#1}
\newcommand{\dosmsign}[3]{#1, $\left\{\mbox{#1}\right\}_{\mbox{#3}}$}
\newcommand{\dosmsignandencrypt}[3]{$\left\{\mbox{#1}\right\}_{\mbox{#2}}$, $\left\{\mbox{#1}\right\}_{\mbox{#3}}$}

\newcommand{\smn}{None}
\newcommand{\sms}{Sign}
\newcommand{\smse}{SignAndEncrypt}

\newcommand{\sm}{\IfStrEq{\smname}{\smn}{\dosmnone}{\IfStrEq{\smname}{\sms}{\dosmsign}{\dosmsignandencrypt}}}

\newcommand{\ifsmnotnone}[1]{\IfStrEq{\smname}{\smn}{}{#1}}
\newcommand{\smname}{\smn}
% ========== Don't touch ==========

\newcommand{\UNSAFE}{{\color{red!50!black} UNSAFE}}
\newcommand{\SAFE}{{\color{green!50!black} SAFE}}
\newcommand{\proverif}{ProVerif}

\title{Verification formelle de protocoles industriels}

\author[1]{Maxime Puys}
\author[1]{Marie-Laure Potet}
\author[2]{Jean-Louis ?}
\author[3]{Pascal ?}

\affil[1]{
    Univ. Grenoble Alpes, VERIMAG, F-38000 Grenoble, France\\
    CNRS, VERIMAG, F-38000 Grenoble, France
}

\date{}

\begin{document}

\maketitle

%\begin{abstract}
%    aa
%\end{abstract}

\section{Introduction}

Les protocoles sont un ingrédient essentiel des communications réseaux de notre époque.
Ces communications pouvant être attaquées soit en espionnant le trafic, soit en le modifiant,
il apparait depuis plusieurs décennies que la preuve formelle de la sécurité de ces protocoles
est un enjeu majeur.
Un exemple connu étant {\em mi.TLS} \cite{BFKPS13}, une implémentation prouvée du protocole TLS \cite{DR08}
servant par exemple au paiement en ligne.
Cependant, les protocoles de communication des systèmes industriels servant à envoyer des commandes
et recevoir des informations n'ont jamais été vérifiés de cette façon.
Ces protocoles régissant parfois jusqu'à des centrales nucléaires, des barrages ou la distribution d'énergie,
il est vital de s'assurer de leur sécurité.

Les traveaux d'analyse de protocoles hors du cadre industriel ont commencé avec la contribution de
G. Lowe \cite{Low96}, brisant le protocole {\em Needham-Schoeder} en 1995.
Depuis, les outils d'analyse et les travaux de modélisation se sont multipliés.
Ces outils reposent sur des approches théoriques variées telles que la réécriture,
la résolution de contrainte ou les clauses de Horn ou les {\em SAT-solvers}.
Ces outils considèrent un intrus dit de Dolev-Yao \cite{DY81} qui contrôle le réseau,
espionne, stoppe, forge, modifie, entrelace ou rejoue des messages en utilisant la connaissance
des messages qu'ils appris précédemment.
En jouant en parallèle plusieurs sessions avec les différents rôles du protocole,
l'intrus tente de violer des propriétés telles que l'{\em autentification} et le {\em secret}.
La première propriété signifie qu'un participant est conviancu qu'il parle avec un autre.
Le secret assure qu'un agent non autorisé (y compris l'intrus) n'accède pas au message désigné.
Enfin, les outils se basent sur l'hypothèse du chiffrement parfait, selon laquelle
il n'est pas possible de déchiffrer un message sans la chef de chiffrement ou d'usurper une signature.

En particulier, \proverif{} \cite{Bla01,PROVERIF14_manual}, développé par B. Blanchet et al. analyse un protocole écrit
soit en clauses de Horn soit dans un sous ensemble du Pi-Calcul pour un nombre non borné
de sessions.
Via des techniques de sur-approximation, il est capable de prouver la sécurité d'un protocole
mais les attaques qu'il trouvent peuvent être des faux-positifs.
Lorsqu'une attaque est trouvée, l'outil reconstruit une trace montrant sa réalisation.

\section{Le protocol OPC-UA}

OPC-UA \cite{opc-ua} est un protocole de communication qui devient un standard dans le milieu industriel.
Ce nouveau protocole très complexe de part la diversité des données qu'il peut transférer
se divise en plusieurs couches. 
Parmi elles, la couche {\em Secure Conversation} est en charge de la sécurité en termes d'authentification,
de contrôle d'accès, d'intégrité et de chiffrement.

OPC-UA

Proverif

Objectifs

\section{Authors}

Biographie ... ?

\end{document}

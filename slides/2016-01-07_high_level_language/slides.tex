\documentclass{beamer}

\usepackage[T1]{fontenc}
\usepackage[utf8]{inputenc}

\usepackage{amsmath}
\usepackage{listings}
\lstset{
    basicstyle=\footnotesize,
    language=Python,
    showstringspaces=false,
    escapeinside={<@}{@>},
}


\usepackage{amssymb}% http://ctan.org/pkg/amssymb
\usepackage{pifont}% http://ctan.org/pkg/pifont
\newcommand{\cmark}{{\color{green!60!black}\ding{51}}}%
\newcommand{\xmark}{{\color{red}\ding{55}}}%


\usetheme{Boadilla}
\usecolortheme{dolphin}
\useoutertheme{infolines}

\setbeamertemplate{footline}{
    \leavevmode%
    \hbox{%
    \begin{beamercolorbox}[wd=.333333\paperwidth,ht=2.25ex,dp=1ex,center]{author in head/foot}%
        \usebeamerfont{author in head/foot}\insertshortauthor%~~\beamer@ifempty{\insertshortinstitute}{}{(\insertshortinstitute)}
    \end{beamercolorbox}%
    \begin{beamercolorbox}[wd=.333333\paperwidth,ht=2.25ex,dp=1ex,center]{title in head/foot}%
        \usebeamerfont{title in head/foot}\insertshorttitle
    \end{beamercolorbox}%
    \begin{beamercolorbox}[wd=.333333\paperwidth,ht=2.25ex,dp=1ex,right]{date in head/foot}%
        \usebeamerfont{date in head/foot}\insertshortdate{}\hspace*{2em}
        \insertframenumber{} / 9%\inserttotalframenumber
                                     \hspace*{2ex}
    \end{beamercolorbox}}%
    \vskip0pt%
}


\graphicspath{{assets/}}
\makeatletter
    \def\input@path{{assets/}}
\makeatother

\newcommand{\ARAMIS}{ARAMIS}

\title{Langage haut niveau pour \ARAMIS{}}
\subtitle{Configuration et règles avant compilation}
\author[Maxime Puys]{Maxime Puys \and Marie-Laure Potet \and Jean-Louis Roch}
\institute{VERIMAG, Univ. Grenoble Alpes}
\date{7 janvier 2016}



\begin{document}

\begin{frame}
    \maketitle
\end{frame}

%\section{Top-down approach}

%\begin{frame}
%    \frametitle{Table of contents}
%
%    \tableofcontents[currentsection]
%\end{frame}

\begin{frame}
    \frametitle{Contexte}

    \begin{figure}[htb]
        \resizebox{\textwidth}{!}{
            \includegraphics{process}
        }
        \caption{Processus de génération des règles}
    \end{figure}
    \vfill
    \begin{itemize}
        \item On s'intéresse ici aux "Règles métier (langage)" sur la gauche.
        \begin{itemize}
            \item {\bf Contient les données de configuration.}
        \end{itemize}
    \end{itemize}
\end{frame}

\begin{frame}[fragile]
    \frametitle{Exemple de langage 1/2}

    \begin{lstlisting}
<@{\color{blue}\# Configuration}@>
<@{\color{red} net1}@> = Network("10.0.0.0/24")
<@{\color{orange} srv}@> = Server(<@{\color{red} net1}@>, ip="10.0.0.1", protocol=Protocol.MODBUS)

<@{\color{yellow!80!black} a}@> = Automaton(<@{\color{orange} srv}@>)
<@{\color{yellow!80!black} a}@>.enableRead()
<@{\color{yellow!80!black} a}@>.enableWrite()

<@{\color{purple} temp1}@> = Variable(uint_16, "%xi5")
<@{\color{purple} temp1}@>.enableRead()
<@{\color{purple} temp1}@>.setValues([0,5])

<@{\color{yellow!80!black} a}@>.addVariable(<@{\color{purple} temp1}@>)

<@{\color{green!60!black} net2}@> = Network("10.1.0.0/24")
<@{\color{cyan} cli}@> = Client(<@{\color{green!60!black} net2}@>, ip="10.1.0.2")
    \end{lstlisting}
\end{frame}

\begin{frame}[fragile]
    \frametitle{Exemple de langage 2/2}

    \begin{lstlisting}
<@{\color{blue}\# Utilisateurs}@>
<@{\color{cyan!60!red} alice}@> = User("Alice", certificate="...")
<@{\color{yellow!60!orange} bob}@> = User("Bob")
UserGroups["SUPERVISORS"] = [<@{\color{cyan!60!red} alice}@>]
UserGroups["OPPERATORS"] = [<@{\color{yellow!60!orange} bob}@>]

<@{\color{blue}\# Regles}@>
Global.enableRead()
Global.enableWrite()
Global.enableAction()
Global.enableProtocol(Protocol.MODBUS)

<@{\color{red} chan}@> = Channel(<@{\color{orange} srv}@>, <@{\color{cyan} cli}@>)
<@{\color{red} chan}@>.enableRead()
<@{\color{red} chan}@>.enableWrite()
<@{\color{red} chan}@>.addRule("SUPERVISORS", FileTypeRule("pdf"))
<@{\color{red} chan}@>.addRule("SUPERVISORS", FileMaxSizeRule("5024Ko"))
<@{\color{red} chan}@>.addRule("SUPERVISORS", FileSignatureRule("..."))
<@{\color{red} chan}@>.addRule("OPPERATORS",  VariableAccessRule(<@{\color{purple} temp1}@>))
    \end{lstlisting}
\end{frame}

\begin{frame}[fragile]
    \frametitle{Vers OPC-UA}

    Actuellement une variable est définie par :
    \begin{itemize}
        \item varType = Son type (ex: uint\_16).
        \item services = Les types de services autorisés (Read/Write/Action).
        \item validValues = La plage de valeurs valides.
        \item A ajouter : Fréquence d'accès, battements, ...
    \end{itemize}
    \vfill
    Cf. PRE.021, il faudrait faire apparaître :
    \begin{itemize}
        \item VariantType = Type custom ?
        \begin{itemize}
            \item Si oui, ajout d'une classe pour les définir (attributs équivalents à ceux du langages de com interne).
        \end{itemize}
        \item Type (Numeric/String/...) = Prend la place de varType.
        \item SpaceID et AttributeID = Ajout des attributs (deux nombres).
        \item Règles applicables = Pour des règles sur plusieurs variables (uniquement les règles cinématiques)
    \end{itemize}
    Quid du mapping ? Si remplacé par les nouveaux champs, retirer l'attribut ?
\end{frame}

\begin{frame}[fragile]
    \frametitle{Vers OPC-UA}

    \begin{lstlisting}
<@{\color{blue}\# Variable Modbus, input register 0x10 = short RO $\in [0,5]$}@>
<@{\color{purple} vModbus}@> = Variable(
    varType=1, <@{\color{blue}\# 1 = Numeric}@>
    spaceId=1, <@{\color{blue}\# 0 = bools, 1 = shorts}@>
    attributeID=0x10 <@{\color{blue}\# Offset 0x10 = 16}@>
)
<@{\color{purple} vModbus}@>.enableRead()
<@{\color{purple} vModbus}@>.setValues([0,5])

<@{\color{blue}\# Variable OPC-UA, PRE.021}@>
<@{\color{purple} vOPCUA}@> = Variable(
    varType=1,
    spaceId=0, <@{\color{blue}\# 0 = bools, 1 = shorts}@>
    attributeID=13, <@{\color{blue}\# Offset 5}@>
    variantType=6,
    objectID=2259
)
    \end{lstlisting}
\end{frame}

\begin{frame}
    \frametitle{Points à vérifier}

    \begin{itemize}
        \item Quels attributs dans la classe VariantType ?
            \vfill
        \item Est-ce que le VariantType peut contenir les types de services autorisés (RO/RW/...) ?
            \vfill
        \item Si oui, quelle priorité entre ceux définis pour le type et ceux définis pour la variable ?
            \vfill
        \item Est-ce qu'une variable peut avoir une plage de valeurs valides non contigüe ? (e.g.: [0,..,5,10,..,15], \{0,2,4,6,..\})
    \end{itemize}
\end{frame}

\begin{frame}
    \frametitle{Idée}
    
    \begin{block}{Langage Ad-Hoc}
        \begin{itemize}
            \item Fermé.
            \item Difficilement extensibles.
            \item Rigide syntaxiquement.
        \end{itemize}
    \end{block}
    \vfill
    \begin{block}{Sous-ensemble d'un langage de programmation}
        \begin{itemize}
            \item Composable (deux fichiers peuvent facilement se combiner).
            \item Extensible (ajout de classes/structures).
            \item Syntaxe souple.
            \item Parseurs existants.
        \end{itemize}
    \end{block}
\end{frame}

\begin{frame}
    \frametitle{Candidat possible : C}
    
    \begin{block}{Processus}
        \begin{enumerate}
            \item On fournit une librairie aux exploitants d'\ARAMIS{} (.so et .h).
            \item L'exploitant écrit donc un fichier .c appelant des fonction de la lib.
            \item Ce fichier est compilé puis exécuté hors du module afin de produire les règles sous forme de texte.
        \end{enumerate}
    \end{block}
    \vfill
    \begin{block}{Problèmes}
        \begin{itemize}
            \item[\xmark] Besoin d'un compilateur sur la machine de l'exploitant (ex. GCC).
            \item[\xmark] Solution peu multiplateforme (32 bits vs. 64 bits, Windows vs. Linux vs. Mac OS, ...).
        \end{itemize}
    \end{block}
\end{frame}

\begin{frame}
    \frametitle{Choix actuel : Python}
    
    \begin{block}{Avantages}
        \begin{itemize}
            \item[\cmark] Solution multiplateforme (32 bits vs. 64 bits, Windows vs. Linux vs. Mac OS, ...).
        \end{itemize}
    \end{block}
    \vfill
    \begin{block}{Problèmes}
        \begin{itemize}
            \item[\xmark] Besoin d'un interpréteur Python sur la machine de l'exploitant.
            \item[\xmark] Lent.
            \begin{itemize}
                \item[\cmark] Hors du module donc pas de contrainte.
                \item[\cmark] Possibilité de déléguer des tâches à C (toujours avec problèmes de plateforme).
            \end{itemize}
        \end{itemize}
    \end{block}
\end{frame}

\begin{frame}
    \frametitle{Choix actuel : Python}
    
    \begin{block}{Processus 1}
        \begin{enumerate}
            \item On fournit une librairie (un package) aux exploitants d'\ARAMIS{} (.py).
            \item L'exploitant écrit donc un fichier .py appelant des fonction de la lib.
            \item Ce fichier est interprété par Python hors du module afin de produire les règles sous forme de texte.
        \end{enumerate}
    \end{block}
    \vfill
    \begin{block}{Processus 2}
        \begin{enumerate}
            \item On fournit un logiciel qui parse du Python et qui contient une librairie aux exploitants d'\ARAMIS{} (.py).
            \item L'exploitant écrit donc un fichier .py appelant des fonction de la lib.
            \item Ce fichier est parsé par le logiciel \ARAMIS{} hors du module afin de produire les règles sous forme de texte.
            \begin{itemize}
                \item Différence : possibilité d'une IHM, ...
            \end{itemize}
        \end{enumerate}
    \end{block}
\end{frame}

\begin{frame}
    \frametitle{Questions diverses}

    \begin{itemize}
        \item Un utilisateur peut-il avoir plusieurs rôles, sur un même client, sur un même channel ?
            \vfill
        \item Une variable peut-elle avoir différentes valeurs valides sur différents channels ?
        \item Une variable peut-elle avoir différentes fréquences d'accès sur différents channels ?
            \vfill
        \item Une variable peut-elle avoir différentes valeurs valides pour différents utilisateurs ?
        \item Une variable peut-elle avoir différentes fréquences d'accès pour différents utilisateurs ?
            \vfill
        \item Deux automates d'un même serveur peuvent-ils avoir une variable identique ?
    \end{itemize}
\end{frame}

\end{document}

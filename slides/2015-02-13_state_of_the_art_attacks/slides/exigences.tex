\section{ANSSI requirements}

\begin{frame}
    \frametitle{Objectives}

    Extracted from {\em La cybers\'ecurit\'e des syst\`emes industriels -- Mesures d\'etaill\'ees}.
    \vfill
    \begin{block}{Objectives of the document}
        Written by a work group piloted by ANSSI and composed of actors of the industry domain and IT security field.\\
        \medskip
        It aims at proposing a set of measures in order to improve the IT security level of industrial systems.
    \end{block}
    \vfill
    \begin{block}{Objectives of the study}
        Find which requirements could be applied to ARAMIS.\\
        \medskip
        Some can be applied directly to the module.\\
        Some can be taken as hypothesis.
    \end{block}
\end{frame}

\begin{frame}
    \frametitle{Field of study}

    The spreadsheet regroups measures for chapter 4. {\em Mesures de s\'ecurit\'e techniques}.
    \vfill
    \begin{block}{Composition}
        This chapter is divided into 4 sections:
        \begin{enumerate}
            \item {\em Authentification des intervenants : contr\^ole d'acc\`es logique}.
            \item {\em S\'ecurisation de l'architecture du syst\`eme industriel}.
            \item {\em S\'ecurisation des \'equipements}.
            \item {\em Surveillance du syst\`eme industriel}.
        \end{enumerate}
    \end{block}
\end{frame}

\begin{frame}
    \frametitle{{\em Authentification des intervenants : contr\^ole d'acc\`es logique}}

    This section deals with the repartition of users within the system and the way they authenticate.
    \vfill
    \begin{block}{Within ARAMIS}
        This section can be related to either the user accounts on the CPUS of the module or the accounts that operators can use to connect for example to an OPC-UA server.\\
        \medskip
        The general idea is to disable unused accounts and strictly check for their priviledges.\\
        It also gives recommendation on the use of passwords {\bf and raises the question of who is responsible to the generation and retrieval of passwords}.
    \end{block}
\end{frame}

\begin{frame}
    \frametitle{{\em S\'ecurisation de l'architecture du syst\`eme industriel} 1/2}

    This section deals with the security of the industrial system (not only IT).
    \vfill
    \begin{block}{Within ARAMIS}
        This section is interesting because it will help refining the vision on what is outside the module.\\
        \medskip
        It emphasizes the need of unidirectional trafic depending on the zone classifications involved (1, 2 or 3).\\
    \end{block}
\end{frame}

\begin{frame}
    \frametitle{{\em S\'ecurisation de l'architecture du syst\`eme industriel} 2/2}
    
    \begin{block}{Within ARAMIS}
        It also raises questions of:
        \begin{itemize}
            \item Internet access {\bf from} the module and {\bf to} the module.
            \item Remote maintenance.
            \item Distributed industrial systems (out of the scope?).
            \item Wireless connections (out of the scope?).
            \item The possility to force secure protocol in the high zone (or opposite?).
        \end{itemize}
    \end{block}
\end{frame}

\begin{frame}
    \frametitle{{\em S\'ecurisation des \'equipements} 1/2}

    This section deals with how to increase the robustness of applications and the devices that run them.
    \vfill
    \begin{block}{Within ARAMIS}
        One of the key point would be the hardening of the UNIX configurations on the module.\\
    \end{block}
\end{frame}

\begin{frame}
    \frametitle{{\em S\'ecurisation des \'equipements} 2/2}
    
    \begin{block}{Within ARAMIS}
        It raises the questions of:
        \begin{itemize}
            \item Who is responsible of updating when a vulnerability is found in a thrid party software.
            \item The presence of USB ports? If present, a decontamination phase?
            \item On the presence of DHCP in the network.
            \item On the securization of remote maintaining (e.g.: storage of SSH key to access the module).
            \item To what extend the code we might produce has to be reviewed.
        \end{itemize}
    \end{block}
\end{frame}

\begin{frame}
    \frametitle{{\em Surveillance du syst\`eme industriel}}

    This section deals with the logging policy.
    \vfill
    \begin{block}{Within ARAMIS}
        It mainly raises the questions of:
        \begin{itemize}
            \item What should be logged.
            \item If the logs are automaticaly pushed on a remote server or fetched manualy.
            \item On how to correlate information stored in (cf. Emmanuel).
            \item How they should be protected (in case of use for legal purpose).
        \end{itemize}
    \end{block}
\end{frame}

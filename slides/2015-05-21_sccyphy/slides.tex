\documentclass{beamer}

\usepackage[T1]{fontenc}
\usepackage{inputenc}

\usepackage{amsmath}
\usepackage{listings}
\lstset{
  basicstyle=\footnotesize,
  language=Caml,
  showstringspaces=false,
}


\usetheme{Boadilla}
\usecolortheme{dolphin}
\useoutertheme{infolines}


\setbeamertemplate{footline}
{
  \leavevmode%
  \hbox{%
  \begin{beamercolorbox}[wd=.333333\paperwidth,ht=2.25ex,dp=1ex,center]{author in head/foot}%
    \usebeamerfont{author in head/foot}\insertshortauthor%~~\beamer@ifempty{\insertshortinstitute}{}{(\insertshortinstitute)}
  \end{beamercolorbox}%
  \begin{beamercolorbox}[wd=.333333\paperwidth,ht=2.25ex,dp=1ex,center]{title in head/foot}%
    \usebeamerfont{title in head/foot}\insertshorttitle
  \end{beamercolorbox}%
  \begin{beamercolorbox}[wd=.333333\paperwidth,ht=2.25ex,dp=1ex,right]{date in head/foot}%
    \usebeamerfont{date in head/foot}\insertshortdate{}\hspace*{2em}
    \insertframenumber{} / \inserttotalframenumber\hspace*{2ex}
  \end{beamercolorbox}}%
  \vskip0pt%
}


\newcommand{\TODO}{{\color{red}\bf [TODO]}}
\graphicspath{ {assets/} }


\title[Certified filters for control systems]{Certified filters generation process for control systems}
\author[Maxime Puys]{Maxime Puys\\Supervisors: Marie-Laure Potet, Jean-Louis Roch}
\institute{Verimag, UGA - MOAIS, INRIA}
\date{May 21, 2015}


\begin{document}

\begin{frame}
    \maketitle
\end{frame}

\begin{frame}
    \frametitle{Short introduction}

    \begin{itemize}
        \item Master 2 SAFE:
        \begin{itemize}
            \item Apprenticeship in SAFRAN Morpho on faults attacks against smart cards.
        \end{itemize}
            \vfill
        \item Magister in computer science:
        \begin{itemize}
            \item Internship in Verimag also on faults attacks against smart cards.
        \end{itemize}
    \end{itemize}
\end{frame}

\begin{frame}
    \frametitle{ARAMIS Project 1/2}

    \begin{itemize}
        \item My PhD is funded through PIA S\'ecurit\'e Num\'erique ARAMIS (2014-2018) : Architecture Robuste pour les Automates et Mat\'eriels des Infrastructures Sensibles.
            \vfill
        \item Partners: ATOS World Grid, SecLab, CEA-leti, SCCyPhy.
            \vfill
        \item Objectives:
        \begin{enumerate}
            \item Define security requirements for industrial system's cyber-security.
            \item Building a module providing protocol break and applicative filtering.
        \end{enumerate}
            \vfill
        \item Other PhD students working on the project: Nicolas Kox, Jean-Baptiste Orfila and Emmanuel Perrier.
    \end{itemize}
\end{frame}

\begin{frame}
    \frametitle{ARAMIS Project 2/2}

    \begin{figure}[htb]
        \centering
        \includegraphics[scale=.5]{model1} % remplacer machine par PLC, remplacer titre
    \end{figure}
    \vfill
    \vspace{-3em}
    \begin{itemize}
        \item I will participate within three parts:
        \begin{itemize}
            \item Security requirements
            \item Applicative filtering,
            \item Certification process.
        \end{itemize}
    \end{itemize}
\end{frame}

\begin{frame}
    \frametitle{Industrial systems}

    \begin{itemize}
        \item Present differences with traditional systems:
            \vfill
        \begin{itemize}
            \item Really long-term installations, hard to patch, lot of legacy hosts.
                \vfill
            \item Almost always fixed network addressing (no DHCP).
                \vfill
            \item Insert a new component should not imply to modify existing devices.
                \vfill
            \item Industrial protocols, often proprietary.
                \vfill
            \item Security objectives are different from traditional systems:
            \begin{itemize}
                \item For audibility reasons and sometime performance, confidentiality is often avoided.
                \item Disponibility, authentication, integrity and non-repudiation are however very important.
            \end{itemize}
                \vfill
            \item Industrial systems usually implement safety be rarely security.
        \end{itemize}
    \end{itemize}
\end{frame}

%\begin{frame}
%    \frametitle{Industrial systems 2/2}
%
%    \begin{itemize}
%        \item Industrial protocols can be divided in two categories:
%            \vfill
%        \begin{itemize}
%            \item {\em Low-level}: Hardly no security at all, used to READ/WRITE in registers (e.g.: MODBUS, FTP)
%                \vfill
%            \item {\em High-level}: Tend to web-services, object oriented, more complexe function such as remote calls (e.g.: OPC-UA)
%        \end{itemize}
%    \end{itemize}
%\end{frame}

\begin{frame}
    \frametitle{Security requirements}

    \begin{itemize}
        \item Reflection on attack models:
        \begin{itemize}
            \item Where could an intruder be?
            \item What is he able to do?
            \item Propose mitigation solutions.
        \end{itemize}
    \end{itemize}
    \vspace{-3em}
    \begin{figure}[htb]
        \centering
        \includegraphics[scale=.7]{generic}
        \caption{Generic architecture}
    \end{figure}
\end{frame}

\begin{frame}
    \frametitle{Applicative filtering}

    \begin{itemize}
        \item Three types of filters will appear within the module:
        \begin{itemize}
            \item Firewall (e.g.: 10.0.1 can speak with 1.0.0.2).
            \item Protocol: with check for malformed frames.
            \item {\bf Applicative: Semantic filtering}.
        \end{itemize}
    \end{itemize}
    \vfill
    \begin{itemize}
        \item Industrial protocols: Mostly READ/WRITE of registers.
        \begin{itemize}
            \item Blacklist/Whitelist of registers: R5 on 10.0.0.1 is read-only.
            \item Regular expressions: R5 on 10.0.0.1 and R1 on 1.0.0.2 should never be True at the same time.
            \item Temporal expressions: 10.0.0.1 and 1.0.0.2 should can speak together only between 9AM and 9PM.
            \item Security expressions: 10.0.0.1 and 10.0.0.2 should authenticate each other.
        \end{itemize}
    \end{itemize}
\end{frame}

\begin{frame}
    \frametitle{Protocol analysis}

    \begin{itemize}
        \item Objective: Establish a benchmark of all protocols analyzers that we know about dealing with algebraic properties
        \begin{itemize}
            \item Obtain timings and memory usage on a variety of protocols (including well known such as SSH or IKEv2)
            \item Be able to explain the results of each tool depending on the parameters of each protocols %(number of participants, transitions, exclusive-ors, Diffie-Hellman exponentiations, ...)
        \end{itemize}
            \vfill
        \item Within my thesis: develop skills in modelization using various tools
        \begin{itemize}
            \item Helpful to provide and test attack scenarios on our model
            \item Could prove the robustness of ad-hoc protocols if some are constructed within ARAMIS
            \item Could help secure weak industrial protocols
        \end{itemize}
    \end{itemize}
\end{frame}

\begin{frame}
    \frametitle{Non-repudiation}
    
    \begin{itemize}
        \item In 2009, Vigneron et al. proposed a fair non-repudiation protocol using a smart card on the receiver side.
            \vfill
        \begin{itemize}
            \item Given the particularities of industrial systems, we have the intuition that it can be lightened
                \vfill
            \item More precisely, we propose to allow the protocol to authenticate the originator instead of the sender.
                \vfill
            \item As a trade of, it would drop the secrecy of the message.
        \end{itemize}
        %    \vfill
        %\item Reflexions started quite lately
    \end{itemize}
\end{frame}

\begin{frame}
    \frametitle{Certification using formal methods}

    \begin{columns}[c]
        \begin{column}{.4\textwidth}
            \begin{figure}
                \centering
                \includegraphics[scale=.5]{levels}
                \caption{Language levels}
            \end{figure}
        \end{column}
        \begin{column}{.6\textwidth}
            \begin{itemize}
                \item Formalize security objectives:
                    \vfill
                \begin{itemize}
                    \item Lot of works on secrecy, authentication, integrity of messages
                        \vfill
                    \item Lack of formalization with more general objectives (filtering policy, integrity of the configuration, flow analysis)
                \end{itemize}
            \end{itemize}
        \end{column}
    \end{columns}
\end{frame}

\begin{frame}
    \frametitle{Conclusion}

    \begin{itemize}
        \item Scientific challenges:
        \begin{itemize}
            \item Associate safety and security within industrial systems.
            \item Achieve end-to-end security (verifying the system and its environment)
        \end{itemize}
            \vfill
        \item Technical challenges: 
        \begin{itemize}
            \item Filtering language (should be usable by industry professionals).
            \item Performance constraints.
            \item Secure coding.
        \end{itemize}
            \vfill
        \item Industrial challenges: 
        \begin{itemize}
            \item No modification to existing devices.
        \end{itemize}
            \vfill
        \item Attendance at meetings, contact with insiders.
    \end{itemize}
\end{frame}

\begin{frame}
    \frametitle{Conclusion}

    \begin{center}
        Thank you for your attention
    \end{center}
\end{frame}

\end{document}

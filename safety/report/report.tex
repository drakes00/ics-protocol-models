\documentclass{article}

\usepackage[utf8]{inputenc}
\usepackage[T1]{fontenc}
\usepackage{listings}
\usepackage{color}
\usepackage{xspace}

\newcommand{\TODO}[1]{{\bf\color{red} TODO: #1}\xspace}
\newcommand{\opcua}{OPC-UA\xspace}
\newcommand{\modbus}{MODBUS\xspace}
\newcommand{\assi}{PEPS/CNRS ASSI\xspace}
\newcommand{\sis}{systèmes industriels\xspace}
\newcommand{\safety}{sûreté\xspace}
\newcommand{\proverif}{ProVerif\xspace}
\newcommand{\tamarin}{Tamarin\xspace}

\title{
    Rapport d'expérimentation des propriétés de \safety avec les outils de
    vérification de protocoles
}
\author{
    Maxime Puys \\
    VERIMAG, Université Grenoble Alpes \\
    prenom.nom@univ-grenoble-alpes.fr
}

\begin{document}

\maketitle

%------------------------------------------------------------
\section{Introduction}\label{sec:intro}

Ce document a pour but de présenter un compte rendu des expérimentations menées
dans le cadre du projet \assi afin de vérifier des propriétés de
\safety via des outils de vérifications de protocoles cryptographiques.
L'objectif est de proposer une méthode d'analyse de propriétés de \safety en
présence d'attaquants malicieux.
Dans l'écosystème actuel des outils, on distingue d'une part les outils de
vérification de protocoles cryptographiques qui codent en dur un attaquant
Dolev-Yao.
Ces outils se focalisent généralement sur des propriétés de sécurité telles que
le secret, l'authentification ou plus recement l'équivalence observationnelle.
Cependant, les outils les plus récents adoptent une expressivité rendant
possible la vérification de propriétés de \safety.

%------------------------------------------------------------
\section{Différences entre les protocoles cryptographiques et les \sis}\label{sec:diff}

\TODO{Définir systèmes indus par rapport aux protocoles crypto (états,
sessions, terminaison, propriétés, nb messages, etc).}

%------------------------------------------------------------
\section{Expérimentations}\label{sec:exp}

\TODO{Définir exemple bouteilles, voir annexe}

\noindent\TODO{Le formaliser en s'aidant des modélisations \proverif et \tamarin}.

L'exemple consiste d'un client, un serveur un procédé.
Le procédé est constitué de variables et d'un automate caractérisant les
interactions entre ces variables.



Un premier canal relie le client au serveur et un second canal relie le serveur
au procédé.
Le client, le serveur et le procédé sont des automates.
Chaque transition de ces automates est un processus.

%------------------------------
\subsection{\proverif}\label{sec:exp_proverif}

En \proverif, 

\end{document}
